% !TEX root = main.tex
\subsection{Données mesurées}

Si le procédée de tambour en rotation a pu être simulé via des méthodes numériques, la validation et la compréhension du procédé se voient renforcés par l'utilisation acru de méthodes de mesure durant la phase de fonctionnement. Issue de capteurs, nous les casserons en deux types. Tout d'abord l'instrumentation sans contact \textit{Off-Shell}, qui ne sont pas solidaire du tambour en rotation. Dans cette famille de capteurs, on trouve des mesures par imagerie. Via l'utilisation d'une paroi tranparente, l'écoulement peut être filmé afin de pouvoir analyser le déplacemement milieu granulaire et des corps broyants. De plus, les mesures par Correlation d'Image comme la méthode de \texit{Particle Image Velocimetry} (PIV) permettent, par l'analyse des variations de motifs ou de structures entre des images successives, d'extraire des informations sur les déplacements, et ainsi d'en déduire le champ de déplacement. Ainsi, dans \cite{jarray_wet_2019} des caméras ont permis d'extraire les profils de température granulaire et de vitesse d'écoulement pour un tambour en rotation. Egalement, dans \cite{Adepu}, une caméra thermique permet qu'en a elle de mesurer la température lors de l'écoulement.
Une autre classe de mesure se base sur des mesures accoustiques. En particulier dans les travaux de \cite{Owusu} des mesures accoustiques ont été réalisées sur un broyeur à boulets pour prédire la broyabilité du minerai et ajuster les densités de pulpe, montrant que l'émission acoustique varie selon la dureté des matériaux broyés.
Finalement, des mesures vibratoires peuvent être acquis instrumentant des corps broyants pour évaluer la charge solide à l'intérieur du broyeur à boulets , en capturant les signaux d'accélération à l'aide d'un accéléromètre triaxial intégré \cite{Wang}.

Une seconde classe d'instrumentation consiste à réaliser des mesures avec contact (\textit{On-Shell}) c'est à dire directement sur la paroi du tambour. Dans ce cadre, l'analyse de données vibratoires permettent de mieux comprendre la position charge du broyeur, son niveau de charge global ainsi que l'impact sur les revêtements \cite{Davey}.
