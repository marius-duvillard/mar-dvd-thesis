% !TEX root = main.tex

\subsection{Material Point Method (MPM)}

\subsubsection{Interprétation 1}

\subsubsection{Interprétation 2}

\subsubsection{Schéma classique}

La méthode MPM est implémentée généralement en trois phases. Dans un premier temps, les quantités définies sur les particules sont transférés sur les noeuds de la grille \textit{p2g}. Le principe fondementale de la dynamique est alors résolu permettant de déterminer une grille déformée. Finalement, les nouvelles quantités nodales permettent de mettre à jour les quantités particulaires dans une phase de transfert grille à particule (\textit{g2p}).

\paragraph*{p2g}

La grille de positions de noeuds $x_I$ est initialisée avec des valeurs nulles.

La masse $m_p$,la quantité de mouvement $m_p \bm v_p$ et les forces $\bm f_p$ sont transférées à la grille à l'aide des fonctions de forme $\phi_I$ associé à chaque noeuds

\begin{eqnarray*}
    m_I = \sum_p \varphi_{Ip}~ m_p, \\
    m_I \bm v_I  =  \sum_p \varphi_{Ip}~ m_p \bm v_p, \\
    \bm f_I  =  \sum_p \varphi_{Ip}~  \bm f_p. \\
\end{eqnarray*}

Des transferts plus complexes capable de préserver les moments angulaires ont été développé comme APIC~\cite{jiang_affine_2015}, Poly-PIC~\cite{fu_polynomial_2017}, et MLS-MPM~\cite{hu_moving_2018}.

\paragraph*{Mise à jour sur la grille}
La grille à chaque étape est initialisée dans un état non déformée. A l'aide du principe fondamentale de la dynamique, la vitesse sur la grille est mise à jour de manière explicite tel que

\begin{eqnarray*}
    m_I \bm a_I &=& \bm f_I + \bm f_g, \\
    m_I \bm v^{n+1} &=& \bm v^{n} + \Delta t~ (\bm f_I + \bm f_g) / m_I, \\
    \bm x_I^{n+1} &=& \bm x_I^{n} + \Delta t~ \bm v^{n+1}.
\end{eqnarray*}

C'est durant cette étape que les conditions limites ou les collisions avec un objet peuvent être prise en compte.

\paragraph{g2p}

Les particules vont suivre la déformation de la grille. Cela aura deux conséquence : La mise à jour de la matrice de déformation $\bm F_p$ et de leurs positions $\bm x_p$ et leur vitesses $\bm v_p$.

La mise à jour de $\bm F_p$ est réalisé avec la déformée de la grille $ x_I^{n+1}$ de manière implicite en utilisant $\bm v^{n+1}$ de telle sorte que

\begin{equation*}
    \bm F_p^{n+1} = \left( \bm I + \Delta t \sum_I \bm v_I^{n+1} (\nabla \varphi_{Ip}^T)\right) \bm  F_p^{n}.
\end{equation*}

En ce qui concerne l'étape d'advection des particules, le schéma PIC suggérait l'interpolation des vitesses tel que

\begin{equation*}
    \bm v_{PIC}^{n+1} = \sum_I \varphi_{Ip} \bm v_I^{n+1}
\end{equation*}.

Si ce schéma est stable, il est toutefois dissipatif. Inversement la mise à jour FLIP propose de mettre à jour la vitesse $\bm v_{PIC}^{n}$ en interpolant l'accélération tel que

\begin{equation*}
    \bm v_{FLIP}^{n+1} = \bm v_{p}^{n} \sum_I \varphi_{Ip} (\bm v_I^{n+1} - \bm v_I^{n})
\end{equation*}.

Dans ce cas, le transfert est concervatif mais instable. Ainsi, il est recommandé d'utiliser pour mettre à jour la vitesse $\bm v^{n+1}$ une combinaison linéaire des deux formulations tel que

\begin{equation*}
    \bm v_{p}^{n+1} = \alpha \left(\bm v_{p}^{n} \sum_I \varphi_{Ip} (\bm v_I^{n+1} - \bm v_I^{n})\right) + (1- \alpha)\sum_I \varphi_{Ip} \bm v_I^{n+1}
\end{equation*}~avec $\alpha \in [0, 1]$.

Les schémas de type APIC,PolyPIC ou MLS-MPM, utilisant de plus un transfert du gradient de $v$, utilise une mise à jour PIC tout en restant conservatif.

La position est elle mise jour en interpolant la déformation de la grille de telle sorte que

\begin{equation*}
    \bm x_p^{n+1} = \bm x_p^{n} + \sum_I \varphi_{Ip}~\bm v^{n+1}
\end{equation*}

Finalement, la grille de calcule peut être effacée et réinitialisée.

La force interne de la particule $\bm f_p$ dépend de la loi de comportement qui lui est associée.

Elle dépends généralement de la contrainte $\bm \sigma_p$ qui peut être mise à jour au début ou à la fin du schéma donnant deux formulations différence USF (\textit{Update Stress First}) et USL (\textit{Update Stress Last}).