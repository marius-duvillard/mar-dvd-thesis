% !TEX root = main.tex

\section{Introduction}


Numerical simulation enables the assessment of complex real-world systems, for instance, to facilitate the optimization of complex systems and perform risk analysis, all while reducing experimental costs. Thanks to the increasing computational resources, they help in understanding and designing processes, particularly in the mechanical field.
The solid and fluid mechanics historically leaned on grid-based or mesh-based methods. These techniques necessitate the use of structured meshes. The shift towards meshless methods offers significant promises for complex physics or large deformations (moving interfaces, material disintegration, or distortion) to avoid computing complex geometries.

Meshless methods, specifically particle-based methods, describe geometry as a collection of particles that move with the deformation flow in a Lagrangian fashion. Each particle transports material properties and internal variables. Particles can discretize a continuum medium and are associated with a kernel to reconstruct continuous fields and differential operators. In this article, we will mainly focus on the Vortex Method \cite{cottet_vortex_2000,mimeau_review_2021}, which discretizes the vorticity field and solves the incompressible fluid flow equation with the vorticity-stream function formulation.
\newline

The computed solution may contain errors that need to be understood, quantified, and minimized. If observations are available, integrating this information can lead to a more accurate estimation of the simulation state. In this context, data assimilation techniques offer an optimal way to combine various sources of information, resulting in a more precise estimation of the system state. Integrating model predictions and observational data has been widely applied in disciplines such as meteorology, oceanography, hydrology, and geosciences~\cite{bocquet_introduction_2014}.

In the domain of data assimilation, two prominent families of approaches have emerged: variational and stochastic methods. Variational approaches \cite{variational_method} focus on minimizing a cost function that measures the misfit between model predictions and observations, seeking the optimal system state. The most common formulations derive from 3D-Var, 4D-Var~\cite{talagrand1997assimilation}.

On the other hand, stochastic approaches go beyond mere state estimation; they delve into the quantification of uncertainty associated with the estimated states. Uncertainty quantification is a critical aspect, especially in dynamic and uncertain systems, where acknowledging and characterizing becomes paramount for reliable decision-making and model improvement. In this case, the estimate is sequentially updated based on previous and current observations. The assimilation process is performed through a Bayesian framework with a forecast and an analysis step. The Kalman filter~\cite{kalman_new_1960} is an example of a sequential formulation considering a linear model and Gaussian distribution assumptions. However, more advanced filters have been introduced to be adapted to nonlinear and arbitrary distributions. One of the most popular Bayesian filters is undoubtedly the Ensemble Kalman Filter, introduced by Evensen~\cite{evensen_sequential_1994}, primarily because of its adaptability to high-dimensional problems with any evolution model and its remarkable resilience to deviations from the initial Gaussian assumptions. It consists of approximating the probability distribution of a state thanks to an ensemble of simulations called particles or members. \newline

This paper introduces new approaches to applying ensemble Data Assimilation techniques to meshless simulations that discretize a continuum domain. The hypothesis considers several members of different particle distributions. The Ensemble Kalman Filter (EnKF) has been extensively employed for Eulerian discretization frameworks. However, its application in the Lagrangian approach presents unique challenges. These issues primarily revolve around defining a unified state representation across all ensemble members and effectively updating this state during the analysis phase.

When particle-based methods discretize a field on a continuum discretization, the particles are point entities and thus allow a certain flexibility to the update. Operations such as agglomeration, splitting, or resampling are utilized to update particle configurations, primarily to mitigate issues like distortion, excessive deformation or to manage particle count~\cite{yue_continuum_2015,cottet_multi-purpose_1999}.

Nevertheless, the crux of the challenge lies in the inherent disparity in discretization across different ensemble members. The first solution is to consider a reference discretization for all members. In fixed-grid methodologies with Multi-Resolution Analysis (MRA) and moving mesh scenarios, the state definition on varied grids with assimilation is managed through projection and interpolation techniques to establish a reference grid for state updates \cite{siripatana_combining_2019,bonan_data_2017}. The selection of the reference and updated grids provides a spectrum of implementation possibilities. Furthermore, Siripatana et al. \cite{siripatana_combining_2019} highlight that the EnKF correction is contingent solely upon the predictions and observations, thereby rendering it independent of the state definition.

Another solution consists of defining the state with the union of the particles, considering the position and associated intensities of each particle~\cite{darakananda_data-assimilated_2018}. Complex filters have been developed to estimate correctly the posterior discretization with a nonlinear observation model or with a deficient number of sensors of preassure~\cite{le_provost_low-rank_2021}.
However, these methods grapple with scenarios involving markedly divergent particle discretizations or highly variant model flows. In this general case, using a particle state for all particles for the update is unfeasible. Indeed, the update implies a linear combination of all members, leading to an exponential increase of particles.
On the other hand, the state could be associated with the spatial field defined in a functional space. The updated fields could be evaluated on the entire domain. Finally, using approximation or regression, a new particle discretization could be approximated. These final modifications have already been introduced in the Vortex Method to better approximate the vorticity field by changing the particle intensities regroup under the label Meshless Rezoning Methods in~\cite{mimeau_review_2021}. It mainly involved iterative methods~\cite{beale_accuracy_1988},  triangulation~\cite{russo_1994} or Radial Basis Function (RBF) interpolation~\cite{barba_lorena_a_vortex_2004,sperotto_2021}. RBF offers to introduce new particles or introduce penalization to regularize optimization problems.\newline

Based on those different tools, we propose two novel EnKF-based filters. First, the Remesh-EnKF uses an Eulerian intermediate update. This way, the analysis is defined on an Eulerian framework, and the method is reduced to the previous development of data assimilation. The projection step is something familiar for Particle-In-Cell methods~\cite{sulsky_particle_1994,brackbill_flip_1988}. Nevertheless, it involves remesh the discretization entirely on a regular grid of particles. This method is based on the regridding of the particle discretization as described by~\cite{cottet_multi-purpose_1999}.
Then, in a case where the particle discretization would be preserved, the Particle-EnKF is introduced. In this case, the analyzed field is approximated on the previous particle discretization of each members. The particle positions are unchanged; only the strengths are modified by regression or approximation of the analyzed solution.
In the next part, background on sequential filtering and EnKF algorithm will be introduced \ref{Background_DA}, then on particle-based methods \ref{Background_Part}. Then, the two categories of method will be described in section \ref{Methods}. Afterward, those filters will be compared with a grid-based filter in a 1D Advection-Diffusion problem in section \ref{App_1D}, and an incompressible viscous flow is solved using a Vortex Method \ref{App_2D} where the filters are quantitatively analyses.






