% !TEX root = main.tex

\appendix
\section{Stochastic Ensemble Kalman Filter}~\label{appendix:enkf}

We define the matrix of states and the matrix of anomalies $\mstate^f = [\state^1, \dots, \state^N]$, $\annomX$ whose columns are the member states and the normalized anomalies.

\begin{equation*}
    \annomX = \frac{1}{\sqrt{N - 1}}(\X^f - \overline{\state}^f \bm{1}^T),
\end{equation*}where$\bm{1} \in \mathbb{R}^N$ is a vector of one.

Respectively the matrix of observation and observation anomalies are $\mathcal Y^f = [\mathcal{H}(\state_1^f), \dots, \mathcal{H}(\state_N^f)]$, $\annomY$ where columns are

\begin{equation*}
    \annomY = \frac{1}{\sqrt{N - 1}} \left(\mathcal Y - \overline{\obs} \bm{1}^T \right) \quad \text{with} \quad \overline{\obs} = \frac{1}{N} \sum_{j=1}^{N} \mathcal{H}(\state_j^f).
\end{equation*}

The ensemble defines the covariance between states and observations $\Cov \bm H^T$, the covariance between observations $\Cov \bm H^T$, and $\tilde{\bm{K}}$

\begin{eqnarray*}
    \Cov \bm H^T &\simeq& \frac{1}{N - 1} \sum_{i = 1}^{N} {(\state_i^f - \overline{\state}^f)} {\left[ \mathcal{H}_k(\state_i^f) - \overline{\bm{y}}\right]}^T = \annomX \annomY^T, \\
    \bm H \Cov \bm H^T &\simeq& \frac{1}{N -1} \sum_{i = 1}^{N}\left[ \mathcal{H}_k(\state_i^f) - \overline{\bm{y}}\right] {\left[ \mathcal{H}_k(\state_i^f) - \overline{\bm{y}}\right]}^T = \annomY \annomY^T,\\
    \tilde{\bm{K}} &=& \Cov \bm H^T{(\bm H \Cov \bm H^T + \bm R)}^{-1} = \annomX \annomY^T {(\annomY \annomY^T + \bm R)}^{-1}.
\end{eqnarray*}

This observation matrix-free implementation rely on the secant method approximation $\mathcal{H}(\state_i^f - \overline{\state}^f) \approx \predi - \overline{\obs}$.
The forecast is then update to a posterior ensemble $\left\{\state_i^a\right\}_{i=1}^{N}$ such as

\begin{equation} \label{enkf_formula}
    \mstate^a = \mstate^f + \tilde{\bm{K}} ( \mdata - \mpred),
\end{equation}where ${[\mdata]}_i = \obs + \bm{\varepsilon}_i$ is the perturbed observation with $\bm{\varepsilon}_i \sim \mathcal{N}(\bm{0}, \bm R) $, $\tilde{\bm{K}}$ the ensemble Kalman gain matrix and $( \mdata - \mpred)$ the innovation term.
The forecast step is then applied to the analyzed ensemble until the following observation.
Based on this formulation, we can deduce a correction formula only based on the member's predictions and observations.

We can rewrite the classical update formula using the previous anomaly matrices.

\begin{equation*}
    \mstate^a = \mstate^f + \annomX \annomY^T {({\annomY \annomY^T + \bm R})}^{-1}(\mdata - \mpred)
\end{equation*}

We reformulate the correction term by remarking that $ \bm{1}^T  \annomY^T = \bm{0}$. We define $\Fcorr$, the correction matrix that gives the update in terms of linear combinations of the forward states

\begin{equation*}
    \mstate^aa = \mstate^f + \mstate^f \Fcorr, \quad \Fcorr = \frac{1}{\sqrt{N - 1}}\annomY^T {(\annomY \annomY^T + \bm R)}^{-1}(\mdata - \mpred).
\end{equation*}

using the Sherman-Morrison-Woodbury formula we obtain

\begin{equation*}
    \Fcorr = \frac{1}{\sqrt{N - 1}} {(\bm I_N + \annomY^T\bm R^{-1}\annomY)}^{-1}\annomY^T \bm R^{-1} (\mdata - \mpred).
\end{equation*}


\section{Moment conservation of particle discretization}~\label{appendix:moment_conservation}

The $m$-th moment of a particle distribution is defined as the quantity $\sum_{p} z_p^{\alpha} \bm{U}_p$.

First, we see that the partition of unity is required

\begin{equation}~\label{eq:unity1}
    \sum_{I \in \Lambda} W\left(\frac{z - z_I}{\ell_I}\right) = 1 ,\quad z \in \Omega
\end{equation}~due to the final particle arrangement $\mathcal{P'}$ on a grid of size $d_p$, it leads to the following property

\begin{equation}~\label{eq:unity2}
    \sum_{p'\in\mathcal P'} W\left(\frac{z - z_{p'}}{\ell_I}\right) = \frac{V_I}{V_p'},\quad z \in \Omega.
\end{equation}.

Attention should be focused on the border. Extending the domain with "ghost" particles or nodes allows for verification of properties within $\Omega$.

This property is the necessary condition for the conservation of the first moment. Primarily for the assignment

\begin{gather}
    \begin{align*}
        \sum_{I \in \Lambda} \bm u_I V_I & = \sum_{p \in \Lambda} \bm U_p  W \left(\frac{z_I - z_p}{\ell_I} \right)                                                           & \\
                                         & = \sum_{p \in \mathcal P} \bm U_p \sum_{I \in \Lambda} W \left(\frac{z_I - z_p}{\ell_I} \right) = \sum_{p \in \mathcal P} \bm U_p. &
    \end{align*}
\end{gather}~using the property \eqref{eq:unity1}. Secondary, for the interpolation process

\begin{gather}
    \begin{align*}
        \sum_{p' \in \mathcal P'} \bm U_{p'} = \sum_{p' \in \mathcal P'} \bm u_g(z_{p'}) V_{p'} & = \sum_{p' \in \mathcal P'} V_{p'} \sum_{I \in \Lambda} \bm u_I W \left(\frac{z_{p'} - z_I}{\ell_I}\right)                                                &   \\
                                                                                                & =  \sum_{I \in \Lambda} \bm u_I  V_{p'}\sum_{p' \in \mathcal P'} W \left(\frac{z_{p'} - z_I}{\ell_I}\right)                                               &   \\
                                                                                                & =  \sum_{I \in \Lambda} \frac{V_I}{V_p'} V_{p'} \bm u_I =                            \sum_{I \in \Lambda} \bm u_I V_{I} = \sum_{p \in \mathcal P} \bm U_p & ,
    \end{align*}
\end{gather}~using equation~\eqref{eq:unity2}.

It can be shown moreover that if for $1 \leq |\alpha| \leq m - 1$, $W$ satisfies,

\begin{equation}
    \sum_{I \in \Lambda} {(\bm z-\bm z_I)}^\alpha W \left(\frac{\bm z - \bm z_I}{\ell_I} \right) = 0, \label{eq:momentProperty}
\end{equation}

The regridding procedure will be ordered at $m$. Equivalently, the previous equality lead, for $0 \leq |\alpha| \leq m - 1$, to
\begin{equation*}
    \sum_{I \in \Lambda} \bm z_I^\alpha W \left(\frac{\bm z_p - \bm z_I}{\ell_I} \right) = \bm z^\alpha,
\end{equation*}~obtained by developing ${(\bm z-\bm z_q)}^\alpha$ and using a recurrence on previous orders. This means that the interpolation is exact for polynomials of degrees less or equal to $m-1$ or that the moment of order $m-1$ is conserved.

\section{Parameters}~\label{appendix:simulation-parameters}

\subsection{One dimension problem}
\begin{table}[htbp]
    \centering
    \caption{Ensemble generation variables}
    \begin{tabular}[t]{|l|l|}
        \hline
        Variables                   & Distributions                              \\
        \hline
        Gaussian mean               & $Z_m \sim  \mathcal{N}(\meanZm, \sigmaZm)$ \\
        Gaussian standard deviation & $S_m \sim\mathcal{U}(\smLow, \smUp)$       \\
        velocity                    & $v \sim \mathcal{N}(\vmean, \vstd)$        \\
        diffusion                   & $D \sim \mathcal{U}(\Dlow, \Dup)$          \\
        \hline
    \end{tabular}
    \label{tab:ens_gen_1d}
\end{table}

\subsection{Two dimension problem}
\begin{table}[htbp]
    \centering
    \caption{Reference parameters}
    \begin{tabular}[t]{|l|l|}
        \hline
        Parameters            & Values                                                       \\
        \hline
        reference viscosity   & $v_{\text{ref}} = 0.001$                                     \\
        reference orientation & $\theta_{\text{ref}}  = \frac{7 \pi}{8} (\text{rad.})$       \\
        barycenter position   & $\bm{z}_{\text{ref}} = \left[\frac\pi2, \frac\pi2 \right]^T$ \\
        translation velocity  & $U_{\text{ref}} = 0.25$                                      \\
        \hline
    \end{tabular}
    \label{tab:ref}
\end{table}

\begin{table}[htbp]
    \centering
    \caption{Nominal assimilation and simulation parameters}
    \begin{tabular}[t]{|l|l|}
        \hline
        Parameters                      & Values                                 \\
        \hline
        time step                       & $dt = 0.005$                           \\
        final time                      & $t_f =10$                              \\
        std. observation                & $\sigma_{obs} =  5.0e^{-2}$            \\
        vorticity threshold             & $\varepsilon_{\omega} = 1.0e^{-4}$     \\
        particle characteristic length  & $dp = \frac{pi}{256} \approx 0.01227 $ \\
        smoothing length                & $h = 2.0 dp$                           \\
        number of assimilation          & $N_{\text{assim}} = 10$                \\
        ensemble size                   & $N_{\text{ens}} = 32$                  \\
        number of observation           & $N_{\text{obs}} = 12^2 = 144$          \\
        grid discretization             & $N_{\text{grid}} = 65^2 = 4225$        \\
        number of remeshing by forecast & $N_{\text{remesh}} =  2 $              \\
        \hline
    \end{tabular}
    \label{tab:simu_2d}
\end{table}

\begin{table}[htbp]
    \centering
    \caption{Ensemble generation variables}
    \begin{tabular}[t]{|l|l|}
        \hline
        Variables   & Distributions                                                                                                           \\
        \hline
        radius      & $R \sim \cN(1.0, 0.05^2)$                                                                                               \\
        orientation & $\theta \sim \cU\left(\pi, \frac\pi2 \right) (\text{rad.}) $                                                            \\
        barycenter  & $z_{\text{mean},x} \sim \cN\left(\frac\pi2,0.1^2\right), \quad z_{\text{mean},y} \sim \cN\left(\frac\pi2,0.1^2\right) $ \\
        velocity    & $U \sim \cU(0, 0.5^2) $                                                                                                 \\
        viscosity   & $v \sim \cN(0.0015, 0.0005^2)$                                                                                          \\
        \hline
    \end{tabular}
    \label{tab:ens_dipole}
\end{table}

\newpage