% !TEX root = main.tex
\newpage

\section{1D density advection-diffusion problem}~\label{App_1D}
\subsection{Description of the problem}

An initial exploration is conducted on a one-dimensional application to assess the filter performance. We define the following one-dimensional $2\pi$-periodic convection-diffusion problem such as
\begin{equation*}
	\frac{\partial u}{\partial t}(z,t) + v \frac{\partial u}{\partial z}(z,t)  = \visc \frac{\partial^2 u}{\partial z^2}(z,t),
\end{equation*}
with $z$ the spatial coordinate, $v$ a constant velocity and $\visc$ a constant diffusion coefficient.
For the following application, the reference solution will use $v = \refv$ and $\visc = \refvisc$ as parameters.
We define the $2\pi$-periodic heat kernel in one dimension, such as

\begin{equation*}
	\phi(u, s) = \sum_{k=-\infty}^{\infty} \frac{1}{\sqrt{4 \pi s}} \exp{\left(-\frac{{(u - 2\pi k)}^2}{4s} \right)}.
\end{equation*}

Considering an initial condition characterized by a Gaussian shape expressed as $u^{gt}(z, 0) = \phi(z-z_0, Dt_0)$, where $\zz$, $t_0 = \frac{\sigma_0^2}{2D}$, and $\sigz$, we derive the comprehensive analytical solution utilizing the Green equation solution
\begin{equation*}
	u^{gt}(z, t) = \phi(z- v t - z_0, \visc (t+t_0)).
\end{equation*}The analytical solution is succinctly described as a Gaussian function, characterized by a mean that moves at the advection velocity and a standard deviation proportional to $t$ and $D$. This solution is visually depicted in Figure~\ref{fig:1d_analytical} across various assimilation time frames.

\begin{figure}[ht]
	\centering
	\includegraphics[width=\linewidth]{images/app1d/analytical_frame.pdf}
	\caption{The analytical solution of the convection-diffusion problem evolves over time, with the final snapshot revealing a complete spatial period.}
	\label{fig:1d_analytical}
\end{figure}

Following a Lagrangian perspective by tracking a fluid particle of position $z_p$ and intensity $U_p$, the equation becomes

\begin{equation*}
	\frac{dz_p}{dt} = v(z_p, t), \quad \frac{dU_p}{dt} = D \frac{d^2 U_p}{dz^2}
\end{equation*}

For solving the convection-diffusion scheme, we employ the two steps of the viscous splitting algorithm. The advection is taken into account by updating the position of the particle with an Euler explicit scheme.
On the other hand, we use a redistribution method called the Particle Strength Exchange Method (PSE)~\cite{degond_1989,cottet_1990} to approximate the laplacian term $\frac{d^2 U_p}{dz^2}$.


\begin{equation*}
	D \frac{d^2 U_p}{dz^2}  = D V_p \frac{d^2 u_p}{dz^2} \approx D V_p \varepsilon^{-d} \int [u(z)  - u(y)] \phi_\varepsilon(y - z) dz,
\end{equation*}

This leads, with the particle approximation, to a redistribution of the particles intensities in their previous locations, such as

\begin{equation*}
	\frac{dU_p}{dt} = D \varepsilon^{-d} V_p \sum_q (U_q - U_p) \phi_\varepsilon( z_q -  z_p),
\end{equation*}where $V_p$ the volume of the particle $p$ and $d$ the dimension of $\Omega$.
For further details on the computation, please refer to \cite{cottet_1990}.

For the periodic boundary problem described in section \ref{App_1D}, we define an equivalent kernel function $\phi_\varepsilon= \phi^P_g = \sum_{n=-\infty}^{+\infty} \phi_g(r - 2 \pi  n )$.

The particle-based model employs a discretization of \npart{} particles with a size of $h = \frac{L}{N_{\text{part}}}$ and a smoothing length of $\varepsilon = 1.3 h$.
For the sake of comparison, we solve the convection diffusion equation with an explicit central finite difference scheme discretized on a regular grid with \ngrid{} nodes.

\subsection{Assimilation parameters and ensemble generation}

\subsubsection{Ensemble distribution}
All filters undergo testing on an identical initial prior ensemble of size $N = 25$ members, characterized by Gaussian shapes that are shifted and scaled. The mean of the function and its standard deviation are sample. The total mass is set equal to 1. Parameter of velocity $v$ and the diffusion $D$ are also sampled. The different distribution are defined in Appendix~\ref{tab:ens_gen_1d}. Moreover the parameters samples and initial state are illustrated in Figure~\ref{fig:initial_gen}.

The observational data is subject to additive noise, denoted as $\eta \sim \mathcal{N}(0, \sigma_y \bm{I})$, where $\sigma_y = \sigmaY$ and $\bm{I}$ represents the identity matrix.

\subsubsection{Error definition}
We define the error as the following relative ratio

\begin{equation}~\label{eq:L2_error}
	e_{L_2} =\frac{ \left[\frac1\nens \sum_{i = 1}^{\nens} \int_\Omega \left(u_i(z) - u^{gt}(z)\right)^2 dz\right]^{1/2}}{\norm{u^{gt}}_{L_2}}
\end{equation}~where $u_i$ denote the $i$-th member of the ensemble and $\norm{u}_{L_2}$ denote the $L_2$ norm of $u^{gt}$.

The $L_2$ norm is computed using a quadrature over a regular grid of an ensemble of cells $\mathcal{C}$ such as for any $f \in L_2$

$$
	\norm{f}_{L_2}  = \int_{\Omega} f^2~d\Omega \approx \sum_{c \in \mathcal{C}} f(z)~V_c
$$~where $z_c$ is the center of the cell $c$ and $V_c$ the volume of the cell. The grid is still the same for all the simulations.
%We compute the parameter error with a norm-2 as $e_{\theta} = \frac{ \left[\frac1\nens \sum_{i = 1}^{\nens} \norm{\theta^{gt} - \theta_{i}}_2^2\right]^{1/2}}{\norm{\theta^{gt}}_2}$.

\subsubsection{Numerical parameters}

We conduct $N_{\text{assim}} = 30$ assimilation steps at evenly spaced intervals until the final time $t_f = 2 \frac{L}{v}$. During each assimilation step, the field $u^{gt}$ is observed at six regularly spaced positions $x_{\text{obs}}$.


In the particle-based simulation, fields are discretized using regularly spaced particles that are shifted. Intensity values are obtained by fitting an interpolation operator like in Section~\ref{interpOp} to the particle intensity.
The parameter $\varepsilon_{\text{mass}}$ is introduced as a cutoff for particle selection, allowing for the definition of varying numbers of particles for each simulation. The particle support poses challenges in the Part-EnKF as described in Section \ref{part_enkf}.

Simultaneously, a standard Ensemble Kalman Filter (EnKF) update is applied to the nodal variables to construct the reference filter Grid-EnKF which use a grid-based model. For grid-based simulation, the fields of each member are interpolated at the node locations. In this way, the ensemble generated is still the same for the sake of comparison.


\begin{figure}[ht!]
	\centering
	\begin{subfigure}{0.49\textwidth}
		\includegraphics[width=\textwidth]{images/app1d/param.pdf}
	\end{subfigure}
	\hfill
	\begin{subfigure}{0.49\textwidth}
		\includegraphics[width=\textwidth]{images/app1d/prior.pdf}
	\end{subfigure}
	\caption{On the left the initial parameters sample, $v$ in abscissa and $D$ in ordinate. On the right is the initial ensemble state.}
	\label{fig:initial_gen}
\end{figure}

\subsection{Results}

We compare the different filters on the assimilation of the state. We first compare the Grid-EnKF, the Remesh EnKF and two Part-ENKF filter with 100 and 60 particules. The parameter sample are still unchanged. We take unknown parameters into account as model uncertainties. The two filters outlined in the Method section~\ref{Methods} and the Eulerian filter are compared with the reference filter based on a grid discretization.

In figure \ref{fig:1d_error_time}, we appreciate different assimilation step for the Remesh-EnKF filter.


\begin{figure}
	\centering
	\begin{subfigure}{\textwidth}
		\includegraphics[width=\textwidth]{images/app1d/wo_calibration/remesh_EnKF.pdf}
	\end{subfigure}
	\caption{Data assimilation ower assimilation step for the Remesh-EnKF filter.}
\end{figure}

The result are quite similar for all the different filters, except the Part-EnKF with 60 particules.

\begin{figure}
	\centering
	\includegraphics[width=0.75\textwidth]{images/app1d/wo_calibration/state_error.png}
	\caption{State error with respect to assimilation time step.}
	\label{fig:1d_error_time}
\end{figure}

The primary issue arises from the regression on non-overlapping support, where the regression struggles to fit the analysis solution defined on a more considerable space support. This leads to heightened variability, particularly in the tail of the distribution. Addressing this common challenge in RBF Regression \cite{fornberg_flyer_2015} involves increasing the Ridge penalization coefficient, a parameter we choose through cross-validation Ridge regression.
Even with a more stable regression, it remains a projection of the analysis solution onto the forecast support. It is imperative to increase the number of particles to achieve a better approximation of the analysis solution using the particle approximation operator in Section~\ref{interpOp} or the regression operator in Section~\ref{regressionOperator}.

We validate this assumption by varying the initial support of particles. Quantitatively, as observed in Figure~\ref{error_support}, the error decreases with an increase in the number of particles. Moreover, qualitatively examining the snapshot on the right reveals that the solution closely aligns with the reference.

\begin{figure}
	\centering
	\begin{subfigure}{0.39\textwidth}
		\includegraphics[width=\textwidth]{images/app1d/error_support/error_support.png}
		\label{error_support1}
	\end{subfigure}
	\hfill
	% Revoir figures en plotly
	\begin{subfigure}{0.29\textwidth}
		\includegraphics[width=\textwidth]{images/app1d/error_support/ok.png}
		\label{error_support2}
	\end{subfigure}
	\hfill
	\begin{subfigure}{0.29\textwidth}
		\includegraphics[width=\textwidth]{images/app1d/error_support/not_ok.png}
		\label{error_support3}
	\end{subfigure}
	\caption{Left: Error with respect to particle support size, Middle: Final step for a support of 100 particles, Right: Final step for a support of 60 particles.}
	\label{error_support}
\end{figure}
However, Adding particles in a more complex solution is a challenging task. Indeed, a good spacing between particles and the density of particles has to be preserved. In this case, we advise defining criteria for the error of the reconstruction. Instead of adding particles, we advise generating a new, regularly spaced grid of particles to reconstruct the solution.

In conclusion, this example underscores the Remesh-EnKF filter capability to yield results comparable to the classical EnKF applied to a grid model. Additionally, it highlights the Part-EnKF capability in assimilating on a particle discretization while also emphasizing the importance of addressing spatial discrepancies between members, which can pose challenges in solution reconstruction. The computation of solution error reconstruction provides a straightforward criterion for remeshing a member and applying the analysis solution approximation.


\newpage