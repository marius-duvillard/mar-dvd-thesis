\documentclass[a4paper,12pt]{article}

% !TEX root = main.tex

\usepackage{xcolor}

\usepackage[T1]{fontenc}
\usepackage{amsmath}
\usepackage{amsfonts}
\usepackage{bm}
\usepackage{a4wide}
\usepackage{hyperref}
\usepackage{authblk}
\usepackage{subcaption}
\usepackage{graphicx}
\usepackage{enumitem}
\usepackage[ruled,vlined]{algorithm2e}
\setlength\parindent{0pt}
% \setcounter{algocf}{1}

\usepackage[standard]{ntheorem}
% \newtheorem{proposition}{Proposition}
\definecolor{mycustomcolor}{RGB}{128, 0, 128}
\newcommand{\mycolor}[1]{\textcolor{mycustomcolor}{#1}}
\newcommand{\lgcolor}[1]{\textcolor{red}{#1}}
\newcommand{\olmcolor}[1]{\textcolor{blue}{#1}}
\DeclareMathOperator*{\argmax}{arg\,max}
\DeclareMathOperator*{\argmin}{arg\,min}
\newcommand{\norm}[1]{\left\lVert #1 \right\rVert}
\newcommand{\nens}{N_{\text{ens}}}
\newcommand{\state}{\bm{z}}
\newcommand{\proj}{\bm{\Pi}}
\newcommand{\mstate}{\bm{Z}}
\newcommand{\obs}{\bm{y}}
\newcommand{\z}{\bm{z}}
\newcommand{\ngrid}{$N_{grid} = 100$}
\newcommand{\npart}{$N_{part} = 100$}
\newcommand{\predi}{\mathcal{H} (\bm z_i^f)}
\newcommand{\annomX}{\bm A}
\newcommand{\annomY}{\bm Y}
\newcommand{\data}{\bm{d}}
\newcommand{\mdata}{\bm{D}}
\newcommand{\Fcorr}{\bm{F}}
\newcommand{\mpred}{\mathcal{Y}}
\newcommand{\sigmaY}{0.05}
\newcommand{\sigmaZm}{0.5}
\newcommand{\meanZm}{\pi/2 + 0.6}
\newcommand{\smLow}{0.8}
\newcommand{\smUp}{1.2}
\newcommand{\tid}{k}
\newcommand{\visc}{D}
\newcommand{\GP}{\mathcal{GP}}
\newcommand{\bR}{\mathbb{R}}
\newcommand{\xs}{X^*}
\newcommand{\map}{\hat{f}}
\newcommand{\cL}{\mathcal{L}}
\newcommand{\pxs}{\pi^\star}
\newcommand{\bE}{\mathbb{E}}
\newcommand{\Om}{\Omega}
\newcommand{\cN}{\mathcal N}
\newcommand{\cU}{\mathcal U}
\newcommand{\cV}{\mathcal V}
\newcommand{\cP}{\mathcal P}
\newcommand{\refv}{1.0}
\newcommand{\refvisc}{0.05}
\newcommand{\Dlow}{0.02}
\newcommand{\Dup}{0.08}
\newcommand{\vmean}{0.9}
\newcommand{\vstd}{1.2}
\newcommand{\X}{\bm{X}}
\newcommand{\Y}{\bm{Y}}
\newcommand{\Cov}{\bm{P}}
\newcommand{\sigz}{\sigma_0^2 = 0.5}
\newcommand{\zz}{z_0 = 0.02}


\title{Ensemble Data Assimilation for Meshless Methods}
\author[1,2]{Marius Duvillard}
\author[1]{Loïc Giraldi}
\author[3]{Olivier Le Maître}
\affil[1]{CEA, DES, IRESNE, DEC, SESC, LMCP, Cadarache, F-13108 Saint-Paul-Lez-Durance, France}
\affil[3]{CNRS, Inria, Centre de Mathématiques Appliquées, Ecole Polytechnique, IPP, Route de Saclay, 91128, Palaiseau Cedex, France}
\affil[2]{Centre de Mathématiques Appliquées, Ecole Polytechnique, IPP, Route de Saclay, 91128, Palaiseau Cedex, France}
\date{}

% \usepackage[
%     backend=biber,
%     style=alphabetic,
%     sorting=ynt
% ]{biblatex}
% \usepackage{biblatex}
% \addbibresource{biblio.bib}
% \bibliographystyle{elsarticle-num}
\bibliographystyle{plain}
% \bibliographystyle{ACM-Reference-Format}


\begin{document}
% \begin{frontmatter}
\maketitle

\begin{abstract}
    This study presents a novel approach for integrating data assimilation techniques into meshless simulations using the Ensemble Kalman Filter. If data assimilation methods have been apply on Eulerian simulations for long, there have never been properly used in the context of a Lagrangian solution discretization. Two specific methodologies are introduced to complete the analysis. The first one is based on the use of an intermediary Eulerian transformation combining a projection and a remeshing process. The second is a purely Lagrangian  scheme useful when remeshing is not adapted. These methods are evaluated using a one-dimensional advection-diffusion model with periodic boundaries. Subsequently, assimilation schere are applied to a non-linear two-dimensional inviscid flow problem, solved via the Vortex-In-Cell method. In the one-dimensional scenario, the performance of these filters is benchmarked against a grid-based assimilation filter. In the two-dimensional case, the study demonstrates the feasibility of applying these methods in more intricate scenarios.

\end{abstract}

{\bf Keywords:} Meshless Methods, Particle-based Method, Data Assimilation, EnKF, Ensemble Methods, Vortex Methods.
% \begin{keyword}
%     meshless methods \sep data assimilation \sep EnKF \sep Ensemble Methods 
% \end{keyword}

% \end{frontmatter}

\tableofcontents
% !TEX root = main.tex

\section{Introduction}


%%%% PLAN INTRODUCTION %%%%

% CONTEXT

%% Contexte industriel
%% La simulation de manière générale
% - La simulation numérique est là pour permettre de faire des prédictions de système réel complexe pour par exemple permettre l'optimsation de systèmes complexes, faire de l'analyse de risque, tout en réduisant les couts expérimentaux\dots
% - En mécanique, on utilise principalement la méthode des éléments finis pour la simulation des solides et la méthode des éléments
% The traditional method for numerical analysis has long been reliant on finite element, finite volume, and finite difference techniques. These methods require a prThis transition to meshless methods holds immense potential, especially in fields like
%%% La simulation lagrangienne plus spécifiquement --> parlé de SPH, VM, ...
%% La quantification de manière générale --> Parler inférence, s'inspirer du papier de Siripatana
%%% Parler de l'assimilation de données --> parler de manière générale de EnKF, méthodes variationelles, optimal interpolation, Nudging


Numerical simulation enables the assessment of complex real-world systems, for instance, to facilitate the optimization of complex systems and perform risk analysis, all while reducing experimental costs. Thanks to the increasing computational resources, they are a tool to understand and design processes, particularly in the mechanical field.
The conventional approach to numerical analysis has historically leaned on grid-based methods. These techniques necessitate the use of structured meshes. The shift towards meshless methods offers significant promises for complex physics or large deformations (moving interfaces, material disintegration, or distortion) to avoid computing complex geometries.

Meshless methods, specifically particle-based methods, describe geometry as a collection of particles that move with the deformation flow in a Lagrangian fashion. Each particle transports material properties and internal variables.
On one hand, particles can represent a discrete medium. Particles are individual entities with kinematic properties that interact locally and balance multi-body equilibrium. The Discrete Element Method (DEM), first introduced by Condall~and~Strack~\cite{cundall_discrete_1979}, has gained much popularity in modeling granular materials.
On the other hand, particles can discretize a continuum medium and are associated with shape functions to reconstruct continuous fields and differential operators. The Smoothed Particle Hydrodynamic (SPH), independently introduced by Gingold, Monaghan, and Lucy~\cite{gingold_monaghan_sph_1977,lucy_1977}, is one of the first continuum particle-based methods. It associates a kernel to each particle to approximate the continuous fields and the derivative operator to solve the strong form of the equilibrium equation. It has first been applied to stellar models but also to fluid dynamics. The Material Point Method (MPM) introduced by Sulsky~\cite{sulsky_particle_1994} is another particle-based that is part of the Particle-In-Cell family, like Fluid Implicit Particle~\cite{brackbill_flip_1988} (FLIP) introduced an auxiliary grid to project, solve, and interpolate back the solution on the particles. \newline

The computed solution involves errors that must be understood, quantified, and reduced.
Uncertainty is a fundamental aspect of scientific inquiry and modeling. It often arises when our knowledge is limited or incomplete. This uncertainty can manifest in various forms, such as the ambiguity surrounding the value of a model parameter, the vagueness regarding initial conditions, or the uncertainty in setting boundary conditions or external forces. Moreover, if numerical models usually bring essential physical principles, they involve some simplifications. The numerical error appears due to the algorithm and discretization.
Besides, it extends to the uncertainty associated with forthcoming experimental measurements calibrating numerical models. \newline

Data assimilation is a crucial methodology in scientific research, especially in complex and chaotic systems. Its fundamental purpose lies in combining different sources of information to obtain a better estimate of the system state. Integrating model predictions and observational data has found extensive application in disciplines such as meteorology, oceanography, hydrology, and geosciences~\cite{bocquet_introduction_2014}.

In the domain of data assimilation, two prominent families of approaches have emerged: variational and stochastic methods. Variational approaches focus on optimizing a cost function that measures the misfit between model predictions and observations, seeking the most plausible estimate of the system state. The most commonly used formulations are 3D-VAR, 4D-VAR, and incremental 4D-VAR.
On the other hand, stochastic approaches go beyond mere state estimation; they delve into the quantification of uncertainty associated with the estimated states. This is a critical aspect, especially in dynamic and uncertain systems, where acknowledging and characterizing uncertainty becomes paramount for reliable decision-making and model improvement. This is particularly crucial in dynamic and uncertain systems, where acknowledging and characterizing uncertainty becomes paramount for reliable decision-making and model improvement. In this approach, the estimate is sequentially updated based on previous and current observations. The assimilation process is performed through a Bayesian framework with a forecast and an analysis step. The Kalman filter~\cite{kalman_new_1960} is an example of a sequential formulation considering a linear model and Gaussian distribution assumptions. However, more advanced filters have been introduced to be adapted to nonlinear and arbitrary distributions. One of the most popular Bayesian filters is undoubtedly the Ensemble Kalman Filter introduced by Evensen~\cite{evensen_sequential_1994} mainly due to its adaptability to high dimensional problems with any evolution model. It consists in approximating the probability distribution of a state thanks to an ensemble of simulations called particles or members. \newline

The goal of this paper is to introduce new approaches to apply ensemble Data Assimilation techniques to particle-based simulation that discretize a continuum domain. The hypothesis considers several members of different particle distributions. The Ensemble Kalman Filter (EnKF) has been extensively employed for Eulerian discretization frameworks. However, its application in the Lagrangian approach presents unique challenges. These primarily revolve around defining a unified state representation across all ensemble members and effectively updating this state during the analysis phase.

Meshless methods are more or less sensible to these issues. For particle-based methods like the Discrete Element Method~\cite{cundall_discrete_1979} (DEM), the update phase is challenging due to interpenetration issues. For the classic soft-sphere approach introduced, the interaction is dependent on the geometry of the particle. Moving one particle to another leads to a complex global nonlinear optimization problem. In~\cite{chen_superfloe_2022}, an EnKF algorithm has been applied to a DEM simulation to study the Sea Ice flow. However, some simplifications have been introduced, like a new parametrization to reduce the number of particles, and changing particle positions have mild stability implications.\newline

In contrast, particle-based methods in continuum discretization treat particles as point entities, thereby circumventing interpenetration issues. Common operations such as agglomeration, splitting, or resampling are utilized to update particle configurations, primarily to mitigate issues like distortion, excessive deformation or to manage particle count~\cite{yue_continuum_2015,cottet_multi-purpose_1999}.

Nevertheless, the crux of the challenge lies in the inherent disparity in discretization across different ensemble members.
The first solution is to consider a reference discretization for all members.
In fixed-grid methodologies with Multi-Resolution Analysis (MRA) and moving mesh scenarios, the state definition on varied grids with assimilation is managed through projection and interpolation techniques to establish a reference grid for state updates \cite{siripatana_combining_2019,bonan_data_2017}. The selection of the reference and updated grids provides a spectrum of implementation possibilities. Furthermore, Siripatana et al. \cite{siripatana_combining_2019} elucidate that the EnKF correction is contingent solely upon the predictions and observations, thereby rendering it independent of the state definition.

Another solution consists of defining the state with the union of the particles, considering the position and associated intensities of each particle. Darakananda et al.~\cite{darakananda_data-assimilated_2018}. Complex filters have been developed to estimate correctly the posterior discretization based on a nonlinear observation model or a deficient number of pressure sensors~\cite{le_provost_low-rank_2021}.
However, these methods grapple with scenarios involving markedly divergent particle discretizations or highly variant model flows. In this general case, using a particle state using all particles for the update is unfeasible. Indeed, the update implies a linear combination of all members, leading to an exponential increase of particles.
On the other hand, the state could be associated with the spatial field defined in a functional space. The updated fields could be evaluated on the entire domain. Finally, using approximation or regression, a new particle discretization could be approximated. These types of methods have already been introduced in the Vortex Method to better approximate the vorticity field by changing the particle intensities. Regroup under the label Meshless Rezoning Methods in~\cite{mimeau_review_2021}. It mainly involved iterative methods~\cite{beale_1988},  triangulation~\cite{russo_1994} or Radial Basis Function (RBF) interpolation~\cite{barba_lorena_a_vortex_2004,sperotto_2021}. The last ones offer to easily introduce new particles or introduce penalization to regularize optimization problems.\newline

Based on those different formulations, we propose two types of adaptation of the EnKF filter. First, the Remesh-EnKF uses a new reference particle discretization. This way, the state could be updated, and the number of particles is controlled. This first method is based on the regridding of the particle discretization as described by~\cite{cottet_multi-purpose_1999} on which the classical EnKF analysis could be performed.
Then, in a case where the particle discretization would be preserved, the Particle-EnKF is introduced. In this case, the analyzed field is approximated with the previous particle discretization. The particle's positions are unchanged; only the strengths are modified by regression.
In the next part, background on sequential filtering and EnKF algorithm will be introduced \ref{Background_DA}, then on particle-based methods \ref{Background_Part}. Then, the two categories of method will be described in section \ref{Methods}. Afterward, those filters will be compared with a grid-based filter in a 1D Advection-Diffusion problem in section \ref{App_1D}, and an incompressible viscous flow is solved using a Vortex Method \ref{App_2D} where the filters are quantitatively analyses.







% !TEX root = main.tex

\section{Background}

\subsection{Data assimilation}~\label{Background_DA}

Data assimilation could be generally formulated with a probabilistic framework. It allows us to rigorously deal with measurement and model error in order to not only deduce an estimate of the real state but also associate uncertainty. Thus, state and observation are modeled as random variables. A filtering approach is then applied to estimate the current state based on past observations sequentially.

The goal is to establish the recurrence in probability distributions that, through Bayesian estimation, will enable us to estimate the current state and predict the future state.


\subsubsection{Data assimilation setting}

A hidden Markov chain is used to model this recurrence. We position ourselves within a finite-dimensional context. The forecast and observation are introduced, such as for $ k \geq 0$,
\[
    \begin{cases}
        \statebis_{k+1} = \mathcal{M}_{k+1} (\statebis_{k}) + \bm{\eta}_{k+1}, \\
        \bm{y}_{k+1} = \mathcal{H}_{k+1} (\statebis_{k+1}) + \bm \varepsilon_{k+1},
    \end{cases}
\]where $\mathcal{M}_{k+1}$ is the model operator describing the time evolution of the state from time $k$ to time $k+1$, and $\mathcal{H}_k$ is the observation operator. The term $\statebis_k$ $\in$ $\mathbb{R}^n$ is the vector state at time $k$ and $\bm{y}_k$ $\in$ $\mathbb{R}^m$ the observation vector, $\bm{\eta}_{k}$ is the model error that accounts for error in the numerical model and the errors due to discretization, and $\bm{\varepsilon}_k$ is the observation error which combine measurement error and representativeness error. We assume that $\bm{\eta}_{k}$, $\bm{\varepsilon}_k$ are random variables following Gaussian distributions with zero mean and covariance matrices $\bm Q_k$ and $\bm R_k$ respectively. Finally, we assume that the observation and the model errors are independent though the time and that initial error on $\statebis_0$, $\bm{\varepsilon}_k$ and $\bm{\eta}_{k}$ are mutually independent.Let $\mathcal{D}_k = \left\{\bm y_l\right\}_{l=1}^k$ represent the accumulated data up to time $k$.
Thus, $\statebis_{k+1}$ and $\mathcal{D}_k$ are conditionally independent with respect to $\statebis_{k}$, as well as $\bm{y}_{k+1}$ and $\statebis_{k+1}$, leading to simplifications in the next paragraph.

\subsubsection{Bayesian filtering}

The filtering problem consists of assessing the current state of the signal by utilizing data observation up to the present moment. Filtering involves the determination of $p(\statebis_{k} \mid \mathcal{D}_{k})$, the probability density function associated with the probability measure on the random variable $\statebis_{k} | \mathcal D_{k}$. Specifically, filtering focuses on the sequential updating of this probability density function as the index $k$ is incremented.
The state density is initialized by the a priori density of the initial state $p_{x_0}$.
Then, for all $k \geq 0$, probability distributions are propagated.
The forecast step is obtained through the law of total probability

\begin{equation*}
    p(\statebis_{k+1} \mid \mathcal D_k) = \mathbb{E}_{\statebis_k}\left[p(\statebis_{k+1} \mid  \statebis_k,\mathcal{D}_k)\right] = \mathbb{E}_{\statebis_k}\left[p(\statebis_{k+1} \mid \statebis_k)\right].
\end{equation*}
The a priori law of the $k+1$ observations can be obtained again through the law of total probability
\begin{equation*}
    p(\bm y_{k+1} \mid \mathcal D_k) = \mathbb{E}_{\bm{x}_{k+1}}\left[p(\bm y_{k+1}\mid \statebis_{k+1}) \mid \mathcal D_k\right].
\end{equation*}
After the $k+1$ observation of the random variable $\bm y_{k+1}$, the analysis step determines the a posteriori law of the state using Bayes law
\begin{equation*}
    p(\statebis_{k+1} \mid \mathcal D_{k+1}) = p(\statebis_{k+1} \mid \bm y_{k+1}, \mathcal D_{k})  = \frac{p(\bm y_{k+1} \mid \statebis_{k+1} ,\mathcal D_k)  p(\statebis_{k+1}\mid \mathcal D_k)}{p(\bm y_{k+1}\mid \mathcal D_k)}.
\end{equation*}

Due to the independence hypothesis the formula reduced to

\begin{equation*}
    p(\statebis_{k+1} \mid \mathcal D_{k+1}) = \frac{p(\bm y_{k+1} \mid \statebis_{k+1})  p(\statebis_{k+1})}{p(\bm y_{k+1})} \propto p(\bm y_{k+1} \mid \statebis_{k+1})  p(\statebis_{k+1}).
\end{equation*}

This finally lead to a mapping from the prior $p(\statebis_{k+1} \mid \mathcal D_k)$ to the posterior $p(\statebis_{k+1} \mid \mathcal D_{k+1})$.
We remove the time subscript $k$ in the rest of the section for simplicity and present the forecast and analysis step for a time increment.

\subsubsection{Ensemble Kalman Filter}~{\label{enkf}}

The Kalman filter \cite{kalman_new_1960} is a Bayesian filter that, in addition to the previously mentioned assumptions, requires $\mathcal{M}_k$ and $\mathcal{H}_k$ to be linear operators. In this case, the posterior distribution of the state is still Gaussian, so only the mean and the variance are updated. The Kalman estimator is thus a recursive version of the Minimum Mean Square Error applied to the Gaussian Linear model.

The ensemble Kalman Filter (EnKF) is a data assimilation method adapted to high dimensional non-linear problems introduced by Evensen~\cite{evensen_sequential_1994}. The formulation uses an ensemble of discrete samples based on the assumptions of a multivariate Gaussian distribution, such as the Kalman filter. We present the stochastic EnKF, where the observations are perturbed to account for observation errors.

Assuming we have an ensemble of $N$ states $\left\{\bm \statebis_i \right\}_{i=1}^N$, we forecast the ensemble by propagating each state with the dynamic model and obtain a forecast ensemble.
The two first moments of the error are given by

\begin{eqnarray*}
    \overline{\statebis}^f &=& \frac{1}{N} \sum_{i = 1}^{N} \statebis_i^f, \\
    \Cov^f &=& \frac{1}{N-1} \sum_{i = 1}^{N} (\statebis_i^f - \overline{\statebis}^f){(\statebis_i^f - \overline{\statebis}^f)}^T,
\end{eqnarray*}
where $\overline{\statebis}^f$ and $\Cov^f$ are respectively the empirical estimates of the mean and  the covariance matrix of the state distribution obtained from the ensemble members.

Likewise, the mean and covariance of the observation $\left\{\mathcal{H}(\statebis_i^f) \right\}_{i=1}^N$ could be estimated as well as the covariance between state and observation.

We develop the general formulation of the EnKF filter in the Appendix~\ref{appendix:enkf}.

Finally, our formulation of EnKF takes advantage of a correction of the state defined in the member space. We define $\Fcorr$, the correction matrix that gives the update in terms of linear combinations of the forward states

\begin{equation}~\label{enkf_formula_Fcorr}
    \mstatebis^a = \mstatebis^f + \mstatebis^f \Fcorr,
\end{equation}where the matrix $\Fcorr$ only depends on the ensemble members through the predicted observations ensemble $\left\{\mathcal{H}(\statebis_i^f) \right\}^N_{i=1}$, the observation $\obs$ and the associate perturbations  $\left\{\bm{\varepsilon}_i \right\}^N_{i=1}$

This independence is made possible by the linearization of the observation operator and the low-rank approach. The same property is verified for the ensemble transform Kalman filter~\cite{bishop_adaptive_2001}. Indeed, the filter is explicitly defined in the perturbation space.
% !TEX root = main.tex

\subsection{Particle-based methods}~\label{Background_Part}
We consider particle methods for solving continuous problems in fluid or solid mechanics. The Lagrangian methods decompose the domain on a set $\mathcal P = \left\{\bm z_p, \bm U_p\right\}_{p = 1}^{n_p}$, where $\bm z_p \in \Omega$, the spatial coordinates a particle $p$ and $\bm U_p \in \mathbb{R}^n$ its intensity.

% The velocity field $\bm{v}$ is used to update the position of the particle such as $\bm z_p(t+dt) = \bm z_p(t) + f(\bm z_p, \bm{v}{\bm z_p})$ with $f$ depending on the time-integration scheme.

% The computation of the velocity field and the solving of the equation of mechanics depend on the class of method.
We focus our work on methods that discretize a solution $\bm u$ on a continuous domain $\Omega \subset \mathbb{R}^d$ with $\Omega$ the spatial domain. This includes methods like Smoothed particle hydrodynamics (SPH) \cite{gingold_monaghan_sph_1977,lucy_1977}, or the Vortex Method (VM) \cite{cottet_vortex_2000} and is extended to other methods like the Material Point Method (MPM) \cite{sulsky_particle_1994}.

\subsubsection{Particle discretization}

Any smooth field $\bm u$ on $\Omega$ could be written

\begin{equation*}
	\bm u(\bm z) = \int_{\Omega} \bm u(\bm z') \delta(\bm z' - \bm z)  d\bm z',
\end{equation*}with $\delta$ the Dirac distribution.

A kernel function $\phi_\varepsilon$ is introduced to obtain an average estimate $\langle \bm u \rangle$ of $\bm u$ such that

\begin{equation*}
	\langle \bm u(\bm z) \rangle = \int_{\Omega} \bm u(\bm z') \phi_\varepsilon(\bm z-\bm z') d\bm z,
\end{equation*}where $\varepsilon$ is the smoothing length. The smooth kernel should at least respect the following properties

\begin{eqnarray*}
	&& \int_{\Omega} \phi_\varepsilon(\bm z) d\bm z = 1,      \\
	&& \phi_\varepsilon(\bm z) \to \delta(\bm z), \quad \varepsilon \to 0, \\
	&& \phi_\varepsilon(\bm z) \in C^k,  \quad k \geq 1,
\end{eqnarray*} where the two first properties are remanent properties of the Dirac distribution and the last is a differentiability requirement.

The average function $\langle \bm u \rangle$ is then used to approximate the original function.

Finally, the original domain $\Omega$ is subdivided with $N_p$ subdomain $\Omega_p$ associated with a Lagrangian particle in the location $z_p \in \Omega_p$. We denote by $V_p$ the volume of $\Omega_p$. This discretization is then used to approximate the average function such that

\begin{eqnarray}~\label{part_approx}
	u(z) &\approx& \sum_{p \in \mathcal P} \bm U_p \phi_\varepsilon (\bm z-\bm z_p).
\end{eqnarray}

Thus, any function defined on a particle discretization is defined by a particle discretization $\mathcal{P}$ and a smoothing kernel $\phi_\varepsilon$.
Based on this discretization, the differential operator could be derived through this formulation.

Several kernels have been used depending on the method. Theoretically, it could be the Gaussian kernel function

\begin{equation*}
	\phi_g(\bm r) = \frac{1}{{(\pi \varepsilon^2)}^{d/2}} \exp(-\|\bm r\|^2/\varepsilon^2).
\end{equation*}

This kernel is infinitely differentiable but defined on non-compact support. In practice, we use a cutoff to remove negligible value for a large distance from a particle.

\subsection{Particle-based function manipulations}~\label{operators}

Based on particle discretization, we present several particle manipulations that will used in our methods. Initially, those manipulations are either dedicated to improving the quality of the approximation, avoiding high distortion by creating a new particle discretization or projecting the solution on an Eulerian configuration. The different operators will be used in the assimilation process in
order to update the particle solution of each member in Section~\ref{Methods}.

\subsubsection{Approximation operator}~\label{interpOp}

The first category of manipulations aims to improve the approximation of the field by modifying particle strength.
A first approximation could be to use the particle approximation to reevaluate the particle intensities like in Equation \ref{part_approx} such as

\begin{equation*}
	\bm U_p = \int_{\Omega_p} \bm u(z) d\bm z = \bm u(\bm z_p) V_p,
\end{equation*}~where $\bm z_p$ is the particle location.

This approximation is easily computable but does not ensure the conservation of all the moments of the field. A better approximation could be obtained using the iterative Beale's formula \cite{beale_accuracy_1988}, which corrected circulation values in order to recover the vorticity field at the particle locations.

\subsubsection{Regression operator}~\label{regressionOperator}

Based on regression methods, the new intensities of the particles defined $\bm{U} = [\bm U_1, \dots, \bm U_p]^T$ could be obtained by minimizing the quadratic error. Assume that we have some vector $\bm{u} = [\bm u_1(z_1), \dots, \bm u_p(\bm z_p)]^T$ of the continuous field evaluations. The particle approximation could be computed on each particle position $\bm z_p$ given

\begin{equation*}
	\bm{u} \simeq \bm \Phi \bm{U},
\end{equation*}where $\bm \Phi_{ij} = \phi_\varepsilon(z_i - z_j)$.

Finding the new intensities $\bm U^*$ corresponds to solving the previous system in the least square sense. It corresponds to the classical problem of finding the minimizer of the following quadratic function

\begin{equation*}
	\bm{U}^*= \argmin_{\bm{U}} \norm{\bm{u} - \Phi \bm{U}}^2_2.
\end{equation*}


In this case, the solution is $\bm U^*  = (\bm \Phi^T \bm \Phi)^{-1} \bm \Phi \bm{u}$. This problem may be ill-posed, particularly in the case of a large set of non-well-distributed particles. We choose to regularize the solution by introducing a penalization term. The Ridge regression introduces a penalization on of the form $\lambda \norm{\bm U}_2^2$, where $\lambda$ is a penalization coefficient, such as the new problem is

\begin{equation*}
	\bm{U}_{\text{ridge}}^* = \argmin_{\bm{U}} \norm{\bm{u} - \bm \Phi \bm{U}}_2^2 + \lambda \norm{\bm{U}}^2_2,
\end{equation*}given the following solution $\bm{U}^*_{\text{ridge}} = (\bm \Phi^T \bm \Phi + \lambda \bm I)^{-1} \bm \Phi \bm{u}$.

\subsubsection{Remeshing operator}~\label{remesh_part}
A second type of manipulation is based this time on a complete projection of the solution on a new regular grid of particles~\cite{cottet_vortex_2000,cottet_multi-purpose_1999}. This method allows us to switch from a Lagrangian discretization $\mathcal P$ to an Eulerian one $\Lambda$ and then switch back to an entirely new regular particles discretization $\mathcal P'$ that conserves as much as possible the moment of the particle discretization.

In our methodology, we propose a two-step approach. First, we execute an assignment step (\ref{assigment}) to transfer the particle discretization to the grid discretization. Subsequently, an interpolation step (\ref{interpolation}) is performed to yield a new set of regularly spaced particles.

Our analysis pertains to the one-dimensional spatial scenario, where $\Omega \in \bR$. The extension to the $n$-dimensional case can be achieved through the tensorization of the one-dimensional approach.

\begin{enumerate}[label=(\alph*)]
	\item  \textit{Assignment on an Eulerian grid} \label{assigment}

	      We denote by $z_{I}$ and $z_{p}$ respectively the grid and the old particle locations. The new particles are defined on a grid of $n_g$ elements with regular spacing $\ell_I = 2 d_p$ where $d_p$ is the characteristic size of the particles. We define the particle intensities as $\bm U_p$ and the nodal field values as $\bm u_I$. By using some shape function $W$, the assignment step from particles to each node $I \in \Lambda$ can be written as

	      \begin{equation*}
		      \bm{u}_I = \frac1{V_I} \sum_{p \in \mathcal P} \bm U_p  W \left(\frac{z_I - z_p}{\ell_I} \right).
	      \end{equation*}

	      Where $W$ determines a redistribution of the intensity on the grid, the new discretization can then be used to approximate the field $\bm{u}_p$, defined by the particle discretization by interpolation given.

	      \begin{equation*}
		      \bm{u}_p(z) \approx \bm{u}_g(z) = \sum_{I \in \Lambda} \bm u_I W \left(\frac{z - z_I}{\ell_I} \right) \quad \forall z \in \Omega.
	      \end{equation*}
	\item  \textit{Interpolation on a new regular particle discretization} \label{interpolation}

	      A new set of particles is defined at the quarter of each cell such that the new position is defined at $z_{p'} = d_p/2 + i~dp, \quad i = 0,\dots, 2n_g $. The value of the field is then interpolated at that new location and multiplied with the volume of the particle $\bm{U}_{p'} = \bm  u_g(z_{p'}) V_{p'}$ in order to give a new particle approximation of the field.

	      \begin{equation*}
		      \bm{u}_g(z)  \approx \bm{u}_{p'}(z) = \sum_{p'\in\mathcal P'} \bm{u}_g(z_{p'}) V_p,
	      \end{equation*}.
\end{enumerate}

The combination of these two steps can initially be utilized to generate a new undistorted particle distribution.
The shape function $W$ determines the type and quality of the transfer. The method effectiveness is evaluated by assessing the conservation of the first moments of the particle distributions, as detailed in the Appendix \ref{appendix:moment_conservation}.

For $W$, one may employ the piecewise linear interpolation function, which ensures the conservation of moment 0. For higher moment conservation, the B-spline function provides a smoothing function for higher order.

However, while higher-order B-splines improve the smoothness of the solution, their accuracy is limited to the second order, allowing only exact interpolation of linear functions.

Monaghan~\cite{monaghan_extrapolating_1985} proposes a systematic approach to enhance accuracy and maintain smoothness through extrapolation. The concept involves constructing a new shape function based on a cutoff and its radial derivative. For $m = 4$, the cubic B-spline is improved by the following new interpolating kernel

\begin{eqnarray*}~\label{cubic_radial_kernel}
	M_4'(z) &=& \left\{ \begin{aligned}
		 & 1 - \frac{5}{2}z^2 + \frac{3}{2} |z|^3 & 0 \leq & |z| \leq  1 & \\
		 & \frac{1}{2}{(2 - |z|)}^2(1 - |z|)      & 1 \leq & |z| \leq 2  & \\
		 & 0                                      & 2 \leq & |z|.
	\end{aligned}
	\right.
\end{eqnarray*}

The drawback of this method is that it does not ensure positivity. Therefore, we opt to utilize the $M_4'$ kernel for our implementation.

Finally, in multidimensional space, the redistribution kernel $W$ can be obtained as the product of the one-dimensional kernel applied to each coordinate, as follows
\begin{eqnarray*}
	\bm U_p &=& \sum_{I \in \Lambda} \bm U_I  W \left(\bm z_p - \bm z_I, \ell_I \right) \\
	&=&  \sum_{I \in \Lambda} \bm U_I  \prod_{i = 1}^d W_{1\text{D}} \left(\frac{\bm z_{I, i} - \bm z_{p, i}}{\ell_I} \right)
\end{eqnarray*}
% !TEX root = main.tex

\section{Methods}

We want to define a formulation of the data assimilation problem that include a correction respectively on position and strength. In this formulation, we suppose that the state error could be .  We made the assumption that the uncertainty decomposition, allowing us to use the data several time.

\subsection*{Strenght correction}

- Comme l'opérateur d'observation de vitesse est défini sur une grille fixe, on peut utiliser la linéarité de l'opérateur d'observation. Ainsi on peut exprimer facilement la correction EnKF --> ca a du sens ici d'utiliser la méthode EnKF.

To adapt the discretization, and correct position erreur, the strength correction is preceded by the alignment step.

\subsection{Position transformation}

To define the position correction, we introduce a mapping $A$ apply to particle positions.
For any state $\bm \omega^f(\bx) = \sum_{p in \mathcal P_i} \Gamma_p ^f \phi_\varepsilon(\bx  - \bx_p)$, the updated state is given by  $\bm \omega^a = \sum_{p in \mathcal P_i} \Gamma_p ^f \phi_\varepsilon(\bx  - A(\bx_p))$.

In order to define a transformation that respect the assumption of an incompressible flows, we choose to define the mapping $A$ through the integration of a divergence-free velocity $w$ over an arbitrary time interval. By this way $A$ is the solution of the following ODE.

\begin{gather*}
    \left\{\begin{aligned}
         & \bm x'(\tau = 0) = \bx,                                        \\
         & \frac{d\bx'}{dt} = \bm u(\bx'), \quad A(\bx) = \bx'(\tau = 1).
    \end{aligned} \right.
\end{gather*}

By applying this type of transformation for position correction, we validate that the position is physically consistent. Finally, the problem of position correction consist to determine the field $\bm u(.; \bm b)$ parametrized with vector $\bm b$.

\subsection*{Alignment step}

As for the strength correction step, we first define a cost function to minimize based on the discrepancy between observations and predictions. For each member $i = 1, \dots, N_{\text{ens}}$  of the ensemble we search

\begin{equation*}
    \bm u_i = \argmin_{\bm u \in \mathcal U} \left\|\mathcal H(\bm \omega^f_i \mid \bm u) - y_i \right\|_{\bm R}^2 + R(\bm u),
\end{equation*}where $\mathcal U$ is the velocity search space, $\mathcal H(\bm \omega^f \mid w)$ is the observation operator conditioned by $\bm u$ such that $\mathcal H(\bm \omega^f \mid \bm u) = \mathcal H(\bm \omega^f \circ A(\cdot; \bm u))$ is the observation of $\omega^f$ after the application of the position correction. Because the problem is ill-posed a regularization terme $R(w)$ is needed. In the problem we choose, the streamlines form a closed loop, meaning that double $w$ will give the same alignment. Moreover, the regularization is needed to be less sensitive to noise and ill-conditioning.This term is based on prior assomption concerning the definition of $w$.
Because the operator is highly non-linear with respect to the coordinate transformation, we will directly minimize the cost function to determined the field parameter

\subsection{Reduced-order modeling based on ensemble decomposition}%Correction field parametrization and search space}

To solve this alignment problem, we need to discretize the velocity field and define the search space $\mathcal U$. We choose to define the correction in the ensemble space of velocity field define as $V = \text{Span}(\{\mathbf{v}_i\}_{i = 1}^N)$, where $v_i$ is the velocity field induced by the vorticity of member $i$. This choice is based on the fact that the error of position is reckon to be due to an error in the integration of the velocity field.  As a first approximation we use the velocity field at the end of simulation. Moreover the fields $u_i$ are divergence-free, and by linearity of the gradient, the search space verify this property. We introduce vectors $\bm a_i \in \mathbb R^N$ such as we write

\begin{equation*}
    \bm u_i =  \sum_{j = 1}^N a_{i,j} \frac{\bm v_j}{\sqrt{N - 1}}.
\end{equation*}

This decomposition lead to the definition of $N$ independent problems of dimension $N$ to solve

\begin{equation*}
    \mathcal L_i =  \min_{\bm a \in \mathbb R^N} \left\| h(\bm a) - \bm y_i\right\|_{\bm R}^2 + \frac{\lambda}{2} \norm{\bm a}^2_2,
\end{equation*}where  $h(\bm a) = \mathcal H(\bm \omega^f_i \mid \bm a,V)$ is the observation operator conditioned by $\bm a,\{\mathbf{v}_i\}_{i = 1}^N$ such as $h(\bm a) = \mathcal H(\bm \omega^f_i \mid \bm a,V) =\mathcal H(\bm \omega^f_i \circ A(\cdot; \bm w = \sum_{j = 1}^N a_{j} \bm v_j))$. We use a Ridge regression on the coefficient of $\bm a$ with a coefficient $\lambda$. Ridge penalization is a commun choice to avoid overfitting by penalize high value of coefficient. It also stabilize solution particularly when $\bm v_i$ are correlated. This choice is also linked to a prior on the coefficient of the decomposition. It supposes that the coefficient $a_i$ are independent and identically distributed such as $a_i \sim \mathcal N(0, 1/ lambda)$ and $p(\bm a) \propto \exp\left( - \frac{\lambda}{2} \norm{\bm a}_2^2\right)$. That also mean that the same weight is given to each field $\bm v_i$ in the decomposition. In fact, this correspond to a regularization of the field $\bm u$ with $\mathcal R(\bm u) = \norm{\bm u}^2_{P}$, where the metric $P = \frac{1}{N-1} \sum_{i = 1}^{N}\bm v_j \otimes \bm v_j$.
% Penalize the norm of $\bm a$ is equivalent as penalize the field $w$ in the 
% This assumption are linked to a prior knowledge for the field $\bm u$ such as

\subsection{Gradient computation}
The cost functions $\mathcal L_i$ to minimize are non-linear. A local descent gradient algorithm is needed for optimization. The gradient with respect to $\bm a$ of $\mathcal L_i$ is defined as $\nabla_a L_i = \nabla_a h(\bm a)~\bm R^{-1}(h(\bm a) - \bm y_i) + \lambda~\bm a$. The $\nabla_a h(\bm a)$ could be compute with finite difference by evaluate $N+1$ times. An other way consist to evaluate the gradient during one model evaluation such as



\subsection{selection of the penalization coefficient}
- On a un lambda aussi qui dépend du temps d'intégration.


- **Induced Norm**: The norm induced by \(\mathbf{X}\mathbf{X}^T\) measures the "length" of a vector \(\mathbf{v}\) in a space where directions are scaled by \(\mathbf{X}\mathbf{X}^T\). This reflects how much each direction in the space is amplified or attenuated by the transformation given by \(\mathbf{X}\mathbf{X}^T\).

\section*{Choix du Paramètre de Régularisation (\(\lambda\))}

Le paramètre de régularisation \(\lambda\) contrôle la force de la pénalisation. Un \(\lambda\) plus grand signifie une régularisation plus forte, ce qui réduit davantage la magnitude des coefficients :
\begin{itemize}
    \item \textbf{Si \(\lambda = 0\)} : Le modèle revient à une régression linéaire classique sans régularisation.
    \item \textbf{Si \(\lambda\) est très grand} : Les coefficients peuvent devenir très petits, voire nuls, ce qui peut conduire à un modèle trop simple (sous-ajustement).
\end{itemize}


- ridge regression sur les coefficient de la comb linéaire pourquoi. Faire le lien avec la dist a priori. Dire que ça signifie mettre le même poids pour tous termes donc pour tous les membres.
- Nécessité de choisir le terme de pénalisation. Ecrire sous une forme qui dépend de N_ens / sigma / nobs
% !TEX root = main.tex
\newpage

\section{1D density advection-diffusion problem}~\label{App_1D}
\subsection{Description of the problem}

An initial exploration is conducted on a one-dimensional application to assess the filter performance. We define the following one-dimensional $2\pi$-periodic convection-diffusion problem such as
\begin{equation*}
	\frac{\partial u}{\partial t}(z,t) + v \frac{\partial u}{\partial z}(z,t)  = \visc \frac{\partial^2 u}{\partial z^2}(z,t),
\end{equation*}
with $z$ the spatial coordinate, $v$ a constant velocity and $\visc$ a constant diffusion coefficient.
For the following application, the reference solution will use $v = \refv$ and $\visc = \refvisc$ as parameters.
We define the $2\pi$-periodic heat kernel in one dimension, such as

\begin{equation*}
	\phi(u, s) = \sum_{k=-\infty}^{\infty} \frac{1}{\sqrt{4 \pi s}} \exp{\left(-\frac{{(u - 2\pi k)}^2}{4s} \right)}.
\end{equation*}

Considering an initial condition characterized by a Gaussian shape expressed as $u^{gt}(z, 0) = \phi(z-z_0, Dt_0)$, where $\zz$, $t_0 = \frac{\sigma_0^2}{2D}$, and $\sigz$, we derive the comprehensive analytical solution utilizing the Green equation solution
\begin{equation*}
	u^{gt}(z, t) = \phi(z- v t - z_0, \visc (t+t_0)).
\end{equation*}The analytical solution is succinctly described as a Gaussian function, characterized by a mean that moves at the advection velocity and a standard deviation proportional to $t$ and $D$. This solution is visually depicted in Figure~\ref{fig:1d_analytical} across various assimilation time frames.

\begin{figure}[ht]
	\centering
	\includegraphics[width=\linewidth]{images/app1d/analytical_frame.pdf}
	\caption{The analytical solution of the convection-diffusion problem evolves over time, with the final snapshot revealing a complete spatial period.}
	\label{fig:1d_analytical}
\end{figure}

Following a Lagrangian perspective by tracking a fluid particle of position $z_p$ and intensity $U_p$, the equation becomes

\begin{equation*}
	\frac{dz_p}{dt} = v(z_p, t), \quad \frac{dU_p}{dt} = D \frac{d^2 U_p}{dz^2}
\end{equation*}

For solving the convection-diffusion scheme, we employ the two steps of the viscous splitting algorithm. The advection is taken into account by updating the position of the particle with an Euler explicit scheme.
On the other hand, we use a redistribution method called the Particle Strength Exchange Method (PSE)~\cite{degond_1989,cottet_1990} to approximate the laplacian term $\frac{d^2 U_p}{dz^2}$.


\begin{equation*}
	D \frac{d^2 U_p}{dz^2}  = D V_p \frac{d^2 u_p}{dz^2} \approx D V_p \varepsilon^{-d} \int [u(z)  - u(y)] \phi_\varepsilon(y - z) dz,
\end{equation*}

This leads, with the particle approximation, to a redistribution of the particles intensities in their previous locations, such as

\begin{equation*}
	\frac{dU_p}{dt} = D \varepsilon^{-d} V_p \sum_q (U_q - U_p) \phi_\varepsilon( z_q -  z_p),
\end{equation*}where $V_p$ the volume of the particle $p$ and $d$ the dimension of $\Omega$.
For further details on the computation, please refer to \cite{cottet_1990}.

For the periodic boundary problem described in section \ref{App_1D}, we define an equivalent kernel function $\phi_\varepsilon= \phi^P_g = \sum_{n=-\infty}^{+\infty} \phi_g(r - 2 \pi  n )$.

The particle-based model employs a discretization of \npart{} particles with a size of $h = \frac{L}{N_{\text{part}}}$ and a smoothing length of $\varepsilon = 1.3 h$.
For the sake of comparison, we solve the convection diffusion equation with an explicit central finite difference scheme discretized on a regular grid with \ngrid{} nodes.

\subsection{Assimilation parameters and ensemble generation}

\subsubsection{Ensemble distribution}
All filters undergo testing on an identical initial prior ensemble of size $N = 25$ members, characterized by Gaussian shapes that are shifted and scaled. The mean of the function and its standard deviation are sample. The total mass is set equal to 1. Parameter of velocity $v$ and the diffusion $D$ are also sampled. The different distribution are defined in Appendix~\ref{tab:ens_gen_1d}. Moreover the parameters samples and initial state are illustrated in Figure~\ref{fig:initial_gen}.

The observational data is subject to additive noise, denoted as $\eta \sim \mathcal{N}(0, \sigma_y \bm{I})$, where $\sigma_y = \sigmaY$ and $\bm{I}$ represents the identity matrix.

\subsubsection{Error definition}
We define the error as the following relative ratio

\begin{equation}~\label{eq:L2_error}
	e_{L_2} =\frac{ \left[\frac1\nens \sum_{i = 1}^{\nens} \int_\Omega \left(u_i(z) - u^{gt}(z)\right)^2 dz\right]^{1/2}}{\norm{u^{gt}}_{L_2}}
\end{equation}~where $u_i$ denote the $i$-th member of the ensemble and $\norm{u}_{L_2}$ denote the $L_2$ norm of $u^{gt}$.

The $L_2$ norm is computed using a quadrature over a regular grid of an ensemble of cells $\mathcal{C}$ such as for any $f \in L_2$

$$
	\norm{f}_{L_2}  = \int_{\Omega} f^2~d\Omega \approx \sum_{c \in \mathcal{C}} f(z)~V_c
$$~where $z_c$ is the center of the cell $c$ and $V_c$ the volume of the cell. The grid is still the same for all the simulations.
%We compute the parameter error with a norm-2 as $e_{\theta} = \frac{ \left[\frac1\nens \sum_{i = 1}^{\nens} \norm{\theta^{gt} - \theta_{i}}_2^2\right]^{1/2}}{\norm{\theta^{gt}}_2}$.

\subsubsection{Numerical parameters}

We conduct $N_{\text{assim}} = 30$ assimilation steps at evenly spaced intervals until the final time $t_f = 2 \frac{L}{v}$. During each assimilation step, the field $u^{gt}$ is observed at six regularly spaced positions $x_{\text{obs}}$.


In the particle-based simulation, fields are discretized using regularly spaced particles that are shifted. Intensity values are obtained by fitting an interpolation operator like in Section~\ref{interpOp} to the particle intensity.
The parameter $\varepsilon_{\text{mass}}$ is introduced as a cutoff for particle selection, allowing for the definition of varying numbers of particles for each simulation. The particle support poses challenges in the Part-EnKF as described in Section \ref{part_enkf}.

Simultaneously, a standard Ensemble Kalman Filter (EnKF) update is applied to the nodal variables to construct the reference filter Grid-EnKF which use a grid-based model. For grid-based simulation, the fields of each member are interpolated at the node locations. In this way, the ensemble generated is still the same for the sake of comparison.


\begin{figure}[ht!]
	\centering
	\begin{subfigure}{0.49\textwidth}
		\includegraphics[width=\textwidth]{images/app1d/param.pdf}
	\end{subfigure}
	\hfill
	\begin{subfigure}{0.49\textwidth}
		\includegraphics[width=\textwidth]{images/app1d/prior.pdf}
	\end{subfigure}
	\caption{On the left the initial parameters sample, $v$ in abscissa and $D$ in ordinate. On the right is the initial ensemble state.}
	\label{fig:initial_gen}
\end{figure}

\subsection{Results}

We compare the different filters on the assimilation of the state. We first compare the Grid-EnKF, the Remesh EnKF and two Part-ENKF filter with 100 and 60 particules. The parameter sample are still unchanged. We take unknown parameters into account as model uncertainties. The two filters outlined in the Method section~\ref{Methods} and the Eulerian filter are compared with the reference filter based on a grid discretization.

In figure \ref{fig:1d_error_time}, we appreciate different assimilation step for the Remesh-EnKF filter.


\begin{figure}
	\centering
	\begin{subfigure}{\textwidth}
		\includegraphics[width=\textwidth]{images/app1d/wo_calibration/remesh_EnKF.pdf}
	\end{subfigure}
	\caption{Data assimilation ower assimilation step for the Remesh-EnKF filter.}
\end{figure}

The result are quite similar for all the different filters, except the Part-EnKF with 60 particules.

\begin{figure}
	\centering
	\includegraphics[width=0.75\textwidth]{images/app1d/wo_calibration/state_error.png}
	\caption{State error with respect to assimilation time step.}
	\label{fig:1d_error_time}
\end{figure}

The primary issue arises from the regression on non-overlapping support, where the regression struggles to fit the analysis solution defined on a more considerable space support. This leads to heightened variability, particularly in the tail of the distribution. Addressing this common challenge in RBF Regression \cite{fornberg_flyer_2015} involves increasing the Ridge penalization coefficient, a parameter we choose through cross-validation Ridge regression.
Even with a more stable regression, it remains a projection of the analysis solution onto the forecast support. It is imperative to increase the number of particles to achieve a better approximation of the analysis solution using the particle approximation operator in Section~\ref{interpOp} or the regression operator in Section~\ref{regressionOperator}.

We validate this assumption by varying the initial support of particles. Quantitatively, as observed in Figure~\ref{error_support}, the error decreases with an increase in the number of particles. Moreover, qualitatively examining the snapshot on the right reveals that the solution closely aligns with the reference.

\begin{figure}
	\centering
	\begin{subfigure}{0.39\textwidth}
		\includegraphics[width=\textwidth]{images/app1d/error_support/error_support.png}
		\label{error_support1}
	\end{subfigure}
	\hfill
	% Revoir figures en plotly
	\begin{subfigure}{0.29\textwidth}
		\includegraphics[width=\textwidth]{images/app1d/error_support/ok.png}
		\label{error_support2}
	\end{subfigure}
	\hfill
	\begin{subfigure}{0.29\textwidth}
		\includegraphics[width=\textwidth]{images/app1d/error_support/not_ok.png}
		\label{error_support3}
	\end{subfigure}
	\caption{Left: Error with respect to particle support size, Middle: Final step for a support of 100 particles, Right: Final step for a support of 60 particles.}
	\label{error_support}
\end{figure}
However, Adding particles in a more complex solution is a challenging task. Indeed, a good spacing between particles and the density of particles has to be preserved. In this case, we advise defining criteria for the error of the reconstruction. Instead of adding particles, we advise generating a new, regularly spaced grid of particles to reconstruct the solution.

In conclusion, this example underscores the Remesh-EnKF filter capability to yield results comparable to the classical EnKF applied to a grid model. Additionally, it highlights the Part-EnKF capability in assimilating on a particle discretization while also emphasizing the importance of addressing spatial discrepancies between members, which can pose challenges in solution reconstruction. The computation of solution error reconstruction provides a straightforward criterion for remeshing a member and applying the analysis solution approximation.


\newpage
% !TEX root = main.tex

\section{2D vortex-in-cell problem}~\label{App_2D}
\subsection{Description of the method}


In this section, we apply the Vortex Method in a two-dimensional scenario, as outlined by Cottet et al. \cite{cottet_vortex_2000}. The Vortex Method is a Lagrangian approach utilizing a particle ensemble to discretize the vorticity field, allowing for the solution of the Navier-Stokes equation for viscous incompressible flow. The method is grounded in the vorticity-velocity formulation of the Euler equation, where $\bm \omega = \nabla \times \bm{v}$ satisfies

\[
	\begin{aligned}
		\frac{\partial \bm \omega}{\partial t} + (\bm{v} \cdot \nabla) \bm \omega - \nu \Delta \bm \omega & = 0, \\
		\nabla \cdot \bm v                                                                                & = 0,
	\end{aligned}
\]where $\omega$ denotes vorticity, $\bm{v}$ represents velocity, and $\nu$ stands for viscosity.

In the context of 2D flow, vorticity is perpendicular to the flow plane, forming a scalar field denoted as $\omega$. In Cartesian coordinates, it is expressed as $\omega = \frac{\partial v_y}{\partial x} - \frac{\partial v_x}{\partial y}$.

The vorticity field is discretized using a collection of discrete vortices, each characterized by a position $\bm z_p$, an associated kernel $\phi_\varepsilon$, and a circulation $\Gamma_p$. For all points $\bm z$ within the domain $\Omega$, the vorticity is expressed as

\begin{equation*}
	\omega(\bm z) = \sum_{i=1}^{N_p} \Gamma_p \phi_\varepsilon(\bm z - \bm z_p).
\end{equation*}

To address the Navier-Stokes equation, we employ a viscous splitting scheme, following the methodology outlined in \cite{cottet_1990}, acknowledging the predominance of the convection term over viscosity. We use the Vortex-In-Cell algorithm \cite{christiansen_1973, birdsall_1969}, coupled with an FFT solver to cumpute the advection velocity. The subsequent steps involve assigning particle vorticity values to the grid using a particle-to-grid formula, computing the velocity field by solving the Poisson equation on the grid verified by the stream function. Finally, the velocity is interpolated back onto the particles using the grid-to-particles formula. A Runge-Kutta 3 time-stepping scheme is employed to update the particle positions through a time integration scheme. The final phase involves solving the heat equation and update the particles intensities thanks to the PSE method previously described in Section~\ref{App_1D}.

\subsection{Lamb-Chaplygin dipole and simulation parameters}

We define the reference as the advection of the Lamb-Chaplygin dipole inside a close domain with stress-free walls. Lamb-Chaplygin dipole is a popular choice for numerical studies \cite{orlandi_vortex_1990}. The model represents a specific steady, inviscid dipolar vortex flow and offers a non-trivial solution to the two-dimensional Euler equations. The dipole is characterized by a translation velocity $U$, a mean position $\bm{z}_0$, a radius $R$, and an orientation $\alpha$.

The dipole vorticity field $\omega$ could be expressed as

\begin{equation*}
	\omega(r) = \begin{cases}
		\frac{-2 k U J_1(kr)}{J_0(kR)} \sin \alpha \quad & \text{for} \quad  r < R, \\
		0 \quad                                          & \text{otherwise},
	\end{cases}
\end{equation*}where $(r, \alpha)$ represent the polar coordinates in the dipole reference frame. Here, $J_0$ and $J_1$ denote the zeroth and first order Bessel functions of the first kind, respectively, and $k$ is determined such that $kR$ corresponds to the first non-trivial zero of the first Bessel function.  The dipole vorticity field is depicted in Figure \ref{fig:lamb_dipole}.

\begin{figure}[ht]
	\centering
	\includegraphics[width=0.6\linewidth]{images/app2d/lamb.pdf}
	\caption{The Lamb-Chaplygin dipole vorticity field on a normalize space.}
	\label{fig:lamb_dipole}
\end{figure}

The dipole is positioned at the center of a box with dimensions $[0, \pi] \times [0, \pi]$, featuring an orientation of $\frac{7\pi}{8}$ rad., a radius of $0.5$ meters, and a velocity $U$ of $0.25 \text{ m.s}^{-1}$. The complete reference setting is listed in Table \ref{tab:ref}.

The boundary box features stress-free walls, meaning fluid cannot pass through them. The velocity perpendicular to the walls is zero, while tangential velocity remains undetermined. When a vortex, such as a dipole, reaches this boundary, it walks along the wall, sensing its reflection and interacting with it.

Because this problem does not have an explicit solution on a closed domain, we simulate the ground truth with the vortex method for a fined discretization and fixed set of parameters described also in Table \ref{tab:ref}. The trajectory of the ground truth is illustrate in the Figure \ref{fig:ref_trajectory} on a regularly spaced grid.

\begin{figure}[htbp]
	\begin{subfigure}{0.32\textwidth}
		\includegraphics[width=\linewidth]{images/app2d/best_estimate_2.pdf}
	\end{subfigure}
	\hfill
	\begin{subfigure}{0.32\textwidth}
		\includegraphics[width=\linewidth]{images/app2d/best_estimate_10.pdf}
	\end{subfigure}
	\hfill
	\begin{subfigure}{0.32\textwidth}
		\includegraphics[width=\linewidth]{images/app2d/best_estimate_20.pdf}
	\end{subfigure}
	\caption{Trajectory of the ground truth. The vorticity is represented on a regularly spaced grid. For $t=[1, 5, 10]s.$}
	\label{fig:ref_trajectory}
\end{figure}
Several parameters in the simulation influence the particle distribution and can lead to different results. The first one is the particle size defined by $d_p$. Another significant parameter is $\varepsilon_\omega$, associated with the remeshing process occurring either during the forecast (to prevent high distortion of the particle distribution) or during the Remesh-EnKF filter. $\varepsilon_\omega$ serves as a threshold, determining whether a particle is retained after the remeshing process based on the condition $V_p \Gamma_p > \varepsilon_\omega$. The impact of this parameter is illustrated for one member after the first forward in Figure~\ref{fig:eps_effect}.

\begin{figure}[htbp]
	\centering
	\begin{subfigure}{0.3\textwidth}
		\includegraphics[width=\linewidth]{images/app2d/part_eps_0.1.png}
	\end{subfigure}
	\hfill
	\begin{subfigure}{0.3\textwidth}
		\includegraphics[width=\linewidth]{images/app2d/part_eps_0.01.png}
	\end{subfigure}
	\hfill
	\begin{subfigure}{0.3\textwidth}
		\includegraphics[width=\linewidth]{images/app2d/part_eps_1e-6.png}
	\end{subfigure}
	\caption{Effect of the parameter $\varepsilon_\omega$ on the particle discretization of the solution for one member. From left to right, results for $\varepsilon_\omega = 0.1, 0.01$, and $1.e^{-6}$.}
	\label{fig:eps_effect}
\end{figure}

For the next paragraphs, if the value is not explicitly changed, we use the nominal parameters described in Table \ref{tab:simu_2d} for the simulation.

\subsection{Assimilation parameters and ensemble generation}

\subsubsection{Ensemble distribution}
An ensemble of 32 members is created by sampling distributions over the dipole parameters. We sample the radius $R$, the prescribed velocity $U$, the orientation $\alpha$, and the barycenter $\bm z_{\text{mean}}$. Additionally, the model viscosity $\nu$ is also sampled. All the distributions are summarized in Table \ref{tab:ens_dipole}. The first six members are plotted in Figure \ref{fig:sample_ens}.

\begin{figure}[ht]
	\centering
	\includegraphics[width=0.9\linewidth]{images/app2d/ensemble_sample.png}
	\caption{Six samples from the initial ensemble.}
	\label{fig:sample_ens}
\end{figure}

The initial vorticity field is first discretized on a regular grid of particles with a characteristic length $d_p$, where each particle receives the circulation $\Gamma_p = \omega(\bm z_p) V_p$, and $V_p = d_p^2$ represents the volume of the particle.

\subsubsection{Error definition}

We use an absolute error absolute \(L_2\)-error defined as $ \frac1\nens \sum_{i = 1}^{\nens} \int_\Omega \left(\omega_i(\z) - \omega^{gt}(\z)\right)^2 \mathrm{d}\z$.
We use the the member error also to evaluate the dispartion of the error estimate.

\subsubsection{Numerical parameters}

The assimilation frequency is defined by the assimilation step $dt_a$. The simulation is performed over a duration of $t_f$. All simulation parameters are summarized in Table \ref{tab:simu_2d}.

Observations are collected on a regular grid of size $N_{\text{obs}}$, measuring both components of the velocity. The observations follow a normal distribution $\mathcal N(0, \sigma_{\text{obs}}^2 \bm{I})$, indicating an ensemble of independent measurements, each characterized by a standard distribution of $\sigma_{\text{obs}}$. An example of observed velocity with and without noise is illustrated in Figure \ref{fig:velocity}.

\begin{figure}[htbp]
	\centering
	\includegraphics[width=0.8\linewidth]{images/app2d/velocity_ref_recadre.pdf}
	\caption{Observed and reference velocity fields. The error on each component is sample from a centred normal distribution with the nominal value $\sigma_{\text{obs}} = 0.05$.}
	\label{fig:velocity}
\end{figure}

\newpage

\subsection{Results}

\subsubsection{Error through time}

We start by analyzing the assimilation error over time. Figure \ref{fig:assim_time} illustrates the error throughout the assimilation process for the nominal set of assimilation parameters, demonstrating comparable results for both filters. At each assimilation step, the error decrease and avoid the solution to diverge elsewhere.

\begin{figure}[htbp]
	\centering
	\includegraphics*[width=0.7\linewidth]{images/app2d/final/error_in_time.pdf}
	\caption{Error curves through assimilation steps. Left: \(L_2\)-error of the field, Right: Error for the viscosity parameter. With Part-EnKF in blue and Remesh-EnKF in red.}
	\label{fig:assim_time}
\end{figure}

\newpage

\subsubsection{Error with respect to assimilation parameters}
We also assess the performances of the different filters by evaluating the convergence of the error with respect to the assimilation parameters.

We observe the convergence rate concerning data assimilation parameters: The observation precision, which is \(1/\sigma_{\text{obs}}^2\), the number of observations \(N_{\text{obs}}\), the number of assimilation step \(N_{\text{assim}}\).

The Figure \ref{fig:obs_precision_1} illustrate a decreasing error bias and variances with respect to observation precision similarly for both filter.What is striking in the same Figure~\ref{fig:obs_precision_2} but in the log-scale is the regular convergence rate for both filter with respect to the observation precision. The order of convergence is about 0.68 for Part-EnKF and 0.75 for Remesh-EnKF.

\begin{figure}[h!]
	\centering
	\begin{subfigure}{0.49\linewidth}
		\centering
		\includegraphics[width=\linewidth]{./images/app2d/final/MSE_obs_precision_box.pdf}
		\caption{}
		\label{fig:obs_precision_1}
	\end{subfigure}
	\begin{subfigure}{0.49\linewidth}
		\centering
		\includegraphics[width=\linewidth]{./images/app2d/final/MSE_obs_precision_box_log.pdf}
		\caption{}
		\label{fig:obs_precision_2}
	\end{subfigure}
	\caption{Box plots of the state error w.r.t. $1/\sigma_{\text{obs}}^2$.}
\end{figure}


In the Figure~\ref{fig:na_1} the reduction of error is still prominent and show a reduction of variance as the number of observation increase. In the log-scale Figure~\ref{fig:na_2}, the error decrease also at a constant rate for both Filter. We notice, we notice a stronger order of 1.8 for the Remesh-EnKF compared to an order of 1.4 for Part-EnKF.

\begin{figure}[h!]
	\centering
	\begin{subfigure}{0.49\linewidth}
		\centering
		\includegraphics[width=\linewidth]{./images/app2d/final/MSE_na_box.pdf}
		\caption{}
		\label{fig:na_1}

	\end{subfigure}
	\begin{subfigure}{0.49\linewidth}
		\centering
		\includegraphics[width=\linewidth]{./images/app2d/final/MSE_na_box_log_log.pdf}
		\caption{}
		\label{fig:na_2}
	\end{subfigure}
	\caption{Box plots of the state error w.r.t. $N_{\text{assim}}$.}
	\label{fig:na}

\end{figure}

Finally, we analyse the error convergence with respect to the number of observation. Observation locations increase regularly in both axis. In Figure~\ref{fig:nobs_1}, the error estimate and variances decrease. For relatively small number of observations the two filters offers similar results when in contrast the Remesh-EnKF have relatively better result when the number of observation is increased. Moreover, the convergence rate seems to change around 200 observation points as illustrate the log-scale Figure~\ref{fig:nobs_2}. Nevertheless, it illustrates adequate performances for both filters.

\begin{figure}[h!]
	\centering
	\begin{subfigure}{0.49\linewidth}
		\centering
		\includegraphics[width=\linewidth]{./images/app2d/final/MSE_nobs_box.pdf}
		\caption{Box plots of the state error w.r.t. $N_{\text{obs}}$.}
		\label{fig:nobs_1}
	\end{subfigure}
	\begin{subfigure}{0.49\linewidth}
		\centering
		\includegraphics[width=\linewidth]{./images/app2d/final/MSE_nobs_box_log_log.pdf}
		\caption{Box plots of the state error w.r.t. $N_{\text{obs}}$.}
		\label{fig:nobs_2}
	\end{subfigure}
	\label{fig:nobs}

	\label{fig:assim_params}
\end{figure}



\subsubsection{Error with respect to simulation parameters}

To better understand the differences, let us now turn to the evolution of the error with respect to particle discretization parameters. For the Part-EnKF, remember that each member has its own particle discretization that flows according to the dipole direction and velocity. Each analyzed member's solutions are then respectively projected on their member discretization. However, this scheme could introduce different sources of error. First, due to particle irregularity in the particle distribution, it introduced severe approximation that led to errors between the analyzed and the approximated solution. Even more seriously, certain parts of the solution may vanish as no particle in the support can interpolate it. This effect could be appreciated on several samples of the ensemble where the analysis is projected on a non-conforming particle discretization. For instance, we analyzed the first assimilation step of one member for the different filters. If the analyzed field is known over the space domain, we observed in Figure~\ref{fig:assim_member} that the Remesh-Filter and Part-Grid-EnKF are only able to interpolate the entire solution. The Part-EnKF is not entirely able to interpolate the solution with the forecast member discretization.
Moreover, some distortions observed in the particle distribution are not in line with the analyzed field flow. These remarks are more critical when the forecast step is longer, leading to high errors, or when the size of the support is lower. Moreover, the approximation of Section~\ref{regressionOperator} will introduce the approximation error function of the size of the particles. For instance, the particle will not conserve the total circulation due to a quadrature error, which is the opposite case for the Remesh-EnKF filter.

\begin{figure}[h!]
	\centering
	\begin{subfigure}{0.32\textwidth}
		\centering
		\includegraphics[width=\linewidth]{./images/app2d/assim_member_forecast.png}
		\caption{Forecast member discretisation.}
	\end{subfigure}
	\hfill
	\begin{subfigure}{0.32\textwidth}
		\centering
		\includegraphics[width=\linewidth]{./images/app2d/assim_member_ppf.png}
		\caption{Part-EnKF analyse discretisation.}
	\end{subfigure}
	\hfill
	\begin{subfigure}{0.32\textwidth}
		\centering
		\includegraphics[width=\linewidth]{./images/app2d/assim_member_rmf.png}
		\caption{Remesh-EnKF analyse discretisation.}
	\end{subfigure}
	\caption{Assimilation of one member with a forecast discretization unadapted to the analyses solution. From left to right and up to down: the forecast member, the analyses member with the Remesh-EnKF filter, the analyses member with the Part-EnKF filter and finally the analyses member with the Part-Grid-EnKF. The forecast discretization use by the Part-EnKF is not always a well support of approximation for the analyses and introduce discretization errors.}
	\label{fig:assim_member}
\end{figure}

To evaluate the effect of the size of the support, we varied the value of $\epsilon_{\omega}$. We have seen in Figure~\ref{fig:eps_effect} that this parameter affects the number of particles and, thus, the size of the support. In the Figure~\ref{fig:cuttoff}, we observe high disparities between the two filters. However, the error stabilizes rapidly by decreasing the threshold. These findings suggest an impact of the thresold and thus the particle support for the Part-EnKF. By contrast, Remesh-EnKF is less sensible as consequence it only thresold low value in the analysed solutions.

\begin{figure}[h!]
	\centering

	\begin{subfigure}{0.48\textwidth}
		\centering
		\includegraphics[width=\linewidth]{./images/app2d/final/MSE_cutoff_box.pdf}
		\caption{state error w.r.t. $\varepsilon_{\omega}$. The High, low and medium cutoff correspond respectively to $\varepsilon_{\omega} = 0.1, 1.e^{-6}$ and $0.01$.}
		\label{fig:cuttoff}
	\end{subfigure}
	\hfill
	\begin{subfigure}{0.48\textwidth}
		\centering
		\includegraphics[width=\linewidth]{./images/app2d/final/MSE_size_box.pdf}
		\caption{state error w.r.t. $d_p$. The large, medium and low size correspond to $d_p = 0.0327, 0.0245$ and $0.0123$.}
		\label{fig:np}
	\end{subfigure}

	\caption{Error box plots for simulation parameters. The effect of $\varepsilon_{\omega}$ on the error is particularly observed on high value. The Part-EnKF error is strongly linked to $d_p$ through the particle approximation error.}
	\label{fig:simu_parameters_error}
\end{figure}

In term of particle size $d_p$, we evaluate its effect on the assimilation. In this section we have used the regression operator defined in Section~\ref{regressionOperator} for Part-EnKF. This choice have been motivated for accelerate the update process.
in Figure \ref{fig:np}. We observe that the error for Part-EnKF and Part-Grid-EnKF increases proportionally with $d_p$ as for the approximation error. This high error confirme in that case the high effect of the particle approximation for Part-EnKF wherease the Remesh-EnKF error is relatively low taking advantages of a high order projection interpolation scheme. Using a regression operator to approximate the analyzed solution should alleviate this effect provided other particle discretization considerations (distortion, support size) as illustrated in Part~\ref{App_1D} and the choice of an adequate penalty coefficient to succeed in approximating the solution.


This discussion pointed out the high dependency of the Part-EnKF on particle discretization. It could be understood that the particle discretization of a member may be too far from the solution support of particles. This opens the question of the choice or modification of the particle discretization. As suggested in Part~\ref{App_1D}, an error estimation could be introduced to choose between the different filters. On the other hand, other approaches could be to select the member with the maximum likelihood estimate to approximate the solution. This proposition has to be evaluated because it could considerably reduce the variance of the ensemble. Finally, an alignment of particle might be introduced to better fit the analysed solution.

\newpage


% % !TEX root = main.tex

\section{Conclusion}
In this study, we introduce a novel framework that combines a sequential ensemble data assimilation approach with particle-based models. Specifically, we have developed two novel Ensemble Kalman Filter schemes dedicated to meshfree Lagrangian simulations. These formulations depend either on updating the particle quantities or on remeshing the particle discretization on a common ensemble grid.

We proposed two update strategies relying on an update matrix that does not directly depend on the state discretization. The first strategy, Remesh-EnKF, is based on the projection of every member onto a new common discretization of particles. The second strategy, Part-EnKF, evaluates the analysis field at particle locations to update the particle quantities.

The different classes have been initially tested on a one-dimensional example and compared with an Eulerian representation of the solution.The results demonstrate comparable performance across the various filters, with the exception of certain configurations of the Part-EnKF. It has been demonstrated that provided the support of the particles is consistent with the analysis solution, the filters yield similar results. However, in cases where the support deviates from the analysis field, members diverge. Increasing the support for the solution is necessary.

A two-dimensional case has also been tested, particularly to assess nonlinear advection schemes with various configurations. We observe good agreement among the different filters. However, this time, the particle approximation is predominant in the Part-EnKF.
Both strategies offer several derivations. Remesh-EnKF is mainly dependent on the redistribution kernel to obtain the new regularly spaced particle set. Part-EnKF could be extended because it is highly flexible when defining the new set of particles for each member. This tuning is essential, as we have highlighted the issue of the non-conforming support of the forward particle position with the analysis.

Several methodologies to introduce new particles or change the previous ones could be explored, particularly at the edge of the distribution.
For the sake of simplicity, we advocate generating a new set of particles with a regular spacing for any member encountering difficulties in accurately reconstructing the analysis field.
Another alternative is to update the positions of the particles in addition to the intensities instead of only changing the intensities. In this way, the error due to a misfit of alignment or approximation could be avoided. These type of adaptation could be derived from optimal transport scheme~\cite{bocquet_bridging_2023} or correction of the background ensemble alignment with the observation~\cite{ravela_data_2007,rosenthal_displacement_2017}.

\newpage
% !TEX root = main.tex

\appendix
\section{Parameters}
\label{appendix:simulation-parameters}
\begin{table}[htbp]
    \centering
    \caption{Reference parameters}
    \begin{tabular}{|l|l|}
        \hline
        Parameters            & Values                                                       \\
        \hline
        reference viscosity   & $v_{\text{ref}} = 0.001$                                     \\
        reference orientation & $\theta_{\text{ref}}  = \frac{7 \pi}{8} (\text{rad.})$       \\
        barycenter position   & $\bm{z}_{\text{ref}} = \left[\frac\pi2, \frac\pi2 \right]^T$ \\
        translation velocity  & $U_{\text{ref}} = 0.25$                                      \\
        \hline
    \end{tabular}
    \label{tab:ref}
\end{table}

\begin{table}[htbp]
    \centering
    \caption{Nominal assimilation and simulation parameters}
    \begin{tabular}{|l|l|}
        \hline
        Parameters                      & Values                          \\
        \hline
        time step                       & $dt = 0.005$                    \\
        final time                      & $tf =10$                        \\
        std. observation                & $\sigma_{obs} =  5.0e-2$        \\
        vorticity threshold             & $\varepsilon_{\omega} = 1.0e-4$ \\
        particle characteristic length  & $dp = 0.01227 $                 \\
        smoothing length                & $h = 2.0 dp$                    \\
        number of assimilation          & $N_{\text{assim}} = 10$         \\
        ensemble size                   & $N_{\text{ens}} = 32$           \\
        number of observation           & $N_{\text{obs}} = 12^2 = 144$   \\
        grid discretization             & $N_{\text{grid}} = 65^2 = 4225$ \\
        number of remeshing by forecast & $N_{\text{remesh}} =  2 $       \\
        \hline
    \end{tabular}
    \label{tab:simu_2d}
\end{table}
\begin{table}[htbp]
    \centering
    \caption{Ensemble generation variables.}
    \begin{tabular}{|l|l|}
        \hline
        Variables   & Distributions                                                                                                           \\
        \hline
        radius      & $R \sim \cN(1.0, 0.05^2)$                                                                                               \\
        orientation & $\theta \sim \cU\left(\pi, \frac\pi2 \right) (\text{rad.}) $                                                            \\
        barycenter  & $z_{\text{mean},x} \sim \cN\left(\frac\pi2,0.1^2\right), \quad z_{\text{mean},x} \sim \cN\left(\frac\pi2,0.1^2\right) $ \\
        velocity    & $U \sim \cU(0, 0.5^2) $                                                                                                 \\
        viscosity   & $v \sim \cN(0.0015, 0.0005^2)$                                                                                          \\
        \hline
    \end{tabular}
    \label{tab:ens_dipole}
\end{table}

\newpage
% \section*{References}

\bibliography{C:/Users/md266594/mar-dvd-thesis/articles/new_part_enkf/biblio}

% \nocite{*}
% \printbibliography

\end{document}