% !TEX root = main.tex

\subsection{Particle-based methods}~\label{Background_Part}
We consider particle-methods for solving continuous problems in fluid or solid mechanics. The Lagrangian methods decompose the domain on a set $\mathcal P = \left\{(\bm z_p, \bm U_plef)\right\}_{p = 1}^{n_p}$, où $\bm z_p \in \Omega$, the spatial coordinates a particle $p$ and $\bm U_p \in \mathbb{R}^m$ its intensity.

% The velocity field $\bm{v}$ is used to update the position of the particle such as $\bm z_p(t+dt) = \bm z_p(t) + f(z_p, \bm{v}{\bm z_p})$ with $f$ depending on the time-integration scheme.

% The computation of the velocity field and the solving of the equation of mechanics depend on the class of method.
We focus our work on methods that discretize a solution $\bm u$ on a continuous domain $\Omega \subset \mathbb{R}^d$ with $\Omega$ the spatial domain. This includes methods like Smoothed particle hydrodynamics (SPH) \cite{gingold_monaghan_sph_1977,lucy_1977} and the Vortex Method (VM) \cite{cottet_vortex_2000} and is extended to other methods like the Material Point Method (MPM) \cite{sulsky_particle_1994}.

\subsubsection{Particle discretization}

Any smooth field $\bm u$ on $\Omega$ could be written

\begin{equation*}
	\bm u(\bm z) = \int_{\Omega} \bm u(\bm z') \delta(\bm z' - \bm z)  d\bm z',
\end{equation*}with $\delta$ the Dirac distribution.

A kernel function $\phi_\varepsilon$ is introduced to obtain an average estimate $\langle \bm u \rangle$ of $\bm u$ such that

\begin{equation*}
	\langle \bm u(\bm z) \rangle = \int_{\Omega} \bm u(\bm z') \phi_\varepsilon(\bm z-\bm z') d\bm z,
\end{equation*}where $\varepsilon$ is the smoothing length. The smooth kernel should at least respect the following properties

\begin{eqnarray*}
	&& \int_{\Omega} \phi_\varepsilon(\bm z) d\bm z = 1,      \\
	&& \phi_\varepsilon(\bm z) \to \delta(z), \quad \varepsilon \to 0, \\
	&& \phi_\varepsilon(\bm z) \in C^k,  \quad k \geq 1,
\end{eqnarray*} where the two first properties are remanent properties of the Dirac distribution and the last is a differentiability requirement.

The average function $\langle \bm u \rangle$ is then used to approximate the original function.

Finally, the original domain $\Omega$ is subdivided with $N_p$ subdomain $\Omega_p$ associated with a lagrangian particle in the location $z_p \in \Omega_p$. We denote by $V_p$ the volume of $\Omega_p$. This discretization is then used to approximate the average function such that

\begin{eqnarray}~\label{part_approx}
	&\approx& \sum_{p \in \mathcal P} \bm U_p \phi_\varepsilon (\bm z-\bm z_p).
\end{eqnarray}

Thus, any function defined on a particle discretization is defined by a particle discretization $\mathcal{P}$ and a smoothing kernel $\phi_\varepsilon$.
Based on this discretization, the differential operator could be derived through this formulation.

Several kernels have been used depending on the method. Theoretically, it could be the Gaussian kernel function

\begin{equation*}
	\phi_g(r) = \frac{1}{{(\pi \varepsilon^2)}^{d/2}} \exp(-r^2/\varepsilon^2).
\end{equation*}

This kernel is infinitely differentiable but defined on non-compact support. In practice, we use a cut-off to remove negligible value for large distance from a particle.

Other kernels, based on B-Spline functions to work on a compact support.

\subsection{Particle-based function manipulations}~\label{operators}

Based on particle discretization, we present several particle manipulation that will used in our methods. Originally, those manipulation are either dedicate to improve the quality of the approximation, avoid high distorsion by creating a new particle discretization or to project the solution on a Eulerian configuration. The different operators will be used in the assimilation process in
order to update particle solution of each members in Section~\ref{Methods}.

\subsubsection{Approximation operator}~\label{interpOp}

The first category of manipulations aims to improve the approximation of the field by modifying particle strength.
A first approximation could be to use the particle approximation to reevaluate the particle intensities like in Equation \ref{part_approx} such as

\begin{equation*}
	\bm U_p = \int_{\Omega_p} \bm u(z) d\bm z = \bm u(\bm z_p) V_p,
\end{equation*}~where $\bm z_p$ is the particle location.

This approximation is easily computable but do not ensure the conservation of the all the moment of the field. A better approximation could be obtain using the iterative Beale's formula \cite{beale_accuracy_1988} which corrected circulation values in order to recover the vorticity field at the particle locations.

\subsubsection{Regression operator}~\label{regressionOperator}

Based on regression methods, the new intensities of the particles defined $\bm{U} = [\bm U_1, \dots, \bm U_p]^T$ could be obtain by minimizing the quadratic error. Assume that we have some vector $\bm{u} = [\bm u_1(z_1), \dots, \bm u_p(\bm z_p)]^T$ of the continuous field evaluations. The particle approximation could be compute on each particle positions $\bm z_p$ given

\begin{equation*}
	\bm{u} \simeq \bm \Phi \bm{U},
\end{equation*}where $\bm \Phi_{ij} = \phi_\varepsilon(z_i - z_j)$.

Finding the new intensities $\bm U^*$ correspond to solving the previous system in the least square sense. It corresponds to the classical problem to find the minimizer of the following quadratic function

\begin{equation*}
	\bm{U}^*= \argmin_{\bm{U}} \norm{\bm{u} - \Phi \bm{U}}^2_2.
\end{equation*}


In this case, the solution is $\bm U^*  = (\bm \Phi^T \bm \Phi)^{-1} \bm \Phi \bm{u}$. This problem may be ill-posed, particularly in the case of large set of non well-distributed particles. We choose to regularized the solution by introducing a penalization term. The Ridge regression introduce a penalization on of the form $\lambda \norm{\bm U}_2^2$, where $\lambda$ is a penalization coefficient, such as the new problem is

\begin{equation*}
	\bm{U}_{\text{ridge}}^* = \argmin_{\bm{U}} \norm{\bm{u} - \bm \Phi \bm{U}}_2^2 + \lambda \norm{\bm{U}}^2_2,
\end{equation*}given the following solution $\bm{U}^*_{\text{ridge}} = (\bm \Phi^T \bm \Phi + \lambda \bm I)^{-1} \bm \Phi \bm{u}$.

\subsubsection{Remeshing operator}~\label{remesh_part}
A second type of manipulation is based this time on a complete projection of the solution on a new regular grid of particles~\cite{cottet_vortex_2000,cottet_multi-purpose_1999}. This method allow us to switch from an Lagrangian discretization $\mathcal P$ to an Eulerian one $\Lambda$, and then switch back to a completely new regular particles discretization $\mathcal P'$ that conserve as much as possible the moment of the particle discretization.

In our methodology, we propose a two-step approach. First, we execute an assignment step (\ref{assigment}) to transfer the particle discretization to the grid discretization. Subsequently, an interpolation step (\ref{interpolation}) is performed to yield a new set of regularly spaced particles.

Our analysis pertains to the one-dimensional spatial scenario, where $\Omega \in \bR$. The extension to the $n$-dimensional case can be achieved through the tensorization of the one-dimensional approach.

\begin{enumerate}[label=(\alph*)]
	\item  \textit{Assignment on an Eulerian grid} \label{assigment}

	      We denote by $z_{I}$ and $z_{p}$ respectively the grid and the old particle locations. The new particles are defined on a grid of $n_g$ elements with regular spacing $\ell_I = 2 d_p$ where $d_p$ is the characteristic size of the particles. We define the particle intensities as $\bm U_p$ and the nodal field values as $\bm u_I$. By using some shape function $W$, the assignment step from particles to each node $I \in \Lambda$ can be written as

	      \begin{equation*}
		      \bm{u}_I = \frac1{V_I} \sum_{p \in \mathcal P} \bm U_p  W \left(\frac{z_I - z_p}{\ell_I} \right).
	      \end{equation*}

	      Where $W$ determine a redistribution of the intensity on the grid. The new discretization can then be used to approximate the field $\bm{u}_p$, defined by the particle discretization by interpolation given

	      \begin{equation*}
		      \bm{u}_p(z) \approx \bm{u}_g(z) = \sum_{I \in \Lambda} \bm u_I W \left(\frac{z - z_I}{\ell_I} \right) \quad \forall z \in \Omega.
	      \end{equation*}
	\item  \textit{Interpolation on a new regular particle discretization} \label{interpolation}

	      A new set of particles is define at the quarter of each cells such that the new position are define at $z_{p'} = d_p/2 + i~dp, \quad i = 0,\dots, 2n_g $. The value of the field is then interpolate at that new location and multiply with the volume of the particle $\bm{U}_{p'} = \bm  u_g(z_{p'}) V_{p'}$ in order to give a new particle approximation of the field

	      \begin{equation*}
		      \bm{u}_g(z)  \approx \bm{u}_{p'}(z) = \sum_{p'\in\mathcal P'} \bm{u}_g(z_{p'}) V_p,
	      \end{equation*}.
\end{enumerate}

The combination of these two steps can initially be utilized to generate a new undistorted particle distribution.
The shape function $W$ determines the type and quality of the transfer. The method effectiveness is evaluated by assessing the conservation of the first moments of the particle distributions, as detailed in the Appendix \ref{appendix:moment_conservation}.

For $W$, one may employ the piecewise linear interpolation function, which ensures the conservation of moment 0. For higher moment conservation, the B-spline function provides a smoothing function for higher order.

However, while higher-order B-splines improve the smoothness of the solution, their accuracy is limited to second order, allowing only exact interpolation of linear functions.

Monaghan~\cite{monaghan_extrapolating_1985} proposes a systematic approach to enhance accuracy and maintain smoothness through extrapolation. The concept involves constructing a new shape function based on a cutoff and its radial derivative. For $m = 4$, the cubic B-spline is improved by the following new interpolating kernel

\begin{eqnarray*}~\label{cubic_radial_kernel}
	M_4'(z) &=& \left\{ \begin{aligned}
		 & 1 - \frac{5}{2}z^2 + \frac{3}{2} |z|^3 & 0 \leq & |z| \leq  1 & \\
		 & \frac{1}{2}{(2 - |z|)}^2(1 - |z|)      & 1 \leq & |z| \leq 2  & \\
		 & 0                                      & 2 \leq & |z|.
	\end{aligned}
	\right.
\end{eqnarray*}

The drawback of this method is that it does not ensure positivity. Therefore, we opt to utilize the $M_4'$ kernel for our implementation.

Finally, in multidimensional space, the redistribution kernel $W$ can be obtained as the product of the one-dimensional kernel applied to each coordinate, as follows
\begin{eqnarray*}
	\bm U_p &=& \sum_{I \in \Lambda} \bm U_I  W \left(\bm z_p - \bm z_I, \ell_I \right) \\
	&=&  \sum_{I \in \Lambda} \bm U_I  \prod_{i = 1}^d W_{1\text{D}} \left(\frac{\bm z_{I, i} - \bm z_{p, i}}{\ell_I} \right)
\end{eqnarray*}