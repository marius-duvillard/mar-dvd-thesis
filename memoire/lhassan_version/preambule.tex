% !TEX root = ./main.tex

\usepackage[utf8]{inputenc}
% \renewcommand{\familydefault}{\sfdefault}
\usepackage{geometry}
\usepackage[pdftex]{graphicx}

\graphicspath{ {../MeshlessEnKF}{../Introduction/images} {../MeshlessMethods/images} {../AlignEnKF/images} {../BiblioIntro/images} {../DataAssimilation/images}}
% \geometry{
%     left=16mm,
%     top=30mm,
%     right=16mm,
%     bottom=30mm
% }
\usepackage{appendix}
\usepackage{etoolbox}
\usepackage{xcolor}
\usepackage[absolute,overlay]{textpos}
\usepackage{lipsum}
\usepackage{array}
\usepackage{enumitem}
\usepackage{caption}
\usepackage{multicol}
\usepackage{bm}
\usepackage{amsmath}
\usepackage{amsfonts}
\usepackage{subcaption}
\usepackage[vlined, french, onelanguage]{algorithm2e}
% Hyperref setup
\usepackage[linktoc=page]{hyperref}
% \hypersetup{
%     colorlinks=true,
%     linkcolor=blue,
%     filecolor=magenta,
%     urlcolor=cyan,
%     pdftitle={Overleaf Example},
%     pdfpagemode=FullScreen,
% }
\hypersetup{
    % General settings
    pdfencoding=auto,           % Automatically handles PDF string encoding
    pdfborder={0 0 0},          % Removes borders around links
    breaklinks=true,            % Allows links to break across lines
    colorlinks=true,            % Colors the text of links and anchors
    % Link colors
    linkcolor=blue,             % Color of internal links (e.g., sections, equations)
    citecolor=green,            % Color of citation links
    filecolor=magenta,          % Color of file links
    urlcolor=cyan,              % Color of external hyperlinks
    % Metadata
    pdftitle={thesis duvillard},       % Title of the document
    pdfauthor={duvillard},              % Author of the document
    % pdfsubject={thesis},          % Subject of the document
    % pdfkeywords={keyword1, keyword2},   % Keywords for the document
    % PDF display settings
    pdfpagemode=UseOutlines,    % Opens the PDF with the bookmarks pane open
    bookmarksopen=true,         % Open the bookmarks by default
    bookmarksnumbered=true,     % Include section numbers in bookmarks
    unicode=true                % Support for non-ASCII characters in bookmarks
}
\urlstyle{same}
\setlength{\columnseprule}{0pt}
\setlength\columnsep{10pt}
\usepackage[most]{tcolorbox}
\usepackage{amsthm}
\theoremstyle{definition}
\newtheorem{definition}{Definition}%[section]

\definecolor{mycustomcolor}{RGB}{128, 0, 128}
\newcommand{\mycolor}[1]{\textcolor{mycustomcolor}{#1}}
\newcommand{\bE}{\mathbb{E}}
\newcommand{\bV}{\mathbb{V}}
\newcommand{\bC}{\mathbb{C}}
\newcommand{\nens}{N}
\newcommand{\bw}{\bm w}
\DeclareMathOperator*{\argmax}{arg\,max}
\DeclareMathOperator*{\argmin}{arg\,min}
\newcommand{\bx}{\bm{x}}
\newcommand{\mstate}{\bm{Z}}
\newcommand{\statebis}{\bm{x}}
\newcommand{\mstatebis}{\bm{X}}
\newcommand{\zz}{z_0 = 0.02}
\newcommand{\sigz}{\sigma_0^2 = 0.5}
\newcommand{\xx}{x_0 = 0.02}
\newcommand{\sigx}{\sigma_0^2 = 0.5}
\newcommand{\npart}{$N_{part} = 100$}
\newcommand{\ngrid}{$N_{grid} = 100$}
\newcommand{\sigmaY}{0.05}
\newcommand{\z}{\bm{z}}

\newcommand{\mpred}{\mathcal{Y}}
\newcommand{\mP}{\mathcal{P}}
\newcommand{\Fcorr}{\bm{F}}
\newcommand{\mdata}{\bm{D}}
\newcommand{\state}{\bm{z}}
\newcommand{\annomX}{\bm A}
\newcommand{\annomY}{\bm Y}
\newcommand{\obs}{\bm{y}}
\newcommand{\Cov}{\bm{P}}
\newcommand{\predi}{\mathcal{H} (\bm{x}^i_f)}
\newcommand{\sigmaZm}{0.5}
\newcommand{\meanZm}{\pi/2 + 0.6}
\newcommand{\bz}{\bm z}
\newcommand{\visc}{D}
\newcommand{\refv}{1.0}
\newcommand{\refvisc}{0.05}
\newcommand{\Dlow}{0.02}
\newcommand{\Dup}{0.08}
\newcommand{\vmean}{0.9}
\newcommand{\vstd}{1.2}
\newcommand{\smLow}{0.8}
\newcommand{\smUp}{1.2}

\newcommand{\by}{\bm{y}}
\newcommand{\bP}{\bm{P}}
\newcommand{\bR}{\bm{R}}
\newcommand{\bh}{\bm{h}}
\newcommand{\bd}{\bm{d}}
\newcommand{\fstate}{\bm{u}}
\newcommand{\cN}{\mathcal N}
\newcommand{\cU}{\mathcal U}
\newcommand{\cP}{\mathcal P}
\newcommand{\cV}{\mathcal V}

\newcommand{\bM}{\bm{M}}
\newcommand{\bH}{\bm{H}}
\newcommand{\bK}{\bm{K}}

\DeclareMathOperator{\Tr}{Tr}
\usepackage[most]{tcolorbox}

% Définition du modèle "boite1"
\newtcolorbox{ObjectifChap}[1][]{
    colback=blue!5!white,
    colframe=blue!75!black,
    fonttitle=\bfseries,
    title=Boîte 1,
    #1
}

% Définition du modèle "boite2"
\newtcolorbox[auto counter]{BilanChap}[1][]{
    colback=red!5!white,
    colframe=red!75!black,
    fonttitle=\bfseries,
    title=Boîte \thetcbcounter,
    #1
}
\newcommand{\norm}[1]{\left\lVert #1 \right\rVert}

\usepackage[french]{babel}
\usepackage{biblatex}
\addbibresource{../biblio.bib}
% \bibliographystyle{plain}
