% !TEX root = main.tex

\section{Développement de méthodes permettant l'adaptation du filtre de Kalman d'ensemble avec des simulations sans maillage}

\subsection{Objectif}
e filtre de Kalman présenté en Section~\ref*{sec:enkf} est un filtre séquentiel adéquat pour appliquer des méthodes d'assimilation pour des modèles de grande dimension et non-linéaire. D'autre part, dans la Section~\ref{sec:method_part}, nous avons montré la compatibilité des méthodes sans maillage pour l'assimilation de données séquentiel. Nous avons montré alors que les méthodes sans maillage discrétisant un milieu continu pouvait être mis à jour lors de l'assimilation en modifiant positions et intensités des particules.

Toutefois, la mise à jour du filtre de Kalman d'Ensemble est dépendante de la discrétisation de l'état de ses membres pour le calcul du gain de Kalman d'ensemble. De plus, la mise à jour du filtre de Kalman d'Ensemble est basé sur une combinaison linéaire des membres au travers du gain de Kalman d'ensemble, celle-ci entraîne une explosion du nombre de particules dans la définition de la solution analysée.

L'objectif de ce chapitre est donc de présenter un certain nombre d'adaptation du filtre de Kalman d'Ensemble qui puisse être appliquées à des simulations sans maillage.

Pour cela, nous reformulons tout d'abord l'expression de la mise à jour afin qu'elle soit indépendante de la définition de l'état.
Enfin, les solutions analysées étant des combinaisons linéaires des particules de l'ensemble des membres, nous proposons de développer des méthodes pour réduire l'augmentation exponentielle des particules.

\subsection{Formulation de la correction dans l'espace des membres}

\subsection{Filtre Remesh-EnKF}
\subsubsection{Méthode de remaillage}
\subsubsection{Définition du Filtre}
\subsection{Filtre Part-EnKF}
\subsubsection{Méthode de régression}
\subsubsection{Définition du Filtre}

\subsubsection{}

\subsection{Bilan}

Nous avons développé deux adaptations du filtre EnKF adaptées au cas des simulations particulaires. Celle-ci tienne compte d'une augmentation exponentielle de particules.
Ces deux adaptations Remesh-EnKF et Part-EnKF correspondent à deux paradigmes. Dans le premier cas, le choix a été fait de regénérer complètement la discrétisation à l'aide de méthode de transfert particule à grille et de remaillage. Dans le second cas, le choix a été fait de conserver la position des particules de chaque membre et d'approcher la solution analysée.
Si ces filtres offre des adaptations du filtre de Kalman d'Ensemble, ils semblent souffrir de plusieurs limitations inhérante à leur schéma. Dans la prochaine section, leur capacité d'assimilation va être vérifié sur plusieurs applications données.

\section{Evaluation de la capacité des méthodes développées à assimiler les données sur plusieurs applications}

Deux filtres dérivée de EnKF ont été développés. A l'aide deux deux applications basées sur une méthode


\subsection{Bilan}