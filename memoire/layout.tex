\documentclass{article}
\usepackage[utf8]{inputenc}
\usepackage[french]{babel}

\title{Plan Manuscript}
\author{Marius Duvillard}

\begin{document}

\maketitle
\tableofcontents
\newpage

\section{Introduction}

\subsection{Contexte industriel}

\subsubsection{Fabrication du combustible de fission}
- voir Giraud: p.1-6
- voir Orozco: p.3-9

\subsubsection{Broyeur à boulet}
- Orozco ?

\subsubsection{Régimes d'écoulement}
- voir pouliquen.pdf
- voir Orozco

\subsubsection{Méthodes de mesures}
- voir Bastien + dossier mesures
% biblio
\subsubsection{Concept de Jumeau Numérique}
- voir session FJOH

\subsubsection{Objectif: Appliquer assimilation de données à ces modèles}
\section{Assimilation de données}

\subsection{Approches stochastiques}

\subsubsection{Modèle stochastique du système}

Inspiré de 3.4.2 de Asch, et Carpentier p.41
\subsubsection{Probability formula}

\subsubsection{Estimation}

Carpentier, chapitre 2, pages 27-36
Asch pages 78-82
Evensen 2.1.7 inférence bayésienne
\subsection{Filtre Bayésien}

Carpentier page 42
Asch page 91
Evensen 2.2
\subsubsection{Filtre particulaire}

3.7 de Asch
CoursEC section 5
\subsubsection{Formulation variationnelle (3DVar)}

\subsubsection{Méthodes Hybrides - RML}

\subsection{Filtre de Kalman}

\subsubsection{Filtre de Kalman d'Ensemble}

Bocquet, Lecture 2
CoursEC 7.2

\section{Modélisation physique (Méthodes particulaires)}

\subsection{Méthode de simulation des écoulements granulaires dans un tambour en rotation}

- voir Arseni 2020
- voir EFEM
- Mishra / Orozco / Chong / Chandra / Zuo / Zhu
- Présenter les méthodes continues et discrètes (voir cours PARTICLES) dans une perspective d'assimilation de données

\subsection{Présentation DEM}

\subsection{Méthode SPH}
% \subsection{Présentation méthodes continues voir généralité: noyau, discrétisation particulaire}
\subsection{Méthode MPM-PIC}
\subsection{Méthode VM $\rightarrow$ Problème fluide incompressible et similarité avec SPH / VIC et MPM}

\subsection{Contenu et objectif}

\section{Ensemble Data Assimilation pour la simulation particulaire - Article 1}

\subsection{Adaptation du Filtre de Kalman d'Ensemble}
--> choix d'une formulation in ensemble space à partir des mesures

\subsection{Focus approximation des méthodes particulaires}
--> approximation et regression

\subsection{Schéma de remaillage}
--> Redistribution

\subsection{Focus problème VM}
--> Cas test

\subsection{Filtres adaptés}

\section{Data Assimilation par alignement de champs}
--> Placer la biblio dans la partie 1.2 si associée à DA ou dans un 1.3 si trop dfférent (ex: OT)

\section{Conclusion}

\end{document}
