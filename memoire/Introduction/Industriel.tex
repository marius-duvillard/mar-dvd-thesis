% !TEX root = main.tex

\chapter{Introduction}

\section{Contexte industriel}

\begin{itemize}
    \item 2023 la filière nucléaire produit X\% de la quantité d'énergie totale d'électricité en france.
    \item Le parc est constitué de 56 tranches réacteurs à eau pressurisée (REP) répartis sur 18 centrales.
    \item Source d'énergie de bon rapport qualité prix et qui est faiblement émetrice en gaz à effet de serre
    \item En effet, le facteur d'émission {mettre la définition}, est estimé à 12 g$CO2$eq/kWh. Il serait même encore plus faible en France. \cite{schlomer_technology-specific_nodate}.
    \item Même ordre de grandeur que l'éolien et le solaire. 100 à 1000 fois plus faible que l'énergie émis par les centrales à énergie fossiles.
    \item Face à l'urgence climatique, énergie nucléaire est une aide pour la transition énergétique et la sortie des énergies fossiles.
    \item Mais il y a des questions à répondre.
    \item Outre les questions relatives à la sureté des installations, ou bien le risque de prolifération nucléaire il faut aussi tenir compte des déchets à vie longues et à forte activité. Ceux-ci se forme à l'intérieur des réacteurs durant toute la durée de leur fonctionnement.
    \item Finalement 2kg de déchet par an et par habitant sont généré pour la fabrication de l'électricité. Parmis eux les déchets à vie longue que représente 0.2\% des stocks pour 95\% de l'activité.
    \item De plus on souhaiterait préserver les ressources en combustible.
    \item Ainsi le retraitement et la fermeture du cycle.
\end{itemize}

En 2023, l'industrie nucléaire a représenté X\% de la production totale d'électricité en France, avec un parc composé de 56 réacteurs à eau pressurisée (REP) répartis sur 18 centrales. Cette source d'énergie est saluée pour son rapport qualité-prix avantageux et ses faibles émissions de gaz à effet de serre, avec un facteur d'émission estimé à 12 gCO2eq/kWh, une valeur encore plus basse en France. Comparativement à d'autres sources telles que l'éolien et le solaire, les émissions de CO2 du nucléaire sont du même ordre de grandeur, mais jusqu'à 1000 fois inférieures à celles des centrales à énergie fossile. Dans le contexte de l'urgence climatique, l'énergie nucléaire est perçue comme un atout pour la transition énergétique et la réduction des énergies fossiles, bien que des questions subsistent, notamment sur la sûreté des installations et le risque de prolifération nucléaire. De plus, la gestion des déchets radioactifs à vie longue et à forte activité, ainsi que la préservation des ressources en combustible, sont des défis à relever, d'où l'importance du retraitement et de la fermeture du cycle du combustible nucléaire.

Le combustible classique utilisé dans les REP est constitué d'oxide uranium enrichi
\subsubsection{Le combustible Nucléaire}

\subsubsection{La fabrication du combustible}

\subsubsection{Broyeur à boulets}