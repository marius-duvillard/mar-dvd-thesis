% !TEX root = main.tex

\chapter{Introduction}

\paragraph{energie et broyage}
Preambule sur le broyage: De nos jours, une préoccupation majeure est la consommation d'énergie. En effet, la rareté des ressources naturelles motive la recherche de nouvelles sources d'énergie et l'optimisation des processus industriels qui consomment de grandes quantités d'énergie, comme le broyage. Le grand nombre de technologies conçues pour la réduction de la taille des particules est spécialisé pour un type de matériau donné et/ou un mécanisme de broyage donné. Cette spécialisation actuelle a été principalement acquise de manière empirique par des tests de tâtonnement.

\paragraph{Sur le broyage sa définition} naturel et artificiel
By definition, grinding is a process in which a material is reduced to small particles or powder. The particle breakage can be a desired or undesired process. It is desired when it belongs to a human activity or a chain of processes that have as a main goal to refine a natural source into a product that matches specific conditions, e.g. wheat, cosmetics and cement processing. It is undesired, unexpected or uncontrolled when it takes place during a natural process e.g. rock fall down a slope, landslides, explosions during the eruption of a volcano, earthquakes, etc . . . The forces that cause the rupture of the constitutive particles of the material often have a dynamic character. These forces are highly variable in time, can reach magnitudes much higher than the particle weights, and are therefore comparable to those involved during the impacts inside ball mills.

\paragraph{Sur les appareils de broyage, la comprehension de la physique et le choix des paramètres de contrôle}
Despite numerous specialized grinding devices, its technological importance, and long past research, there is still a substantial gap between the present fundamental knowledge of the physical mechanisms ruling the grinding process and the present need for its predictive and quantitative modeling in view of its improved engineering applications. Undoubtedly, one of the main reasons is the complexity of the process itself due to the continuous changes of material properties. As a consequence, in current industrial applications, the choice of the grinding device and its optimal values of operational parameters remain basically empirical tasks.

\paragraph{Trouver la même chose sur la coulabilité ? Le mélange chez Giraud ?}

\paragraph{crucial stage of nuclear fuel manufacture and recycling: the nuclear powders grinding inside ball mills}



\section{Contexte industriel}

\begin{itemize}
    \item 2023 la filière nucléaire produit X\% de la quantité d'énergie totale d'électricité en France en Europe il s'agit de près ... de l'électricité.
    \item Le parc est constitué de 56 tranches réacteurs à eau pressurisée (REP) répartis sur 18 centrales.
    \item En fait la demande mondiale ne fait qu'augmenter, la demande dans les pays en développement et émergeant en étant la cause principale. De plus, avec l'urgence climatique et la raréfication des sources d'énergie fossile encourage à développer des ressourses de faible émission carbonne.
    \item Le nucléaire se trouve être une source d'énergie de bon rapport qualité prix et qui est faiblement émetrice en gaz à effet de serre
    \item En effet, le facteur d'émission {mettre la définition}, est estimé à 12 g$CO2$eq/kWh. Il serait même encore plus faible en France. \cite{schlomer_technology-specific_nodate}.
    \item Même ordre de grandeur que l'éolien et le solaire. 100 à 1000 fois plus faible que l'énergie émis par les centrales à énergie fossiles.
    \item Face à l'urgence climatique, énergie nucléaire est une aide pour la transition énergétique et la sortie des énergies fossiles.
    \item Mais il y a des questions à répondre.
    \item Outre les questions relatives à la sureté des installations, ou bien le risque de prolifération nucléaire il faut aussi tenir compte des déchets à vie longues et à forte activité. Ceux-ci se forme à l'intérieur des réacteurs durant toute la durée de leur fonctionnement.
    \item Finalement 2kg de déchet par an et par habitant sont généré pour la fabrication de l'électricité. Parmis eux les déchets à vie longue que représente 0.2\% des stocks pour 95\% de l'activité.
    \item De plus on souhaiterait préserver les ressources en combustible.
    \item Ainsi le retraitement et la fermeture du cycle.
\end{itemize}

En 2023, l'industrie nucléaire a représenté X\% de la production totale d'électricité en France, avec un parc composé de 56 réacteurs à eau pressurisée (REP) répartis sur 18 centrales. Cette source d'énergie est saluée pour son rapport qualité-prix avantageux et ses faibles émissions de gaz à effet de serre, avec un facteur d'émission estimé à 12 gCO2eq/kWh, une valeur encore plus basse en France. Comparativement à d'autres sources telles que l'éolien et le solaire, les émissions de CO2 du nucléaire sont du même ordre de grandeur, mais jusqu'à 1000 fois inférieures à celles des centrales à énergie fossile. Dans le contexte de l'urgence climatique, l'énergie nucléaire est perçue comme un atout pour la transition énergétique et la réduction des énergies fossiles, bien que des questions subsistent, notamment sur la sûreté des installations et le risque de prolifération nucléaire. De plus, la gestion des déchets radioactifs à vie longue et à forte activité, ainsi que la préservation des ressources en combustible, sont des défis à relever, d'où l'importance du retraitement et de la fermeture du cycle du combustible nucléaire.

Le combustible classique utilisé dans les REP est constitué d'oxide uranium enrichi
\subsection{Le combustible Nucléaire}

\subsection{La fabrication du combustible}

\subsubsection{Broyeur à boulets}



Nous nous intéressons à l'étape de mélange-broyage dans le procédé de fabrication du MOX à l'aide du broyeur à boulet. Un mélange d'oxyde d'uranium, d'oxyde de plutonium et de chamotte est concassé dans un broyeur à boulet. Ce dispositif cylindrique, rempli de boules de broyage, met en oeuvre un processus de rotation pour broyer finement le mélange de poudres d'oxyde. Sa performance est critique pour la qualité du produit final et sa sûreté d'utilisation dans les réacteurs nucléaires.


