
% A mettre dans la partie simulation du tambour et méthode particulaire
\subsection{Ecoulement granulaire et régime dans le broyeur}
Les écoulement granulaire sont des mécani


\subsection{Rhéologie des milieux granulaire sous conditions homogènes}

Les travaux récents convergent vers une loi de comportement viscoplastique pour modéliser les milieux granulaires défini sous le nom de loi $\mu(I)$ \cite{gdr_midi_dense_2004,jop_constitutive_2006}.
C'ets tout d'abord dans \cite{POULIQUEN_FORTERRE_2002} que le coefficient de friction $\mu$ suit la loi suivante

\begin{equation*}
    \mu(I) = \mu_1 + \frac{\mu_2- \mu_1}{I_0/I + 1},
\end{equation*}où $mu_2, \mu_1$ sont deux constantes liés au matériau et $I_0$ une constante qui dépend de la configuration de l'écoulement. Finalement $I$ est appelé le nombrre inertiel. Ce nombre adimentionnel est défini comme le rapport entre le temps de relaxation $\tau_r = \sqrt{m/P}$ et le temps de cisaillement $\tau_c = 1 / \dot \gamma$

\begin{equation*}
    I = \frac{\tau_i}{\tau_c} = \frac{\dot \gamma d}{\sqrt{P/\rho_g}},
\end{equation*}où $\dot \gamma$ est le taux de cisaillement, $P$ la pression de confinement, $d$ la taille caractéristique des particules, et $\rho_g$ la densité du grain.

Cette loi été étendu au cas tri-dimentionnel et testé sur un certain nombre de cas en particulier celui du tambour en rotation \cite{Cortet_2009} et montre une bonne correspondance pour le cas d'écoulement avec surface libre \cite{chou_cross-sectional_2009}. Toutefois, cette loi trouve certaines limites dans le cas d'écoulement confiné où le coefficient de tassement change et où le mouvement de chaque grain entraîne des modifications significatives dans les chaînes de force. Si la prédiction est bonne au niveau des bords, elle reste toutefois insufisante au niveau des parois \cite{Rognon_Miller_Metzger_Einav_2015}.
De plus, elle ne permet pas de prendre en compte les régimes cinétiques, c'est à dire le régime gaz. Ainsi, le développement de nouvelles loi rhéologique est un sujet de recherche constant.

\subsection{Ecoulement dans le tambour en rotation}

Dans le tambour en rotation l'ensemble des trois zones d'écoulement sont présentes. Une diversité de six régimes d'écoulement est généralement étudié: glissement, ballotement, éboulement, roulement, en cascade, cataracte,  centrifuge \cite{MELLMANN2001251}.
Ces différents régimes déterminent la qualité du mélange, du broyage.
C'est le régime en cascade qui est nécessaire pour la réduction de taille de grain dans le broyeur à boulets. C'est dans ce régime que la surface libre prend la forme caractéristique d'un \textit{S}.


\section{Simulation de l'écoulement dans un tambour en rotation}

\subsection{Approche discrète}
\begin{itemize}
    \item Méthodes discrètes traitant les particules comme des objets solides indépendants.
    \item Caractérisation des particules par leur géométrie.
    \item Interaction des particules à travers des lois de contact, de frottement et de cohésion.
    \item Résolution du problème mécanique en décrivant toutes les trajectoires des particules.
    \item Méthode des Éléments Discrets (DEM).
    \item Méthode de Dynamique des Contacts (CD).
    \item Caractère numérique élégant de la méthode CD.
    \item Particularité des particules considérées comme parfaitement rigides dans la méthode CD.
    \item Stabilité du système dans la méthode CD.
    \item Description des vitesses discontinues dans la dynamique des contacts.
    \item Avantages de l'approche DEM en termes de détails des déformations dans les contacts.
    \item Complexité algorithmique de la mise en œuvre de la méthode CD.
    \item Choix d'utiliser la méthode DEM dans la thèse.
    \item Simplicité de programmation de la méthode DEM.
    \item Extension des approches lagrangiennes à "N-corps".
    \item Flexibilité de la méthode DEM dans l'introduction d'interactions complexes entre particules.
    \item Traitement des forces entre particules par des fonctions régulières de la déflexion au contact dans la méthode DEM.
    \item Nécessité parfois d'introduire un terme dissipatif dans la méthode DEM.
    \item Difficulté à justifier le sens physique du terme dissipatif dans certains problèmes.
    \item La première mise en pratique d'une approche par éléments discrets pour les matériaux granulaires a été réalisée par Cundall à la fin des années soixante.
    \item Les approches par éléments discrets reposent sur deux familles de méthodes numériques : les méthodes régulières (Smooth Methods) et les méthodes non-régulières (Nonsmooth Methods).
    \item La méthode dite classique et la méthode de dynamique des contacts (Contact Dynamics ou CD) sont respectivement les méthodes régulière et non-régulière les plus employées à ce jour.
\end{itemize}
\subsection{Approche continue}

\begin{itemize}
    \item Approches basées sur une description continue du milieu granulaire.
    \item Utilisation en ingénierie, notamment pour la modélisation des instabilités dans les sols et la rhéologie des poudres dans les procédés.
    \item Description du comportement à partir de lois constitutives et d'équations de bilan.
    \item Formulation des problèmes sous forme de problèmes aux dérivées partielles.
    \item Capacité à intégrer des conditions aux limites complexes.
    \item Simulation des problèmes à "grande échelle".
    \item Limitation à fournir des informations sur le lien entre la physique des phénomènes à l'échelle des particules et le comportement à l'échelle macroscopique.
    \item Méthodes permettant de simuler le comportement continu.
    \item Méthodes en Éléments Finis (MEF).
    \item Adaptations pour prendre en compte le comportement granulaire.
    \item Modèles poro-elasto-plastiques.
    \item Minimisation de l'erreur introduite en remplaçant le problème continu par un problème discrétisé.
    \item Prise en compte des conditions aux limites complexes et fortement non-linéaires.
    \item Difficultés à rendre compte des grandes déformations.
    \item Nécessité de processus de remaillage coûteux en temps de calcul.
    \item Alternatives plus récentes comme le Material Point Method (MPM).
    \item Description Lagrangienne de points matériels.
    \item Domaine Eulérien maillé pour le calcul des lois de comportement.
    \item Algorithme itératif pour le transfert des informations.
    \item Interpolation des variables entre points matériels et maillage.
\end{itemize}


\subsection{Méthodes discrètes}

L'écoulement d'un milieu granulaire a des comportements complexe sachant qu'il a différents régimes d'écoulement.
De plus, il s'agit d'un milieu soumis à de grandes transformations d'un milieu qui est discret.
De ce fait, les méthodes de simulation doivent tenir compte de ces spécificités.

\mycolor{Pour DEM bien citer les papier des amis thésards}

\paragraph*{Zuo 2020 (MPM)}
Biblio
- deux méthodes: discrètes et continues
- en discret méthode DEM (citer cundall and strack 1979), les particules sont des particules individuels et le mouvement de chacune est calculer grace au PFD.
- Mishra BK, Rajamani RK. The discrete element method for the simulation of ball mills. Appl. Math. Model. 1992;16:598-604. DEM pour Rotating drum.
- 2 points négatifs:
- coût de calcul élevé. En effet, taille de particule très faible p/ à la taille du broyeur.
- Modélisation des particules régulières et plus large que la réalité.
- De l'autre côté, on a des méthodes qui modélise le milieu granulaire comme un milieu continu. Le système est gouverné par les équations de conservation de mass et de quantité de mouvement.
- Les méthodes de discrétisation particulaire sont adaptées pour les problèmes de grandes transformations parce que représentation hybrid Euler-Lagrange du mouvement.
- Il donne des papiers où MPM a été utilisé pour simuler écoulements granulaires.
- On peut aussi coupler MPM et DEM pour interaction avec des solides comme les boulets du broyeur. (Liu C, Sun Q, Zhou GG. Coupling of material point method and discrete element method for granular flows impacting simulations. International Journal for Numerical Methods in Engineering. 2018;115:172-188.)
- Dans ce papier utilise MPM avec une loi de plasticité de Drucker–Prager pour modéliser le mélange dans le tambour en rotation.

\paragraph*{Zhu 2022, SPH}
- use SPH pour étudier l'écoulement dans un broyeur à boulets avec une loi de réologie $\mu (I)$ couplé à un modèle élastoplastique de Drucker-Prager.
- En effet, l'objectif est de traiter correctement des cas de cohabitation de solid liqui régime comme dans le tambour en rotation.

Biblio:
- Au début études étaient purement analytiques et expérimentales (qq citations), mais les modèles sont empiriques donc peu généralisables.
- Ensuite développement de la DEM d'un côté, mais également approche Eulerian FEM (papiers qui peuvent être citer).
- Le souci dans ce dernier cas est de pouvoir gérer les interfaces et les surfaces libres.
- Les méthodes particulaires Lagrangiennes peuvent plus facilement traiter des écoulements à surface libre.
- Présente SPH. Méthode sans maillage basée sur un ensemble de particules transportant des quantités matérielles qui évoluent en fonction des forces internes et externes.
- Sur la loi de comportement: difficile du fait de la cohabitation de différents régimes. Modèle élastoplastique sont souvent utilisé pour le comportement solide. Pour le comportement fluid-like, utilise des loi viscoélastique ou réhologique. En particulier, la loi $\mu (I)$ cherche à modéliser ses deux régimes à partir d'un nombre adimentionnel, le nombre Inertiel $I$.
- il relie the shear strength with the normal stress à travers un coefficient de friction qui dépend de $I$.
- Présente $\mu(I)$.

\paragraph*{Chandra MPM 2021}
- Mets en avant le traitement sensible de la gestion des conditions limites dans le cas du broyeur à boulets et propose dans le cas MPM des méthodes pour les prendre en compte.

\paragraph*{Arseni}
- Utilise 3D FV simulations de l'écoulement granulaire en utilisant la loi $\mu(I)$. Permet de reprodure les différents régimes d'écoulement.

Biblio:
- Beaucoup de modèles ont été proposées.
- Présente (D. A. Santos, I. J. Petri, C. R. Duarte, and M. A. S. Barrozo, “Experimental and CFD study of the hydrodynamic behavior in a rotating drum,” Powder Technol. 250, 52–62 (2013).) comme une étude complète d'une approche Euler-Euler basée sur un milieu continu assumant des interpénétrations et une équation dérivée de la théorie cinétique.
- Précise que $\mu(I)$ permet de prendre en compte à la fois un critère de pasticité et une dépendance complexe de la viscosité en fonction du taux de déformation et de la pression.

# BIBLIO REUNIE

- Au début études étaient purement analytiques et expérimentales (Ding2001,Boateng1998,Nicholas2001), mais les modèles sont empiriques donc peu généralisables.

- Ensuite développement de la DEM d'un côté,
- en discret méthode DEM (citer cundall and strack 1979), les particules sont des particules individuels et le mouvement de chacune est calculer grace au PFD.
- Mishra BK, Rajamani RK. The discrete element method for the simulation of ball mills. Appl. Math. Model. 1992;16:598-604. DEM pour Rotating drum.
- 2 points négatifs:
- coût de calcul élevé. En effet, taille de particule très faible p/ à la taille du broyeur.
- Modélisation des particules régulières et plus large que la réalité.
- Bon caractère prédictif. Des grandeurs interprétables mais pas tous comme des coefficients de dissipation

mais également approche Eulerian FEM (\textsc{papiers qui peuvent être citer}).
- De l'autre côté, on a des méthodes qui modélise le milieu granulaire comme un milieu continu. Le système est gouverné par les équations de conservation de mass et de quantité de mouvement.
- Présente (D. A. Santos, I. J. Petri, C. R. Duarte, and M. A. S. Barrozo, “Experimental and CFD study of the hydrodynamic behavior in a rotating drum,” Powder Technol. 250, 52–62 (2013).) comme une étude complète d'une approche Euler-Euler basée sur un milieu continu assumant des interpénétrations et une équation dérivée de la théorie cinétique.

- Chandra MPM 2021 Mets en avant le traitement sensible de la gestion des conditions limites dans le cas du broyeur à boulets et propose dans le cas MPM des méthodes pour les prendre en compte.

- Sur la loi de comportement: difficile du fait de la cohabitation de différents régimes. Modèle élastoplastique sont souvent utilisé pour le comportement solide. Pour le comportement fluid-like, utilise des loi viscoélastique ou réhologique. En particulier, la loi $\mu (I)$ cherche à modéliser ses deux régimes à partir d'un nombre adimentionnel, le nombre Inertiel $I$.
- il relie the shear strength with the normal stress à travers un coefficient de friction qui dépend de $I$.
- Présente $\mu(I)$.

- deux méthodes: discrètes et continues
- Les méthodes de discrétisation particulaire sont adaptées pour les problèmes de grandes transformations parce que représentation hybrid Euler-Lagrange du mouvement.
- Il donne des papiers où MPM a été utilisé pour simuler écoulements granulaires.
- On peut aussi coupler MPM et DEM pour interaction avec des solides comme les boulets du broyeur. (Liu C, Sun Q, Zhou GG. Coupling of material point method and discrete element method for granular flows impacting simulations. International Journal for Numerical Methods in Engineering. 2018;115:172-188.)
- Dans ce papier utilise MPM avec une loi de plasticité de Drucker–Prager pour modéliser le mélange dans le tambour en rotation.