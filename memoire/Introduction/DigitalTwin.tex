% !TEX root = memoire/main.tex


\subsection{Jumeau Numérique}

Riche à la fois de données simulées et d'outils de mesures sur le dispositif réel,

Le jumeau numérique est une réplique virtuelle d'un système physique, permettant de simuler, d'analyser et de prédire le comportement du système physique en temps réel. Ce modèle numérique intègre des données dynamiques et historiques, permettant une représentation précise et synchronisée dans le temps. Dans le contexte industriel, les jumeaux numériques utilisent l'intelligence artificielle (IA), l'analyse de données, et les capteurs physiques pour améliorer la compréhension des processus et faciliter la prise de décisions.

\subsection{Apport du jumeau numérique pour le broyeur à boulet}
Apport 1 : compréhension du procédé
Le jumeau numérique permet de comprendre les phénomènes se déroulant à l'intérieur du broyeur à boulets, sans avoir à l'ouvrir. Il permet ainsi d'extraire de l'information quant aux mécanismes se déroulant lors de la comminution.

Apport 2 : optimisation du processus de broyage
Le jumeau numérique du broyeur à boulets permet d'analyser et de simuler le processus de broyage en temps réel. Il peut prédire l'efficacité du broyage, le degré de mélange des matériaux, et les impacts des variables opérationnelles comme la vitesse de rotation et le taux de remplissage des boules. Cela aide à optimiser les paramètres de fonctionnement pour obtenir un mélange homogène et efficace. En particulier, un apprentissage par renforcement permet de construire un modèle capable d'optimiser les paramètres du procédé en temps réel.

Apport 3 : maintenance prédictive
En surveillant l'état du broyeur à boulets, le jumeau numérique peut prédire les besoins de maintenance avant que les défaillances ne surviennent. Cela réduit les temps d'arrêt imprévus, augmente la durée de vie de l'équipement et assure une production continue et fiable.

Apport 4 : contrôle de la qualité du produit
La précision du jumeau numérique dans la modélisation du processus de broyage aide à garantir que le combustible MOX répond aux normes de qualité attendues.

\subsection{Défis associés à la construction du jumeau numérique}
Défi 1 : données physiques
La première difficulté réside dans la collecte des données physiques. Le broyeur à boulets étant un système complexe, il nécessite un suivi détaillé des paramètres tels que la vitesse de rotation, des mesures acoustiques, ou d'un champ de vitesse. La collecte de ces données en temps réel et de manière fiable est cruciale pour assurer la représentativité du jumeau numérique.

Défi 2 : modélisation et simulation numérique
Le deuxième défi est la modélisation précise du broyeur à boulets. Ce processus nécessite une compréhension approfondie des mécanismes du broyage. La création de modèles numériques qui capturent fidèlement ces phénomènes est complexe et exige une expertise en mécanique. De plus, pour une utilisation en temps réel, ces modèles doivent être suffisamment efficaces pour permettre de suivre le procédé. Ce problème de rapidité de calcul peut néanmoins être évité par l'utilisation d'un métamodèle.

Défi 3 : assimilation de données (data assimilation, DA)
Le dernier défi concerne l'intégration des données réelles dans les modèles numériques. La DA implique l'ajustement des modèles basés sur les données physiques collectées, pour améliorer leur précision et leur fiabilité. Cela nécessite des algorithmes avancés capables de traiter de grandes quantités de données, souvent hétérogènes, tout en gérant les incertitudes et les erreurs inhérentes aux mesures. Le développement de ces algorithmes doit tenir compte des spécificités du processus de broyage dans le contexte du MOX, ce qui représente un véritable défi en matière de fusion de données et d'apprentissage automatique.

La construction d'un jumeau numérique pour un broyeur à boulets dans la fabrication du MOX est une activité multidisciplinaire, exigeant une expertise en modélisation et simulation, en expérimentation, et en sciences des données.