% !TEX root = ./memoire/main.tex


\section{Objectifs de la thèse}

Pour modéliser le mélange de poudre dans le broyeur à boulets intervenant la fabrication du combustible MOX, des simulations dites lagrangiennes ou particulaires ont été développées. D'autre part, à partir de méthodes de traitement d'images, il est également possible de mesurer des profils d'écoulement ou les champs de vitesse dans le broyeur à boulets. Pour construire un modèle de Jumeau Numérique capable d’intégrer les observations issues de l’observation, des méthodes d'assimilation de données doivent être utilisées pour mettre à jour l'état de la simulation à partir d'observations en tenant compte à la fois des incertitudes sur l’état et les observations. Toutefois, les méthodes d'assimilation sont définies sont pour des discrétisations (maillage, grille) qui ne varient pas au cours du temps. Or les méthodes particulaires/lagrangiennes sont définies à l’aide particules qui se déplacent en suivant l’écoulement du milieu.
C'est ce qui justifie cette thèse, elle consiste à développer des méthodes d'assimilation de données adaptées à des simulations lagrangiennes.

Tout d'abord, un état de l'art respectif sur les méthodes d'assimilation de données et les simulations lagrangiennes est tout d'abord réalisé. Le chapitre~\ref{sec:prob_contribution}, offre en particulier une analyse critique sur l'adaptabilité des méthodes d'assimilation aux différents types de méthodes lagrangiennes.
%Il montrera en particulier un certain nombre de limitations, motivant le développement des contributions de thèse.
Conscient des spécificités de ce type de simulation, le chapitre suivant présentera le développement d'adaptation du filtre de Kalman d'ensemble à des simulations particulaires. Pour cela, l'idée a été de définir une correction des intensités de la discrétisation. Cette correction a été évalué sur plusieurs applications.
Dans un deuxième temps, l'objectif du chapitre qui suit a été de développer des méthode d'assimilation de donnée par correction de position pour des simulations lagrangienne. En effet, Pour cela nous formulons le problème d'assimilation de données tenant compte d'une erreur d'alignement afin d'être capable de corriger la position des particules. L'idée a été d'introduire une transformation pour aligner les particules dans un cadre variationnel.
