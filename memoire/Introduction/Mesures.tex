% !TEX root = main.tex
\section{Données mesurées}

Si le procédée de tambour en rotation a pu être simulé via des méthodes numériques, la validation et la compréhension du procédé se voient renforcés par l'utilisation acru de méthodes de mesure durant la phase de fonctionnement. Issue de capteurs, nous les classerons en deux types. Tout d'abord l'instrumentation sans contact \textit{Off-Shell}, qui ne sont pas solidaire du tambour en rotation. Dans cette famille de capteurs, on trouve des mesures par imagerie dans le visible et l'infrarouge. Via l'utilisation d'une paroi tranparente, l'écoulement peut être filmé pour analyser le déplacemement du milieu granulaire et des corps broyants. De plus, les mesures par correlation d'image comme la méthode de \textit{Particle Image Velocimetry} (PIV) permettent, par l'analyse des variations de motifs ou de structures entre des images successives, d'extraire des informations sur les déplacements, et ainsi d'en déduire le champ de vitesse. Ainsi, dans \cite{jarray_wet_2019} des caméras ont permis d'extraire les profils de température granulaire et de vitesse d'écoulement pour un tambour en rotation. Egalement, dans \cite{Adepu}, une caméra thermique permet qu'en a elle de mesurer la température lors de l'écoulement.

Un autre type de mesure se base sur des analyse accoustiques. En particulier dans les travaux de \cite{Owusu} des mesures accoustiques ont été réalisées sur un broyeur à boulets pour prédire la broyabilité du minerai et ajuster les densités de pulpe, montrant que l'émission acoustique varie selon la dureté des matériaux broyés. Dans \cite{almond}, la charge d'un broyeur est corrélée pour suivre les mouvements et les différents régimes à l'intérieur du tambour.

Finalement, des mesures vibratoires peuvent être acquis instrumentant des corps broyants pour évaluer la charge solide à l'intérieur du broyeur à boulets, en capturant les signaux d'accélération à l'aide d'un accéléromètre triaxial intégré \cite{Wang}.

Une seconde classe d'instrumentation consiste à réaliser des mesures avec contact (\textit{On-Shell}) c'est à dire directement sur la paroi du tambour. Dans ce cadre, l'analyse de données vibratoires à partir d'accéléromètres permettent de mieux comprendre la position de la charge du broyeur, son niveau global ainsi que l'impact sur les revêtements \cite{Davey}. Des jauges de contraintes peuvent aussi être appliquées. Dans \cite{tano_2005} celle-ci permettre d'étudier la position du pied et de l'épaule de l'écoulement. Enfin, les données issues du moteur peuvent être utilisées tel que le couple moteur, la vitesse de rotation ou sa puissance. Ainsi, dans \cite{pedrayes_frequency_2017}, l'analyse dans le domaine fréquenciel du couple a été calibré pour prédire le taux de remplissage.

Une large gamme de mesures sont donc disponibles. La qualité des résultats analysée reste néanmoins très variable. En effet, l'interprétabilité de bon nombre ce fait au travers de méthode de corrélation et non pas via des approches inductives.

\paragraph{partie Loïc}
Deux types de données sont disponibles au SA3E pour alimenter ce modèle virtuel.

Mesures du champ de vitesse
Nous disposons de mesures du champ de vitesse capturées à travers la face avant d'un hublot transparent. Ces données offrent un aperçu direct du comportement dynamique des matériaux à l'intérieur du broyeur, permettant notamment d'observer le profil d'écoulement du mélange et le champ de vitesse du mélange.

Mesures acoustiques
L'analyse du son émis par le broyeur peut révéler des informations sur divers aspects du processus, tels que les conditions de fonctionnement, les anomalies potentielles, ou encore l'usure des éléments du broyeur.