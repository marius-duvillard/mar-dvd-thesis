% !TEX root = memoire/main.tex

\chapter{Introduction}

\section*{Contexte Général}
\subsection*{Contexte Industriel}
Dans le domaine de la production d'énergie electrique, l'énergie nucléaire est une source d'énergie qui s'est imposé à de nombreux pays industrialisés.

En 2023, l'industrie électronucléaire a représenté 65\% de la production totale d'électricité en France, avec un parc composé de 56 réacteurs à eau pressurisée (REP) répartis sur 18 centrales \cite{rte2023}. Dans le monde, la production nucléaire ne fait qu'augmenter. Fin 2022, la capacité totale des 438 réacteurs nucléaires de puissance en exploitation dans 32 pays s’établissait à 393,8 gigawatts électriques (GWe). Si aujourd'hui, le nucléaire représente près de 9,8\% de la production mondiale, elle pourrait attendre 14\% du bouquet électrique en 2025 \cite{aiea2023}. D'ici 2035, le nombre de pays qui exploitent des centrales nucléaires pourrait augmenter de quelque 30\% d'ici 2050.

Le secteur est également en constante mutation avec le développement de nouvelles technologies en particulier avec les technologies de 4\textsubscript{ème} génération mais également le nouveau paradigme des SMR (\textit{Small Modular Reactor}) où la modularité permet une chaîne de déploiment et de production souple et avec une financement moindre~\cite{academie2022}.

Si ce secteur atire, c'est en particulier car il offre un très bon rapport qualité prix et est faiblement émettrice en gaz à effet de serre. Son facteur d'émission, c'est à dire la quantité d'émissions de gaz à effet de serre par unité d'énergie produite, est estimé à 12 g$CO2$eq/kWh. Il serait même encore plus faible en France~\cite{schlomer_technology-specific_nodate}. Ainsi elle est aussi émetrice que l'éolien ou bien la production photovoltaïque et est 100 à 1000 fois moins émettrice que les centrale à énergie fossiels.

L'urgence climatique pousse à considérer l'énergie nucléaire comme un levier essentiel dans la transition énergétique, offrant une alternative viable aux énergies fossiles. Cependant, cette option soulève une série de préoccupations. Outre les inquiétudes liées à la sécurité des installations et au risque de prolifération nucléaire~\cite{npt_resolution}, il est crucial d'aborder la question des déchets hautement radioactifs générés tout au long du fonctionnement des réacteurs. Chaque année, la production d'électricité entraîne la création de près de 2 kg de déchets par habitant. Une infime proportion constitue les déchets à vie longue, mais ils représentent la majorité de l'activité radioactive (0.2\% des stocks pour 95\% de l'activité). En outre, il est impératif de préserver les réserves de combustible nucléaire. Dans cette optique, les avancées technologiques telles que les réacteurs de quatrième génération visent à optimiser l'utilisation des ressources en transmutant l'uranium 238 en plutonium 239.

Ainsi, la question du retraitement et de la fermeture du cycle nucléaire reste cruciale. C'est dans cette perspective que le combustible MOX (Mélange d’OXyde de plutonium et d’OXyde d’uranium) a été développé afin de recycler une partie des matières nucléaires issues du traitement des combustibles à Uranium Naturel Enrichi (UNE).
Le combustible nucléaire est généralement composé de dioxyde d'uranium (UO2), enrichi à 3-5\% en uranium 235. Une alternative intéressante est le combustible MOX (Mixed Oxide), qui combine du dioxyde de plutonium (PuO2) et du dioxyde d'uranium (UO2). Ce type de combustible est particulièrement utile pour les réacteurs à neutrons rapides, mais peut également être employé dans les réacteurs à eau pressurisée actuels. La teneur en PuO2 dans le MOX varie entre 8\% et 30\%, en fonction des besoins spécifiques du réacteur. Le PuO2 provient du recyclage dans les usines de retraitement, car le plutonium est un produit de fission de l'uranium 235. Ces deux oxydes diffèrent par leurs propriétés, notamment leur surface spécifique, avec 2 m²/g pour UO2 et 6 m²/g pour PuO2. En termes de morphologie, les particules d'UO2 forment des agglomérats, tandis que le PuO2 présente des plaquettes submicroniques.

%Fabrication du combustible
Tout comme le combustible à base d'uranium, le combustible est présent dans les réacteurs sous forme de pastille cylindrique de diamètre et de hauteur d'environ 1 cm. Elles sont ensuite empillées dans des gaines métalique et consititues un élément de crayon d'environ 4 m de haut. Ces éléments sont ensuite réuni dans un assemblage dans une grille de près de 250 éléments. Pour obtenir ces pastilles, la fabrications passe par différentes étapes de fabrication en particulier une phase de mélange et de broyage qui a lieu au sein d'un broyeur à boulets.

Ce dispositif cylindrique, rempli de boules de broyage, appelés corps broyants, met en oeuvre un processus de rotation pour broyer finement le mélange de poudres d'oxyde. C'est une étape qui dure entre 2 et 4h et qui va permettre de mélanger les deux poudres afin d'avoir un mélange homogène à une granulométrie inférieure à . En effet il est nécessaire d'avoir une poudre fine afin d'obtenir une bonne coulabilité de poudre pour permettre la mise en forme des pastilles lors de l'étape suivante de frittage. D'autre part, cette étape est déterminante pour éviter d'avoir des haute concentration d'aggloméra de Pu, à l'origine de points chauds facteur d'accident dans le réacteur \textcolor{red}{mettre une citation Bouloré ?}.

Cependant, son contrôle est déterminé de manière empirique par l'expérimentateur sur une variété de paramètres tel que la vitesse de rotation, le degré de remplissage, les proportions d'alimentation et de corps broyants.

En effet, cette étape critique même si simple dans son principe reste encore mal compris~\cite{Austin1981,Brandao2020,Mankosa1986,Datta2002,Capece2014}.
Le large spectre des physiques mise en jeu.
C'est dans cette perspective que des outils complémentaire d'analyse sont nécessaires à la fois à l'aide d'outils de simulation mais également de mesures expérimentales.

Ainsi, des études expérimentales ont été mises en place. En particulier, la thèse de Giraud pour pouvoir déterminer des relations empirique entre les propriétés microscopiques du combustible MOX et du comportement rhéologique macroscopique du mélange~\cite{giraud_analyse_2020}. Ces études reposent sur des campagnes de mesures complexes nécessitant de la manipulation de poudres irradiées. Or, les poudres mise en jeu sont irradiés, ce qui nécessite l'utilisation d'un environnements scellés pour éviter la contamination, des équipements spécialisés et exclusifs ainsi que des blindage contre les radiations et la production de chaleur. De plus, les mesures effectués sont soumises à des incertitudes.

Ainsi, ces études sont complétées par des méthodes de simulation haute fidélité pour représenter le comportement des poudres. Ces méthodes se basent en grande majorité sur des méthodes sans maillage afin de pouvoir soit pour pouvoir traiter chaque poudres indépendemment, soit pour pouvoir traiter des problèmes de grandes transformations en utilisant une représentation continue du milieu. C'est en particulier la méthode des éléments discrets (DEM) qui a été utilisée dans les thèses de Orozco~\cite{Orozco2019} et Vu~\cite{vu_quasi-static_2023} pour simuler l'écoulement et les mécanismes de fragmentation au sein du tambour. Ces études ont permis en particulier de déterminer des grandeurs adimentionnelles pour pouvoir réaliser des changements d'échelle.

Cependant l existe encore un écart substantiel entre les connaissances fondamentales actuelles sur les mécanismes physiques régissant le processus de broyage et le besoin actuel de sa modélisation prédictive et quantitative en vue de ses applications d'ingénierie.

Cependant, il existe encore un écart substantiel entre les connaissances fondamentales actuelles sur les mécanismes physiques régissant le processus de broyage et le besoin actuel de sa modélisation prédictive et quantitative en vue de ses applications d'ingénierie.

C'est dans cette perspective que
% Broyeur à boulets
