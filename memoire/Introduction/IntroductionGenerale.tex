\chapter{Introduction}

Dans le domaine de la production d'énergie electrique, l'énergie nucléaire est une source d'énergie qui s'est imposé à de nombreux pays industrialisés.

En 2023, l'industrie électronucléaire a représenté 65\% de la production totale d'électricité en France, avec un parc composé de 56 réacteurs à eau pressurisée (REP) répartis sur 18 centrales \cite{rte2023}. Dans le monde, la production nucléaire ne fait qu'augmenter. Fin 2022, la capacité totale des 438 réacteurs nucléaires de puissance en exploitation dans 32 pays s’établissait à 393,8 gigawatts électriques (GWe). Si aujourd'hui, le nucléaire représente près de 9,8\% de la production mondiale, elle pourrait attendre 14\% du bouquet électrique en 2025 \cite{aiea2023}. D'ici 2035, le nombre de pays qui exploitent des centrales nucléaires pourrait augmenter de quelque 30\% d'ici 2050.

Le secteur est également en constante mutation avec le développement de nouvelles technologies en particulier avec les technologies de 4\textsubscript{ème} génération mais également le nouveau paradigme des SMR (\textit{Small Modular Reactor}) où la modularité permet une chaîne de déploiment et de production souple et avec une financement moindre~\cite{academie2022}.

Si ce secteur atire, c'est en particulier car il offre un très bon rapport qualité prix et est faiblement émettrice en gaz à effet de serre. Son facteur d'émission, c'est à dire la quantité d'émissions de gaz à effet de serre par unité d'énergie produite, est estimé à 12 g$CO2$eq/kWh. Il serait même encore plus faible en France~\cite{schlomer_technology-specific_nodate}. Ainsi elle est aussi émetrice que l'éolien ou bien la production photovoltaïque et est 100 à 1000 fois moins émettrice que les centrale à énergie fossiels.

L'urgence climatique pousse à considérer l'énergie nucléaire comme un levier essentiel dans la transition énergétique, offrant une alternative viable aux énergies fossiles. Cependant, cette option soulève une série de préoccupations. Outre les inquiétudes liées à la sécurité des installations et au risque de prolifération nucléaire~\cite{npt_resolution}, il est crucial d'aborder la question des déchets hautement radioactifs générés tout au long du fonctionnement des réacteurs. Chaque année, la production d'électricité entraîne la création de près de 2 kg de déchets par habitant. Une infime proportion constitue les déchets à vie longue, mais ils représentent la majorité de l'activité radioactive (0.2\% des stocks pour 95\% de l'activité). En outre, il est impératif de préserver les réserves de combustible nucléaire. Dans cette optique, les avancées technologiques telles que les réacteurs de quatrième génération visent à optimiser l'utilisation des ressources en transmutant l'uranium 238 en plutonium 239. Ainsi, la question du retraitement et de la fermeture du cycle nucléaire reste cruciale. C'est dans cette perspective que le combustible MOX (Mélange d’OXyde de plutonium et d’OXyde d’uranium) a été développé afin de recycler une partie des matières nucléaires issues du traitement des combustibles à Uranium Naturel Enrichi (UNE).

La fabrication de ce combustible passe par différentes étapes de fabrication en particulier une phase de mélange et de broyage qui a lieu au sein d'un broyeur à boulets. Ce dispositif cylindrique, rempli de boules de broyage, met en oeuvre un processus de rotation pour broyer finement le mélange de poudres d'oxyde. Sa performance est critique pour la qualité du produit final et sa sûreté d'utilisation dans les réacteurs nucléaires. Cependant, son contrôle est déterminé de manière empirique par l'expérimentateur sur une variété de paramètres tel que la vitesse de rotation, le degré de remplissage, les proportions d'alimentation et de corps broyants. Le large spectre des physiques mise en jeu à l'intérieur du tambour mais aussi le contexte de manipulation de poudres irradié rendent le contrôle et l'optimisation très complexe. Ceci a pour conséquence d’augmenter le temps de broyage, la consommation du procédé et le nombre d’intervention nécessaire sur l’installation.

C'est dans ce contexte que des outils d'aide à la compréhension par la modélisation et la simulation des étapes de la fabrication du combustible sont développées. L'objectif étant de pouvoir faire le suivi de l'état du milieu granulaire sur une large gamme de paramètre ainsi que de comprendre les mécanismes intervenant dans ce processus.

En particulier, le CEA a développé des outils pour la simulation l'état du mélange dans le milieu granulaire à l'aide de méthodes sans maillage afin de représenter.

Le problème est que ces modèles reposent toujours sur des hypothèses simplificatrices limitant leurs champs d'action. De plus, les modèles dépendent de paramètres qu'il est nécessaire de calibrer. Enfin, l'incertitude de modélisation qui en découle a pour conséquence d'augmenter l'erreur de prédiction de modèle.

Par données expérimentales nous entendons toutes données qui peut être fournies par le dispositif expérimental.

C'est ce qui justifie cette thèse, elle consiste à développer des méthodes capables de combiner les données issue de la simulation et des données exéprimentales au sein du développement du Jumeau Numérique du procédé. Le jumeau numérique est une réplique virtuelle d'un système physique, permettant de simuler, d'analyser et de prédire le comportement du système physique en temps réel. Ce modèle numérique intègre des données dynamiques et historiques, permettant une représentation précise et **synchronisée** dans le temps. Dans le contexte industriel, les jumeaux numériques utilisent l'**intelligence artificielle** (IA), l'analyse de données, et les capteurs physiques pour améliorer la compréhension des processus et faciliter la prise de décisions.


C'est dans ce contexte de la fabrication du combustible MOX (pour Mélange d’OXyde de plutonium et d’OXyde d’uranium)


\section*{Objectifs de la thèse}
% Plan de la démarche comme évoqué dans la formation.