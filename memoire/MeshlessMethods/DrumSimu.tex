% !TEX root = main.tex

\chapter{Simulation du tambour et Assimilation de données}

Nous avons vu dans le chapitre précédent que l'assimilation dépendait de la définition d'un modèle et de son état pour définir le prior et la vraissemblance. Les simulations de l'écoulement dans le broyeur à boulets qui reposent sur des discrétisations sans maillage dites particulaires, définissent ainsi notre état et son modèle d'évolution. Cependant, le caractère Lagrangien de la définition de l'état implique d'évaluer jusqu'à quel point les méthodes classiques d'assimilation peuvent être adaptée. En effet celles-ci se base sur des états dont la discrétisation restent identique. De plus, le fait que l'espace soit continu ou discret va également modifier la signification de l'état et sa capacité à être mis à jour.
Dans cette partie, nous reprendrons les principales familles de formulation particulaire en développant les formulations des méthodes DEM, SPH et MPM.  Pour chacune d'entre elles, nous évoquerons les sigularités et limites à l'adaptation des méthodes d'assimilation de données. De plus, nous présenterons la Méthode Vortex comme une méthode particulaire modèle pour la suite manuscrit.


\section{Méthode des éléments discrets (DEM)}

La méthode consiste à considérer le mouvement d'un ensemble de $N$ grains composant le milieu. Celui-ci est décrit par l'équation de la dynamique qui peut s'écrire sous la forme

\begin{equation*}
    \left\{
    \begin{aligned}
         & m_{i} \frac{ d^{2}\vec{r}_i }{dt^2}=\vec{f}_{i},\; i=1\ldots N      \\
         & I_{i} \frac{d \vec{\omega}_{i}}{dt}=\vec{\Gamma}_{i},\; i=1\ldots N
    \end{aligned}
    \right.
\end{equation*}où $`N`$ est le nombre de grains, $`m_{i}`$ est la masse, $`I_i`$ est le moment d'inertie, $`\vec{r}_{i}`$ est la position, $`\vec{\omega_{i}}`$ est la rotation, $`\vec{f}_{i}`$ est la force exercée sur le grain considéré et $`\vec{\Gamma_{i}}`$ le moment associé à la force $`\vec{f}_{i}`$.

La force $`\vec{f}_{i}`$ peut être décomposée de la manière suivante

\begin{equation*}
    \vec{f}_{i}=\underset{{\scriptstyle j\neq i}}{\sum}\vec{f}^{c}_{ij}+\vec{f}_{ext}
\end{equation*}

où $`\underset{{\scriptstyle j\neq i}}{\sum}\vec{f}^{c}_{ij}`$ représente les forces de contact qui s'exercent à la surface de la particule $i$ et les forces externes $`\vec{f}_{ext}`$ sont celles appliquées au centre de la particule $`i`$ (par exemple la force de gravité).

Les forces de contact sont décrite par des loi de contact entre les grains ainsi qu'entre les grains et la paroi. Dans le cas élastique, la modèle de Hertz est adapté pour décrire l'intéraction normale. Le modèle Hertz-Mindlin permet de déterminer les interactions tangentielles élastiques. Ces loi sont fonctions de l'interpénétration inter-particule et de la vitesse relative interpaticules. Par exemple, dans le cas de particules sphérique l'interpénétration est défini comme $\delta_{ij} = \norm{\bm x_i - \bm x_j} - (R_i - R_j)$ où $x$  est la position et $R$ le rayon d'une particule.

La force de contact tend à pénaliser l'interpénétration via des coefficients de raideur $k$ et d'amortissement $\gamma$ qui dépendent des propriétés mécaniques des grains de de la paroi. Il existe plusieurs algorithme pour résoudre numériquement les équations de la dynamique. Le plus utilisé est l'algorithme de Verlet en vitesse.


La méthode des éléments discrets (DEM, pour Discrete Element Method) est une technique de simulation numérique utilisée pour étudier le comportement des systèmes de particules, tels que lors du mélange broyage à l'intérieur du broyeur à boulets. Cette approche est particulièrement pertinente pour modéliser les interactions complexes entre les particules dans ces systèmes, où la dynamique individuelle de chaque particule peut avoir un impact significatif sur le processus global.

Dans un mélangeur-broyeur, les particules interagissent entre elles, avec les parois du broyeur, et avec le corps broyant. La DEM modélise chaque particule individuellement, en tenant compte de ses propriétés physiques telles que la taille, la forme, la masse, la rigidité, et le modèle de fragmentation. Les interactions incluent les forces de contact, les forces de frottement, et les forces de cohésion.

Le processus de simulation DEM dans un mélangeur-broyeur commence par la définition des propriétés des particules et des conditions initiales du système. Le mouvement de chaque particule est ensuite calculé en résolvant les équations de Newton pour le mouvement et la rotation. Ces calculs tiennent compte des forces et des moments résultant des collisions et des interactions entre particules, ainsi que de l'interaction des particules avec les parois du broyeur.

L'un des principaux avantages de la DEM est sa capacité à fournir des informations détaillées sur le mélange et le broyage des particules à l'échelle microscopique. Elle permet d'analyser comment les variations dans la configuration des particules, la vitesse de rotation du broyeur, et d'autres paramètres opérationnels influencent l'efficacité du broyage et l'homogénéité du mélange.

Cependant, l'utilisation de la DEM pour la simulation de mélangeurs-broyeurs peut être exigeante en termes de ressources informatiques, en particulier pour les systèmes avec un grand nombre de particules.


\subsection{DEM et assimilation de données}

L'assimilation de données, lorsqu'appliquée à des systèmes simulés par la méthode des éléments discrets (DEM), se heurte à plusieurs limites importantes présentées ci-dessous.

\subsubsection{Limites de la DEM avec les méthodes variationnelles}

Dans le cadre des méthodes variationnelles d'assimilation de données, telles que 3D-Var, le principal défi est la grande dimensionnalité du problème d'optimisation.
En effet, La DEM simule le comportement de chaque particule individuellement. Cela signifie que l'état du système comprend les variables cinématique de chacune d'elles : position, la vitesse, accélération mais aussi la position angulaire, la vitesse angulaire et l'accélération angulaire ou la densité de chaque particule. Pour un système avec des milliers voire des millions de particules, cela conduit à un problème d'optimisation de très grande dimension.

De plus, il existe un nombre extrêmement élevé de contraintes, notamment l'interdiction de l'interpénétration des particules. Ces contraintes doivent être prises en compte pour assurer que la solution d'optimisation soit physiquement admissible.

L'application des méthodes 3D-Var et 4D-Var est donc trop exigeante d'un point de vue des temps de calcul.

De plus, les filtres variationnels corrigeant des erreurs d'intensité selon des normes euclidiennes, la variation selon les positions des observations ne sera pas linéaire.

Ainsi, la définition d'un état de particules discrètes implique de résoudre un problème d'optimisation non-linéaire de grande dimension.

\subsubsection{Limites de la DEM avec EnKF}
Pour l'EnKF, l'état estimé du système est une combinaison linéaire des états prédits par les différents membres de l'ensemble. Cependant, dans le contexte de la DEM, cette combinaison linéaire des états n'est pas nécessairement physiquement admissible. Par exemple, elle pourrait conduire à des situations où les particules s'interpénètrent ou violent d'autres lois physiques.
En d'autres terme, la mise à jour ne peut être réalisé que via le solveur lui-même capable de vérifier des contraintes de non interpénétrabilité mais ne peut être réalisé directement.

Un autre problème avec l'EnKF dans le contexte de la DEM est la difficulté de faire correspondre les particules entre les différents membres de l'ensemble. Même si initialement chaque membre possèdait la même configuration particulaire mais avec des propriétés différentes, chaque particule a sa propre trajectoire unique en cohérence avec les intéractions dans son voisinage. Aligner ces trajectoires à travers les différents membres de l'ensemble pour une assimilation de données cohérente est un défi complexe. Cette difficulté est exacerbée par le nombre élevé de particules et par la nature dynamique et chaotique de leurs interactions.

Dans le cas pus général où les membres n'aurait pas le membre de particule la correspondance des états est d'autant plus complexe.

Ainsi, le caractère discret de la méthode DEM rend l'assimilation par EnKF à la fois complexe par :

\begin{itemize}
    \item une définition de l'état et de ses statistiques non univoque ;
    \item la construction du gain de Kalman d'Ensemble ;
    \item la contrainte d'interpénétration lors de la mise à jour.
\end{itemize}

\section{Méthodes particulaires continues}
\subsection{Méthodes basées sur les particules}~\label{Background_Part}
Nous envisageons des méthodes par particules pour résoudre des problèmes continus en mécanique des fluides ou des solides. Cela inclut des méthodes telles que l'hydrodynamique par particules lissées (SPH) \cite{lucy_1977,gingold_monaghan_sph_1977} et la méthode des vortex (VM) \cite{cottet_vortex_2000}, et s'étend à d'autres méthodes comme la méthode des points de matériau (MPM) \cite{sulsky_particle_1994}. Elles partagent toutes de décomposer cette fois le domaine en un ensemble $\mathcal{P}$ de particules qui suivent la dynamique du problème. Ainsi, la discrétisation suit la transformation appliquée au milieu en transportant des quantités attachées à chaque particule. Elles sont en cela des méthodes Lagrangienne.

\subsubsection{Discrétisation par particules}

La solution représentée par le jeu de particule est obtenue grâce à deux éléments: une appproximation grâce à un noyau de lissage et l'approximation particulaire d'un opérateur intégral. Si la représentation de la solution peut évoluer suivant la méthode (en particulier avec la méthode MPM), la solution peut toujours être exprimé de cette manière suivant le choix du noyau.

Tout champ relativement régulier $\bm{u}$ sur $\Omega$ peut être écrit grâce à la propriété de filtrage de Dirac

\begin{equation*}
    \bm{u}(\bm{z}) = \int_{\Omega} \bm{u}(\bm{z'}) \delta(\bm{z'} - \bm{z})  d\bm{z'},
\end{equation*}avec $\delta$ la distribution de Dirac.

Une fonction de noyau $\phi_\varepsilon$ est introduite pour obtenir une estimation moyenne $\langle \bm{u} \rangle$ de $\bm{u}$ telle que

\begin{equation*}
    \langle \bm{u}(\bm{z}) \rangle = \int_{\Omega} \bm{u}(\bm{z'}) \phi_\varepsilon(\bm{z}-\bm{z'}) d\bm{z},
\end{equation*}où $\varepsilon$ est la longueur de lissage. Le noyau lisse doit au moins respecter les propriétés suivantes

\begin{eqnarray*}
    && \int_{\Omega} \phi_\varepsilon(\bm{z}) d\bm{z} = 1,      \\
    && \phi_\varepsilon(\bm{z}) \to \delta(z), \quad \varepsilon \to 0, \\
    && \phi_\varepsilon(\bm{z}) \in C^k,  \quad k \geq 1,
\end{eqnarray*} où les deux premières propriétés sont des propriétés résiduelles de la distribution de Dirac et la dernière est une exigence de différentiabilité nécessaire pour approcher les opérateurs différentielles.

La fonction moyenne $\langle \bm{u} \rangle$ est ensuite utilisée pour approximer la fonction d'origine.

Dans un second temps, le domaine d'origine $\Omega$ est subdivisé avec $N_p$ sous-domaines $\Omega_p$ associés à une particule Lagrangienne à l'emplacement $z_p \in \Omega_p$. Nous notons $V_p$ le volume de $\Omega_p$. Cette discrétisation est ensuite utilisée pour approximer la fonction moyenne de telle sorte que

\begin{eqnarray*}~\label{eq:part_approx}
    \langle \bm{u}(\bm{z}) \rangle &=& \sum_p \int_{\Omega_p} \bm{u}(\bm{z'}) \phi_\varepsilon(\bm{z}-\bm{z'}) d\bm{z'} \\
    &\approx& \sum_p \bm{u}(\bm{z}_p) V_p \phi_\varepsilon (\bm{z}-\bm{z}_p) \\
    &\approx& \sum_p \bm{U}_p \phi_\varepsilon (\bm{z}-\bm{z}_p).
\end{eqnarray*}

Ainsi, toute fonction définie sur une discrétisation par particules est définie par un ensemble de positions de particules $\bm{z}_p$ associées à une valeur de particule $\bm{U}_p = \bm{u}(z_p) V_p$ et un noyau lisse $\phi_\varepsilon$.

Sur la base de cette discrétisation, l'opérateur différentiel peut être dérivé à travers cette formulation.

Tout comme le champ $\bm u$, la même interpolation peut être appliquée pour obtenir

\begin{equation*}
    \nabla \bm{u}(\bm{z}) = \sum_p \bm{U}_p \nabla \phi_\varepsilon (\bm{z}-\bm{z}_p).
\end{equation*}

Ils existent toutefois une grande variété de formule pour approximer l'opérateur gradient. Dans ces cas, ce sont les propriétés de conservation associées au champ qui sont privilégiées.
Généralement, un terme additionnel est introduit lorsque le champ est évalué à la position d'une particule $q$ tel que

\begin{equation*}
    \nabla \bm{u}(\bm{z}_p) = \sum_q (\bm{U}_q - \bm{U}_p) \nabla \phi_\varepsilon (\bm{z_q}-\bm{z}_p)
\end{equation*}, où $\sum_q \bm{U}_p \nabla \phi_\varepsilon (\bm{z_q}-\bm{z}_p)$ est nul par propriété de localisation du noyau $\phi_\varepsilon$.

% Finalement dire qu'il y a donc des équations qui permettent de définir l'évolution des intensités ainsi que des positions.

\subsubsection{Exemple de fonctions de noyau}

Plusieurs noyaux ont été utilisés en fonction de la méthode. La formulation originale de la MPM n'utilisait pas de noyau de substitution et écrivait la densité comme suit

\begin{equation*}
    \bm{u}(\bm{z}) = \sum_p \bm{U}_p \phi_\varepsilon (\bm{z}-\bm{z}_p)
\end{equation*}

Et la résolution est basée sur une projection sur une grille de fond associée à certaines fonctions de forme \cite{sulsky_particle_1994}.

La méthode GIMP est une formulation différente qui utilise la fonction de Heaviside \cite{bardenhagen_generalized_2004} et associe donc un volume autour de chaque particule

\begin{equation*}
    M_1(r) = \frac{\alpha}{\varepsilon}\left\{\begin{aligned}
         & 1; & r \leq \varepsilon \\
         & 0; & \text{sinon}
    \end{aligned}
    \right.
\end{equation*}où $r = \norm{\bm{z}}_2$.

Cette méthode a été introduite pour éviter le problème de passage de cellule lorsque une particule se déplace d'une cellule à une autre à travers la grille de fond.

Dans la méthode SPH, comme son nom l'indique, un noyau lisse est associé pour approximer la solution. Théoriquement, il pourrait s'agir de la fonction de noyau gaussien

\begin{equation*}
    \phi_g(r) = \frac{1}{{(\pi \varepsilon^2)}^{d/2}} \exp(-r^2/\varepsilon^2)
\end{equation*}.

Ce noyau est infiniment différentiable mais défini sur un support non compact. En pratique, nous utilisons une coupure pour supprimer les valeurs négligeables pour une grande distance par rapport à une particule.

D'autres noyaux, basés sur des fonctions B-Spline pour travailler sur un support compact. Ces fonctions sont également positives, ce qui est une exigence pour certains champs comme la densité.

Par exemple, le B-spline quadratique, que nous appelons $M_3$, est défini avec
\begin{equation}~\label{quadratic_kernel}
    M_3(r) = \frac{\alpha}{\varepsilon^d}\left\{ \begin{aligned}
         & \frac{3}{4} - |q|^2                            & 0 \leq           & |q| < \frac{1}{2} \\
         & \frac{1}{2} {\left(\frac{3}{2} - |q|\right)}^2 & \frac{1}{2} \leq & |q| < \frac{3}{2} \\
         & 0                                              & \frac{3}{2} \leq & |q|
    \end{aligned}
    \right.
\end{equation}avec $r = \norm{z}_2 $ et $q = r / \varepsilon$ et $\alpha$ la condition de normalisation et $d$ la dimension spatiale.

Ce noyau garantit la continuité $C^1$.
Le noyau cubique est un autre noyau B-Spline qui est
\begin{eqnarray}~\label{cubic_kernel}
    M_4(r) &=&  \frac{\alpha}{\varepsilon^d} \left\{ \begin{aligned}
         & \frac{1}{6}{(-|q|+2)}^3 - \frac{4}{6}{(-|q|+1)}^3 & 0 \leq      & |q| \leq  1 & \\
         & \frac{1}{6}{(- |q|+2)}^3                          & 1      \leq & |q| \leq 2  & \\
         & 0                                                 & 2 \leq      & |q|
    \end{aligned}
    \right.
\end{eqnarray}

Dans ce dernier cas, le facteur de normalisation $\alpha$ est

\begin{equation*}
    \alpha = \left\{ \begin{aligned}
         & 1;    \quad      & 1\text{ d} \\
         & 30/14 \pi; \quad & 2\text{ d} \\
         & 3/ 2\pi; \quad   & 3\text{ d}
    \end{aligned}
    \right.
\end{equation*}

Notez que ces noyaux ont été définis avec la coordonnée radiale $r$. Une autre possibilité serait de définir le noyau multidimensionnel comme le produit tensoriel du noyau 1D

\subsection{SPH}

La méthode SPH a étét développée indépendemment par Lucy~\cite{lucy_1977}, et Gingold et Monhagan~\cite{gingold_monaghan_sph_1977}. Elle a été formulé initialement pour dess problème de formation et d'évolution des systèmes stellaires. Tout comme en mécanique quantique sont but est de réprésenter le système discret en le lissant pour obtenir un milieu continu discrétisé par un ensemble de particules. En mécanique, cette méthode est vu comme une méthode de discrétisation sans maillage d'un milieu continu.
Elle consiste à résoudre la forme forte des équations de la dynamique en approchant les champs et les opérateurs différentiels à l'aide de l'approximation particulaire et l'approximation par noyau précédemment évoqué.

Les équations résoluent sont l'équation de continuité et l'équation de conservation de la quantité de mouvement à travers l'équation d'Euler

\begin{eqnarray*}
    \frac{d\rho}{dt} + \rho \nabla \cdot \bm{v} = 0, \\
    \frac{d\bm v}{dt} = \frac1\rho \nabla \cdot \bm \sigma,
\end{eqnarray*}où $\rho$ est la densité, $\bm v$ la vitesse, $\bm \sigma$ la contraite de Cauchy.

En utilisant la règle d'approximation du gradient le terme $\rho \nabla \cdot \bm{v}$ peut être approximé pour chaque particule, ce qui donne pour l'équation de continuité

\begin{equation*}
    \frac{d\rho_p}{\sum_{q} m_j (\bm v_j - \bm v_i) \cdot \nabla \phi_\varepsilon(\bm z_p - \bm z_q)}.
\end{equation*}

De la même manière, l'équation d'équilibre des quantités de mouvement peut êter discrétisé. La forme suivante est généralement utilisée

\begin{equation*}
    m_p \frac{d \bm v}{dt} = \sum_{q} m_p m_q \left(\frac{\bm \sigma_p}{\rho_p^2} + \frac{\bm \sigma_q}{\rho_q^2} \right) \cdot \nabla \phi_\varepsilon(\bm z_p - \bm z_q).
\end{equation*}

Cette version est symétrique par rapport aux indices $p$ et $q$ ce qui favorise les propriétés de conservation.

Finalement, les équations de la dynamique sont intégrées dans le temps généralement à l'aide d'un algorithme dit \textit{leap-frog}.

\section{Méthode des points matériaux (material point method, MPM)}
\subsection{Material Point Method (MPM)}

La méthode MPM est une version de FLIP pour résoudre le problème de mécanique des solides.
Elle consiste, comme en élément finis, à résoudre le problème aux valeurs sous sa forme faible, en utilisant conjointement deux discrétisations, une grille et des particules. En cela elle est une méthode qui utilise à la fois une description lagrangienne et eulérienne.

Le problème aux conditions limites, sous sa forme forte, se composent des équations d'équilibre, des lois matériaux, de l'équation cinématiqu et des conditions limites et initiales donnant

\begin{equation*}
    \begin{cases}
        \begin{aligned}
             & \frac{D \rho}{Dt} + \rho \nabla \cdot \bm v  =  0                          ,                                                                         & \quad \text{(conservation de la masse)}                  \\
             & \rho \frac{D \bm v}{Dt}                      =  \nabla \cdot \bm \sigma + \rho \bm b,                                                                & \quad  \text{(conservation de la quantité de mouvement)} \\
             & \bm \sigma = LdC(\bm F),                                                                                                                             & \quad  \text{(loi de comportement)}                      \\
             & \bm u(\bm z, t) = \bar{\bm u}, \quad \forall \bm z \in \Gamma_u,    \quad  \bm \sigma (\bm z) \cdot \bm n = \bm t, \quad \forall \bm z \in \Gamma_t, & \quad  \text{(conditions limites)}                       \\&\bm v(\bm z, t = 0), \quad \bm \sigma(\bm z, t= 0) = \bm \sigma_0. & \quad  \text{(conditions initiales)} \\
        \end{aligned}
    \end{cases}
\end{equation*}

La forme faible de l'équation de conservation du moment est, en introduisant une fonction test $\bm q$

\begin{equation*}
    \int_\Omega \rho \bm a \cdot \bm q d\bm z + \int_\Omega \bm \sigma : \nabla \cdot \bm q d\bm z = \int_\Omega \rho \bm b d\bm z + \int_{\partial \Omega_t} \bm t dS.
\end{equation*}

Le schéma MPM peut être obtenu en utilisant une discrétisation particulaire de la densité sur les particules et en utilisant un  $\Omega = \bigcup_p \Omega_p$.

En concentrant la masse sur la position de chaque particule, on peut représenter la densité comme un somme de dirac tel que

\begin{equation*}
    \rho(\bm z) = \sum_p V_p~\rho_p \delta(\bm z - \bm z_p) = \sum_p m_p \delta(\bm z - \bm z_p)
\end{equation*}

De même, on discrétise la contrainte $\bm \sigma$

\begin{equation*}
    \bm \sigma(x) = \sum_p \bm \sigma_p \delta(\bm z - \bm z_p).
\end{equation*}
Une généralisation a été proposée avec la méthode GIMP \cite{bardenhagen_generalized_2004} en définissant une représentation particulaire de la densité en attroduisant une fonction caractéristique $\chi_p$ tel que

\begin{equation*}
    \rho(\bm z) = \sum_p m_p \chi(\bm z - \bm z_p)
\end{equation*}

Ce qui donne en choisissant $i$ comme indice de dimension d'espace

\begin{equation*}
    \sum_p m_p q_i(\bm z_p) a_i(\bm z_p) + \sum_p \sigma_{ij} \partial_j u_i + \int_{\partial \Omega_t} t_i q_i dS.
\end{equation*}


\subsubsection{Interprétation 1}

\subsubsection{Interprétation 2}

\subsubsection{Schéma classique}

La méthode MPM est implémentée généralement en trois phases. Dans un premier temps, les quantités définies sur les particules sont transférés sur les noeuds de la grille \textit{p2g}. Le principe fondementale de la dynamique est alors résolu permettant de déterminer une grille déformée. Finalement, les nouvelles quantités nodales permettent de mettre à jour les quantités particulaires dans une phase de transfert grille à particule (\textit{g2p}).

\paragraph*{p2g}

La grille de positions de noeuds $x_I$ est initialisée avec des valeurs nulles.

La masse $m_p$,la quantité de mouvement $m_p \bm v_p$ et les forces $\bm f_p$ sont transférées à la grille à l'aide des fonctions de forme $\phi_I$ associé à chaque noeuds

\begin{eqnarray*}
    m_I = \sum_p \varphi_{Ip}~ m_p, \\
    m_I \bm v_I  =  \sum_p \varphi_{Ip}~ m_p \bm v_p, \\
    \bm f_I  =  \sum_p \varphi_{Ip}~  \bm f_p. \\
\end{eqnarray*}

Des transferts plus complexes capable de préserver les moments angulaires ont été développé comme APIC~\cite{jiang_affine_2015}, Poly-PIC~\cite{fu_polynomial_2017}, et MLS-MPM~\cite{hu_moving_2018}.

\paragraph*{Mise à jour sur la grille}
La grille à chaque étape est initialisée dans un état non déformée. A l'aide du principe fondamentale de la dynamique, la vitesse sur la grille est mise à jour de manière explicite tel que

\begin{eqnarray*}
    m_I \bm a_I &=& \bm f_I + \bm f_g, \\
    m_I \bm v^{n+1} &=& \bm v^{n} + \Delta t~ (\bm f_I + \bm f_g) / m_I, \\
    \bm x_I^{n+1} &=& \bm x_I^{n} + \Delta t~\bm v^{n+1}.
\end{eqnarray*}

C'est durant cette étape que les conditions limites ou les collisions avec un objet peuvent être prise en compte.

\paragraph{g2p}

Les particules vont suivre la déformation de la grille. Cela aura deux conséquence : La mise à jour de la matrice de déformation $\bm F_p$ et de leurs positions $\bm x_p$ et leur vitesses $\bm v_p$.

La mise à jour de $\bm F_p$ est réalisé avec la déformée de la grille $ x_I^{n+1}$ de manière implicite en utilisant $\bm v^{n+1}$ de telle sorte que

\begin{equation*}
    \bm F_p^{n+1} = \left( \bm I + \Delta t \sum_I \bm v_I^{n+1} (\nabla \varphi_{Ip}^T)\right) \bm  F_p^{n}.
\end{equation*}

En ce qui concerne l'étape d'advection des particules, le schéma PIC suggérait l'interpolation des vitesses tel que

\begin{equation*}
    \bm v_{PIC}^{n+1} = \sum_I \varphi_{Ip} \bm v_I^{n+1}
\end{equation*}.

Si ce schéma est stable, il est toutefois dissipatif. Inversement la mise à jour FLIP propose de mettre à jour la vitesse $\bm v_{PIC}^{n}$ en interpolant l'accélération tel que

\begin{equation*}
    \bm v_{FLIP}^{n+1} = \bm v_{p}^{n} \sum_I \varphi_{Ip} (\bm v_I^{n+1} - \bm v_I^{n})
\end{equation*}.

Dans ce cas, le transfert est concervatif mais instable. Ainsi, il est recommandé d'utiliser pour mettre à jour la vitesse $\bm v^{n+1}$ une combinaison linéaire des deux formulations tel que

\begin{equation*}
    \bm v_{p}^{n+1} = \alpha \left(\bm v_{p}^{n} \sum_I \varphi_{Ip} (\bm v_I^{n+1} - \bm v_I^{n})\right) + (1- \alpha)\sum_I \varphi_{Ip} \bm v_I^{n+1}
\end{equation*}~avec $\alpha \in [0, 1]$.

Les schémas de type APIC,PolyPIC ou MLS-MPM, utilisant de plus un transfert du gradient de $v$, utilise une mise à jour PIC tout en restant conservatif.

La position est elle mise jour en interpolant la déformation de la grille de telle sorte que

\begin{equation*}
    \bm x_p^{n+1} = \bm x_p^{n} + \sum_I \varphi_{Ip}~\bm v^{n+1}
\end{equation*}

Finalement, la grille de calcule peut être effacée et réinitialisée.

La force interne de la particule $\bm f_p$ dépend de la loi de comportement qui lui est associée.

Elle dépends généralement de la contrainte $\bm \sigma_p$ qui peut être mise à jour au début ou à la fin du schéma donnant deux formulations différence USF (\textit{Update Stress First}) et USL (\textit{Update Stress Last}).
La méthode des points matériaux (MPM) est une technique de simulation numérique innovante, particulièrement adaptée à la modélisation de phénomènes complexes comme ceux rencontrés dans les mélangeurs-broyeurs. Cette méthode représente un compromis entre les approches par éléments finis et par particules, offrant ainsi une modélisation efficace des interactions matérielles dans des environnements dynamiques et déformables.

Dans la MPM, l'intérieur du tambour d'un mélangeur-broyeur est conceptualisé comme un milieu continu, adoptant une perspective macroscopique. La réponse du milieu est alors représenté par une loi de comportement mécanique telle que la loi de Drucker-Prager. Cela contraste avec la méthode des éléments discrets (DEM), qui se concentrent sur les interactions particule-par-particule.

Le processus de simulation avec la MPM implique des itérations entre une grille de calculs et des particules matérielles. Chaque particule porte des informations essentielles associées au matériau, telles que la masse, le volume, et les propriétés mécaniques (variables internes, gradient de déformation...). Ces particules sont utilisées pour transférer des informations sur et hors d'une grille de calculs, où les équations de mouvement et de comportement du matériau sont résolues.

Cette approche hybride permet à la MPM de capturer efficacement les déformations importantes, les ruptures, et d'autres comportements complexes du milieu qui sont fréquents dans les opérations de mélange et de broyage. En revanche, la description fine du phénomène proposée par la DEM n'est plus disponible.

En termes de temps de calculs, la MPM est plus efficace que la DEM.

\subsection{La MPM et la DA}
La MPM offre un cadre exploitable pour mettre en place une méthode de DA.
La structure de grille sous-jacente à la MPM permet une modélisation l'utilisation des méthodes variationnelles ou des méthodes d'ensemble. La structure de particules est aussi plus flexible dans le sens où elles représentent une densité de matière : elles peuvent donc s'interpénétrer.

\subsubsection{MPM et méthodes variationnelles}
Le principal défi est de gérer la dimension du problème d'optimisation pour la 3D-Var, ainsi que construire un modèle adjoint pour la 4D-Var.
Deux pistes sont envisageables : mettre à jour les champs nodaux et les champs particulaires. Comparativement à la DEM, le nombre de variables et le nombre de contraintes sont drastiquement réduits au prix d'une représentation plus grossière.

\subsubsection{MPM et EnKF}
Pour l'EnKF, l'état estimé est une combinaison linéaire des états prédits. Dans le contexte de la MPM, cela signifie que la mise à jour de l'état peut être directement effectuée sur la grille de calcul, plutôt que sur les particules individuelles. Cette approche réduit la complexité des calculs et facilite l'assimilation de données dans des systèmes à grande échelle.

Cependant, l'intégration de la MPM avec l'EnKF soulève plusieurs questions importantes :

1. **Transfert d'Informations de Particules à la Grille** : La première question concerne le transfert efficace des informations des particules vers la grille. Cela nécessite des algorithmes précis pour garantir que les informations pertinentes sur les propriétés des matériaux, telles que la densité, la contrainte, et le gradient de déformation déformation, sont correctement représentés sur la grille de calcul.

2. **Remaillage de Particules pour Représenter l'État Mécanique** : Une autre question clé est de savoir comment effectuer un remaillage des particules pour représenter fidèlement l'état mécanique du système après assimilation. Cela est crucial pour maintenir la cohérence et l'exactitude du modèle MPM, en particulier après des mises à jour successives de l'état du système.

Au travers des précédentes méthodes particulaires, nous constatons que l'application des méthodes d'assimilation sont innégalement applicable. En particulier, les méthodes discrètes n'offre pas la possibilité de corriger directement l'état de la discrétisation, mais nécessite une correction au travers du schémas d'intégration. D'autre part, les méthodes particulaires continues (par exemple MPM, SPH), permettent de modifier et faire varier dans son intégralité la discrétisation particulaire. En effet, chaque particule est définie en un point, ce qui annule tout problème d'interpénétration. Toutefois, il reste nécessaire de définir les
Afin de facilité le développement de nouveaux filtres, une autre méthode particulaire a été utilisée la méthode Vortex. Elle a été choisie car elle offre une modélisation plus simple que les méthodes SPH et MPM. En effet, chaque particule ne transporte qu'une quantité scalaire.Toutefois, elle dispose de toutes les caractéristiques d'une méthode particulaire continue. On retrouve de plus différentes version de cette méthode. Ainsi, la formulation classique de la méthode se rapproche de la méthode SPH et la méthode Vortex-In-Cell de la méthode MPM en mécanique des solides.

\section{Méthode Vortex}

La méthode Vortex (VM) est une méthode particulaire utilisé pour résoudre  dans le cas d'écoulements incompressibles~\cite{Cottet_Koumoutsakos_2000}. Elle a été pour la première fois développé indépendemment par Prager~\cite{prager1928druckverteilung} et Rosenhead~\cite{rosenhead1931formation}. Elle se base sur la discrétisation du champ de vorticité par un ensemble de particules, et résoud la formulation vorticité-vitesse des équations de Navier-Stokes

\begin{eqnarray*}
    \frac{\partial \bm \omega}{\partial t} + (\bm{u} \cdot \nabla) \bm \omega & = &(\bm \omega \cdot \nabla) \bm u + \nu \Delta \bm \omega, \\
    \Delta u  & =&  -\nabla \times \bm \omega,
\end{eqnarray*}où $\bm{u}$ la vitesse, $\omega= \nabla \times \bm u$ le champ de tourbillon, et $\nu$ pour la viscosité.

Les formes lagrangiennes des équations précédentes deviennent

\begin{eqnarray*}
    \frac{d \bx_p}{dt} = \bm u(\bx_p, t) \\
    \frac{d\bm \omega}{dt} = - [\nabla \times \bm u (\bx_p, t)]\bm \omega_p + \nu \Delta \bm \omega(\bx_p, t)
\end{eqnarray*}

Dans le cas d'un écoulement bi-dimentionnelle, champ tourbillon est définie comme un champ scalaire porté par la troisième dimension. En particulier dans un repère cartésien $\omega = \frac{\partial v_y}{\partial x} - \frac{\partial v_x}{\partial y}$. De plus le terme d'étirement disparaît $(\bm \omega \cdot \nabla) \bm u$, ainsi les équations lagrangiennes deviennent

\begin{eqnarray*}
    \frac{d \bx_p}{dt} = \bm u(\bx_p, t) \\
    \frac{d\omega}{dt} = \nu \Delta \omega(\bx_p, t)
\end{eqnarray*}

Le champ de vorticité est discrétisé à l'aide d'un ensemble de particules $p$ défini à une position $\bm z_p$, une quantité de circulation locale $\Gamma_p$ qui est par définition la circulation autour de la particule : $\Gamma_p = \oint_{\partial \Omega_p} \bm v = \int_{\Omega_p} \omega dS$. Ainsi, pour tout point $z \in \Omega \subset \mathbb R^2$ la vorticité peut être exprimée comme

\begin{equation*}
    \omega(\bm z, t) = \sum_{i=1}^{N_p} \Gamma_p(t) \phi_\varepsilon(\bm z - \bm z_p(t)),
\end{equation*}où $\phi_\varepsilon$ est le noyau associé à la particule avec une distance de lissage $\varepsilon$.

La vitesse $\bm u$ peut être obtenue en résolvant l'équation de Poisson suivante

\begin{equation*}~\label{eq:poisson}
    \lambda \bm u = - \nabla \times \omega.
\end{equation*}

Finalement, par une représentation intégrale et en choisissant $\phi= \delta$, on obtient dans le cas 2D l'équation de Biot-Savart suivante

\begin{equation*}
    \bm u(\bm z) = \sum_p \frac{\Gamma_p}{2\pi} \frac{(\bm z - \bm z_p)\times \bm k}{\|\bm z - \bm z_p\|^2},
\end{equation*}où $\bm k$ est le vecteur unitaire normal au plan.

En pratique, le choix d'un noyau en Dirac rend ainsi impossible le calcul de la vitesse sur la discrétization particulaire à cause du dénominateur en ${\|\bm z - \bm z'\|^2}$. En choisissant un noyau de type gaussien de taille $\varepsilon$ on obtient alors

\begin{equation*}
    \bm u(\bm z) = \sum_p \frac{\Gamma_p(1 - \exp(-r^2 / \varepsilon^2)) }{2\pi r^2} (\bm z - \bm z')\times \bm k, \quad r = \|\bm z - \bm z'\|.
\end{equation*}

L'idée d'utiliser un noyau de lissage est en cela assez proche de ce qui est fait avec la méthode SPH.

Afin de tenir compte de la diffusion, une approche par fractionnement est généralement utilisé. Introduite pour la première fois par Chorin~\cite{chorin_discretization_1973}, elle permet dans le cas de problème où le terme de transport est dominant, de traiter séparemment et successivement les termes d'advection et de diffusion.

Ainsi après avoir mis à jour la position des particules sans tenir compte de la viscosité, l'équation suivante est résolue

\begin{equation*}
    \frac{d\bm omega_p}{dt} = \nu \Lambda \omega(\bx_p).
\end{equation*}

Pour se faire, deux méthodes sont principalement utilisées : soit la méthode par marche aléatoire introduite dans~\cite{chorin_discretization_1973} ou par échange d'intensité introduite par~\cite{1989MaCom..53..485D}.
Dans le premier cas, la position de chaque particule est perturbée avec un vecteur de variables indépendantes tirées selon une distribution gaussienne de moyenne zero et d'écart-type $2\nu \Delta t$. Dans le second cas, l'opérateur différentiel est approximé à l'aide de la discrétisation comme il est fait dans la méthode SPH. Dans ce dernier cas l'intensité évolue comme

\begin{equation*}
    \frac{d \omega_p}{dt} = \nu \varepsilon^{-2} \sum_q V_q [\omega_q - \omega_p] \phi_\varepsilon(\bx_p - \bx_q).
\end{equation*}

\subsection{Vortex-In-Cell, VIC}

La méthode Vortex-In-Cell~\cite{christiansen_1973} est une version Particle-In-Cell~\cite{birdsall_1969} de la méthode Vortex précédemment décrite. Elle a été développé pour tenir compte de ses faiblesses. Tout comme la méthode MPM, celle-ci tient bénéfice de la représentation particulaire des particules pour tenir compte du terme d'advection mais également d'une grille de calcul pour résoudre l'équation de Poisson ou le terme de diffusion en utilisant des méthodes eulérienne.

Le schéma de transfert est similaire d'avec celui de la méthode MPM. D'abord une projection du champ de vorticité sur la grille (p2g) pour obtenir les valeurs nodales $\omega_i$. L'équation de Poisson~\ref{eq:poisson} est résolue sur la grille, soit par différences finies, soit par une méthode FFT pour obtenir des vitesses au noeud $\bm u_i$. La vitesse est ensuite interpolée sur le particule (g2p) pour mettre à jour leur position (advection).

Avec de la diffusion, l'équation de diffusion peut être ensuite résolue sur la grille et interpolé pour mettre à jour les quantités particulaire.

\subsection{Similarité avec les méthodes SPH et MPM}
Cette approche diffère de la MPM dans le sens où les particules portent des informations complexes sur les propriétés mécaniques du matériau.

Le processus de simulation avec la VIC implique d'abord le calcul des champs de vitesse sur la grille. Ces champs sont ensuite utilisés pour déplacer les particules dans le fluide, qui à leur tour transportent la vorticité à travers le domaine de simulation. L'avantage de cette méthode est sa capacité à modéliser avec précision les phénomènes complexes d'écoulement de fluides, tels que la formation et l'évolution de tourbillons, tout en maintenant une structure de calcul relativement simple.

