% !TEX root = main.tex

\chapter{Simulation du tambour et Assimilation de données}

Nous avons vu dans le chapitre précédent que l'assimilation dépendait de la définition d'un modèle et de son état pour définir le prior et la vraissemblance. Nous nous interessons à simuler l'écoulement dans le broyeur à boulets. La physique de l'écoulement granulaire sera dans un premier temps décrit puis différentes modélisations seront introduites. Celle-ci seront mise au regard des méthodes d'asimilation de données.

\section{Ecoulement granulaire dans un tambour en rotation}


\section{Méthode des éléments discrets (DEM)}

La méthode des éléments discrets (DEM, pour Discrete Element Method) est une technique de simulation numérique utilisée pour étudier le comportement des systèmes de particules, tels que lors du mélange broyage à l'intérieur du broyeur à boulets. Cette approche est particulièrement pertinente pour modéliser les interactions complexes entre les particules dans ces systèmes, où la dynamique individuelle de chaque particule peut avoir un impact significatif sur le processus global.

Dans un mélangeur-broyeur, les particules interagissent entre elles, avec les parois du broyeur, et avec le corps broyant. La DEM modélise chaque particule individuellement, en tenant compte de ses propriétés physiques telles que la taille, la forme, la masse, la rigidité, et le modèle de fragmentation. Les interactions incluent les forces de contact, les forces de frottement, et les forces de cohésion.

Le processus de simulation DEM dans un mélangeur-broyeur commence par la définition des propriétés des particules et des conditions initiales du système. Le mouvement de chaque particule est ensuite calculé en résolvant les équations de Newton pour le mouvement et la rotation. Ces calculs tiennent compte des forces et des moments résultant des collisions et des interactions entre particules, ainsi que de l'interaction des particules avec les parois du broyeur.

L'un des principaux avantages de la DEM est sa capacité à fournir des informations détaillées sur le mélange et le broyage des particules à l'échelle microscopique. Elle permet d'analyser comment les variations dans la configuration des particules, la vitesse de rotation du broyeur, et d'autres paramètres opérationnels influencent l'efficacité du broyage et l'homogénéité du mélange.

Cependant, l'utilisation de la DEM pour la simulation de mélangeurs-broyeurs peut être exigeante en termes de ressources informatiques, en particulier pour les systèmes avec un grand nombre de particules.

\subsection{DEM et la DA}


L'assimilation de données, lorsqu'appliquée à des systèmes simulés par la méthode des éléments discrets (DEM), se heurte à plusieurs limites importantes présentées ci-dessous en plus du temps de calculs élevé pour des systèmes de grande taille.

\subsubsection{Limites de la DEM avec les méthodes variationnelles}

Dans le cadre des méthodes variationnelles d'assimilation de données, telles que 3D-Var et 4D-Var, le principal défi est la grande dimensionnalité du problème d'optimisation.

La DEM simule le comportement de chaque particule individuellement, ce qui signifie que l'état du système comprend la position et la vitesse de chaque particule. Pour un système avec des milliers voire des millions de particules, cela conduit à un problème d'optimisation de très grande dimension.

De plus, il existe un nombre extrêmement élevé de contraintes, notamment l'interdiction de l'interpénétration des particules. Ces contraintes doivent être prises en compte pour assurer que la solution d'optimisation soit physiquement admissible.

L'application des méthodes 3D-Var et 4D-Var est donc trop exigeante d'un point de vue des temps de calcul.

\subsubsection{Limites de la DEM avec EnKF}
Pour l'EnKF, l'état estimé du système est une combinaison linéaire des états prédits par les différents membres de l'ensemble. Cependant, dans le contexte de la DEM, cette combinaison linéaire des états n'est pas nécessairement physiquement admissible. Par exemple, elle pourrait conduire à des situations où les particules s'interpénètrent ou violent d'autres lois physiques.

Un autre problème avec l'EnKF dans le contexte de la DEM est la difficulté de faire correspondre les particules entre les différents membres de l'ensemble. Chaque particule a sa propre trajectoire unique, et aligner ces trajectoires à travers les différents membres de l'ensemble pour une assimilation de données cohérente est un défi complexe. Cette difficulté est exacerbée par le nombre élevé de particules et par la nature dynamique et chaotique de leurs interactions.

\section{Méthode des points matériaux (material point method, MPM)}

La méthode des points matériaux (MPM) est une technique de simulation numérique innovante, particulièrement adaptée à la modélisation de phénomènes complexes comme ceux rencontrés dans les mélangeurs-broyeurs. Cette méthode représente un compromis entre les approches par éléments finis et par particules, offrant ainsi une modélisation efficace des interactions matérielles dans des environnements dynamiques et déformables.

Dans la MPM, l'intérieur du tambour d'un mélangeur-broyeur est conceptualisé comme un milieu continu, adoptant une perspective macroscopique. La réponse du milieu est alors représenté par une loi de comportement mécanique telle que la loi de Drucker-Prager. Cela contraste avec la méthode des éléments discrets (DEM), qui se concentrent sur les interactions particule-par-particule.

Le processus de simulation avec la MPM implique des itérations entre une grille de calculs et des particules matérielles. Chaque particule porte des informations essentielles associées au matériau, telles que la masse, le volume, et les propriétés mécaniques (variables internes, gradient de déformation...). Ces particules sont utilisées pour transférer des informations sur et hors d'une grille de calculs, où les équations de mouvement et de comportement du matériau sont résolues.

Cette approche hybride permet à la MPM de capturer efficacement les déformations importantes, les ruptures, et d'autres comportements complexes du milieu qui sont fréquents dans les opérations de mélange et de broyage. En revanche, la description fine du phénomène proposée par la DEM n'est plus disponible.

En termes de temps de calculs, la MPM est plus efficace que la DEM.

\subsection{La MPM et la DA}
La MPM offre un cadre exploitable pour mettre en place une méthode de DA.
La structure de grille sous-jacente à la MPM permet une modélisation l'utilisation des méthodes variationnelles ou des méthodes d'ensemble. La structure de particules est aussi plus flexible dans le sens où elles représentent une densité de matière : elles peuvent donc s'interpénétrer.

\subsubsection{MPM et méthodes variationnelles}
Le principal défi est de gérer la dimension du problème d'optimisation pour la 3D-Var, ainsi que construire un modèle adjoint pour la 4D-Var.
Deux pistes sont envisageables : mettre à jour les champs nodaux et les champs particulaires. Comparativement à la DEM, le nombre de variables et le nombre de contraintes sont drastiquement réduits au prix d'une représentation plus grossière.

\subsubsection{MPM et EnKF}
Pour l'EnKF, l'état estimé est une combinaison linéaire des états prédits. Dans le contexte de la MPM, cela signifie que la mise à jour de l'état peut être directement effectuée sur la grille de calcul, plutôt que sur les particules individuelles. Cette approche réduit la complexité des calculs et facilite l'assimilation de données dans des systèmes à grande échelle.

Cependant, l'intégration de la MPM avec l'EnKF soulève plusieurs questions importantes :

1. **Transfert d'Informations de Particules à la Grille** : La première question concerne le transfert efficace des informations des particules vers la grille. Cela nécessite des algorithmes précis pour garantir que les informations pertinentes sur les propriétés des matériaux, telles que la densité, la contrainte, et le gradient de déformation déformation, sont correctement représentés sur la grille de calcul.

2. **Remaillage de Particules pour Représenter l'État Mécanique** : Une autre question clé est de savoir comment effectuer un remaillage des particules pour représenter fidèlement l'état mécanique du système après assimilation. Cela est crucial pour maintenir la cohérence et l'exactitude du modèle MPM, en particulier après des mises à jour successives de l'état du système.

\section{Vortex-In-Cell : un modèle grille-particules simplifié}

La méthode Vortex-In-Cell (VIC) est une technique de simulation numérique, considérée ici comme une version simplifiée de la MPM. Cette méthode est particulièrement utilisée pour simuler les écoulements de fluides et les dynamiques de vortex. La VIC combine l'utilisation d'une grille de calcul avec des particules pour modéliser le mouvement des fluides.

Dans la méthode VIC, la grille sert de structure pour effectuer les calculs numériques liés à la dynamique des fluides, tandis que les particules, dispersées dans le fluide, transportent une quantité scalaire, la vorticité. Cette approche diffère de la MPM dans le sens où les particules portent des informations complexes sur les propriétés mécaniques du matériau.

Le processus de simulation avec la VIC implique d'abord le calcul des champs de vitesse sur la grille. Ces champs sont ensuite utilisés pour déplacer les particules dans le fluide, qui à leur tour transportent la vorticité à travers le domaine de simulation. L'avantage de cette méthode est sa capacité à modéliser avec précision les phénomènes complexes d'écoulement de fluides, tels que la formation et l'évolution de tourbillons, tout en maintenant une structure de calcul relativement simple.