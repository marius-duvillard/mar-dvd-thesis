\documentclass[aspectratio=169]{beamer} %% for 16:9 use this line
%%\documentclass{beamer} %% For 4:3 ratio use this line
\usepackage[utf8]{inputenc}
\usepackage[T1]{fontenc}
\usepackage{lipsum} 
\usepackage{bm}
\usepackage{graphicx}
\usepackage{animate}
\usepackage{amsmath}
\usepackage{subcaption}
\usepackage{tikz}
\captionsetup[figure]{font=footnotesize}
% import citation package
\usepackage[backend=biber, style=authoryear]{biblatex}
\addbibresource{./biblio.bib}
\AtBeginBibliography{\small}
\DeclareMathOperator*{\argmin}{\arg\!\min}
% footnote without number
\newcommand\blfootnote[1]{
\begingroup
\renewcommand\thefootnote{}\footnote{#1}
\addtocounter{footnote}{-1}
\endgroup
}

\usetheme{CEA2023}
\setlength{\columnsep}{0.05cm}
\title[ECCOMAS2024 - Data Assimilation for Meshless Simulation] %optional
{Ensemble Data Assimilation Method Applied to Meshless Simulations}
\subtitle{ECCOMAS 2024 - Lisbon}
\date[06-03-2024] %optional
{June the 3\textsuperscript{rd} 2024}
\author[M. Duvillard] %optionam
{Marius Duvillard \inst{1} \inst{2} \texttt{(\small marius.duvillard\myat cea.fr)} \\
Olivier Le Maître \inst{2} \inst{3} \texttt{(\small olivier.le-maitre\myat polytechnique.edu)} \\
Loïc Giraldi \inst{1} \\
}

\institute[short-inst]{
 \inst{1} CEA DES/IRESNE/DEC/SESC Cadarache 
 \inst{2} Centre de Mathématiques Appliquées, Ecole Polytechnique 
 \inst{3} CNRS, Inria
}

% uncomment the following lines if you do not want dedicated outlines before
% each section
\AtBeginSection{}

% use your thanks-message in the last frame
\setvalue{\ThxMessage}{Thanks! Any questions?}
% Change the logo 
\titlegraphic{logos/LOGO_CEA_ORIGINAL.png}

% introduce another logo for a second author or affiliation
% secondlogo applies to the first page only
\setvalue{\secondlogo}{logos/Polytechnique_logo.pdf}

\begin{document}

% info: the plain removes the footline from the titlepage; noframenumbering
% neglects it from the total count of the slides
\begin{frame}[decorated] %Decorated bring the logo and corner
    \titlepage
\end{frame}

\begin{frame}[righttransition]{Outline} % or Table of Contents
    \tableofcontents
\end{frame}

\section{Introduction}
\subsection{Contexte}
\subsection{Bro-IA-ge}
\subsection{Assimilation de données}
\subsection{Méthodes particulaire}

\section{Assimilation appliquée à des méthodes sans maillage}
\subsection{Mise à jour à la EnKF}
\subsection{Correction d'intensité}
\subsection{Correction de la position}
\subsection{Correction conjointe}
\section{Application}
\subsection{Méthode vortex}

\section{Conclusion et perspective}
\subsection{Et le tambour ? la DEM ?}
% Présenter le cas de MPM
% Présenter le cas DEM
\closingframe

\begin{frame}[allowframebreaks, noframenumbering]
    \frametitle{References}
    \printbibliography % Print the bibliography
\end{frame}

% \section*{Addendum}
% \begin{frame}{EnKF update}
%   \emph{Be successful with your presentation!}
% \end{frame}

\end{document}