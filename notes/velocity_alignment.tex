\documentclass{article}
\usepackage{graphicx} % Pour inclure des images
\usepackage{amsmath} % Pour les environnements mathématiques
\usepackage{amssymb} % Pour les symboles mathématiques
\usepackage{bm} % Pour les symboles gras en mathématiques
\usepackage{geometry}
\usepackage[french]{babel}
\usepackage[T1]{fontenc}
\usepackage{hyperref}

\DeclareMathOperator*{\argmax}{arg\,max}
\DeclareMathOperator*{\argmin}{arg\,min}

\newcommand{\bv}{\bm{v}}
\newcommand{\norm}[1]{\left\lVert #1 \right\rVert}


\graphicspath{ {images} }

\bibliographystyle{plain}
\begin{document}

\title{Alignement des champs de vorticité}
\author{Marius Duvillard}
\date{\today}
\maketitle


\section{Définition du problème}
On suppose que les champs de vorticité ont été mal aligné. Pour corriger cette erreur, on souhaite appliquer une correction qui respecte les contraintes d'un écoulement incompressible. Pour cela, on introduit un transformation $\Phi$ qui doit corriger l'ébauche.
On réécrit la loi de Bayes avec cette information

\begin{equation*}
    p(\omega,\Phi \mid y) \propto p(y \mid \omega, \Phi) p(\omega \mid \Phi) p(\Phi)
\end{equation*}

Pour les deux premiers termes on retrouve facilement la vraissemblance conditionnée par l'ébauche déformée c'est à dire $\Phi(\omega)$ qui est appliqué sur les coordonnées lagrangienne du champ.
Le prior est également défini sur le champ déformé. Enfin il faut définir un prior pour $\Phi$. Ce terme est assez arbitraire. On veut simplement qu'il vérifie la condition d'un écoulement incompressible.

% \section{Correction dans le span des fonctions de courant des membres}

% On suppose que l'on veut résoudre le problème dans le span des fonctions courants courant de l'ensemble.

% Pour cela il suffit de supposer que la fonction courant de la correction $\hat \psi$ admet le prior $\mathcal N (\psi_0,  P_\psi)$ avec $\psi_0 = 0$ et $P_\phi =\frac{1}{N - 1}\sum (\psi_i - \bar \psi) (\psi_i - \bar \psi)^T$

% \begin{equation*}
%     \mathcal L(\psi) = \norm{d_v - h_\psi(\psi; \omega_i)}^2_{R^{-1}} + \norm{\psi}^2_{P^{-1}_{\psi}}
% \end{equation*}

\section{Correction dans le span des champs de vitesse des membres}

On cherche une transformation définie par l'intégration de la position d'un jeu de particule $P = \{(x_p, \Gamma_p)\}$ par un champs de vitesse $u$ tel que

\begin{eqnarray*}
    x(t;u) &=& x_p + \int_{0}^{t} u(x(t')) \mathrm dt' \\
    \tilde x_p &=& x(1; u).
\end{eqnarray*}

On souhaite identifier ce champ de vitesse tel que

\begin{equation*}
    u = \argmin_{u \in \mathcal{V}} \norm{u^{obs} - h_u(\{(x_p, \Gamma_p)\}; x_{obs}, u)}^2_{R^{-1}}.
\end{equation*}où $u^{obs}$ sont les vitesse observées en $x_{obs}$, $h_u$ l'opérateur d'observation après application du champ de vitesse $u$ pour déplacer les particules.

Le champ $u$ est défini dans un espace vectoriel $\mathcal{V}$ dans lequel il admet une décomposition.

Dans un premier temps on choisi $\mathcal{V} = \text{Span}(u_m)$ où $u_m$ est le champ de vitesse du $m$-ème membre.

Ce choix est cohérent car par CL, il permet de définir une transformation à divergence nulle.
De plus, il est cohérent avec la philosophie du filtre de Kalman d'utiliser la distribution des membres.

Ainsi le champ $u$ se réécrit comme $u = \sum_m a_m u_m$.

On peut réécrire la fonction objectif en fonction du vecteur $a \in \mathbb{R}^N$

\begin{equation*}
    \mathcal L_a =  \norm{u^{obs} - h_a(\{(x_p, \Gamma_p)\}; x_{obs}, a)}^2_{R^{-1}}.
\end{equation*}

Il convient de définir l'opérateur $h$ comme l'observation du champ de vitesse aux coordonnées $x_{obs}$ après déplacement des particules selon $u$

\begin{equation*}
    h_a(\{(x_p, \Gamma_p)\}; x_{obs}, a) = h(\{(\tilde x_p, \Gamma_p)\}; x_{obs}) = U_{obs}(\tilde x_p, \Gamma_p).
\end{equation*}

Du fait de la non-linéarité du problème (non linéarité de $x(t;a)$ par rapport à $a$), le problème n'est pas convexe. On se propose de résoudre le problème à l'aide d'un algoritme de minimisation NL. Pour cela on souhaite évaluer le gradient de $\mathcal L_a$.

Dans un premier temps, ce vecteur est évalué par approximation de la dérivée directionnelle

\begin{equation*}
    \frac{\partial \mathcal L_a}{\partial a_i}(a) = D_{u_i} \mathcal L_a(a) \approx \frac{\mathcal L_a(a + \delta a_i u_i) - \mathcal L_a(a)}{\delta a_i}
\end{equation*}

Connaissant $\mathcal L_a$ et son gradient, on peut appliquer des algorithmes de minimisation NL.
% \section{Correction dans le span des champs de vitesse des membres 2}

% Après simplification, on voudrait arriver à la forme suivante du problème: Définir la transformation comme l'intégration d'un champ de vitesse combinaison des champs de vitesse des membres $v_i$ à l'instant $t$.

% Pour cela, on cherche un champ de vitesse $v$ qui aura comme distribution a priori $\mathcal N (v_0,  P_v)$ avec $\bv_0 = 0$ et $P_v = \frac{1}{N - 1}\sum (v_i - \bar v) (v_i - \bar v)^T$

% Le problème à minimiser est alors pour le membre $i$

% \begin{equation*}
%     \mathcal L(v) = \norm{d_v - h_v(v; \omega_i)}^2_{R^{-1}} + \norm{v - \bar v}^2_{P_{v}^{-1}}
% \end{equation*}

% En fait on peut chercher à réécrire le problème dans le Span de l'ensemble et ainsi trouver la combinaison linéaire en $v_i$.


% On note $V \in \mathbb R^{d \times N}$ l'ensemble des champs de vitesse où $d$ est la dimension de $v$ et $N$ la taille de l'ensemble. On cherche alors $b \in \mathcal R^{N}$ tel que

% \begin{equation*}
%     v = v_i + Vb
% \end{equation*}

% Tel que $\Phi$ devient

% \begin{equation*}
%     \Phi(x, y) = (x, y) + dt v_i(x, y) + dt \sum_j b_j v_j(x, y)
% \end{equation*}

% On a bien $\Phi$ qui est linéaire avec $b$

% \begin{equation*}
%     \norm{v - \bar v}^2_{P^{-1}} = \norm{b}^2_2
% \end{equation*}

% et d'autre part on doit linéarise $h$ par rapport à $v$.

% Pour cela,  étudions $h$. C'est la mesure de la vitesse après application intégration de l'advection de $\omega_i$ c'est à dire faire

% \begin{equation*}
%     h_v(v; \omega_i) = h_\omega(\Phi(\omega_i)) = h_{\omega}(\omega((x, y)+ dt v_i(x, y) + dt \sum_j b_j v_j(x, y)))
% \end{equation*}

% La difficulté maintenant est de déduire du déplacement de $\omega$, la variation de vitesse

% Pour cela on doit étudier la linéarisation de l'équation de la fonction de courant.

% On note $\hat \omega$ la nouvelle vorticité

% \begin{eqnarray*}
%     \Delta \psi  &=& \hat \omega \\
%     &=& \omega((x, y)+ dt v_i(x, y) + dt \sum_j b_j v_j(x, y)) \\
%     &=&
% \end{eqnarray*}


% \begin{eqnarray*}
%     h(v; \omega_i) = v
% \end{eqnarray*}


\end{document}