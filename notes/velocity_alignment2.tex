\documentclass{article}
\usepackage{graphicx} % Pour inclure des images
\usepackage{amsmath} % Pour les environnements mathématiques
\usepackage{amssymb} % Pour les symboles mathématiques
\usepackage{bm} % Pour les symboles gras en mathématiques
\usepackage{geometry}
\usepackage[french]{babel}
\usepackage[T1]{fontenc}
\usepackage{hyperref}

\DeclareMathOperator*{\argmax}{arg\,max}
\DeclareMathOperator*{\argmin}{arg\,min}

\newcommand{\bv}{\bm{v}}
\newcommand{\norm}[1]{\left\lVert #1 \right\rVert}


\graphicspath{ {images} }

\bibliographystyle{plain}
\begin{document}

\title{Alignement par application d'un champ de vitesse}
\author{Marius Duvillard}
\date{\today}
\maketitle

\section{Définition du problème}

On suppose $\omega$ un champ de vorticité défini sur $\Omega$. On suppose que le champ est incertain à cause d'une erreur d'alignement.
Nous supposons que cette erreur est issue d'une erreur dans l'intégration du champ de vitesse lors de l'étape de propagation.
Lors de l'étape d'analyse, nous souhaitons réduire cette erreur de position. Pour cela, nous introduisons une variable de contrôle au travers d'un déplacement défini par une transformation $\Phi(\bm u)$, paramétrée par un champ de vitessse de correction $\bm u$ de telle sorte que pour tout point $\bm z \in Omega$

\begin{equation*}
    \Phi(\bm x, \bm u) = X(u) = x + \int_{0}^{1} u(x(t)) \mathrm dt.
\end{equation*}

Puisque l'erreur est supposée issue de perturbation dans la position du champ, la correction doit préserver la propriété d'incompressibilité de l'écoulement. Pour cela, nous supposons que $\bm u$ est un champ de vitesse à divergence nulle. Pour des observations, et un champs de vorticité donnée, nous cherchons à déterminer $\bm u$ la distribution a posteriori $p(\bm u \mid \omega, y) \propto p(y \mid \bm u, \omega) p(u)$.

\textbf{transition vers la forme variationnelle d'ensemble}

Ainsi, on obtient un ensemble de $N$ fonction coût à minimiser

\begin{equation*}
    \mathcal L_i (\bm u) = \frac{1}{2}\norm{y - h(\omega, \Phi(u))}^2_{R^{-1}} + \frac \lambda 2 \norm{u}^2_{P^{\dag}},
\end{equation*}où $P^{\dag}$ est l'opérateur pseudo-inverse de

\end{document}
